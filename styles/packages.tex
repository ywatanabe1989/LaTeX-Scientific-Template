%% preamble
\usepackage[english]{babel}
\usepackage[table]{xcolor} % For coloring tables
\usepackage{booktabs} % For professional quality tables
\usepackage{colortbl} % For coloring cells in tables
\usepackage{amsmath, amssymb} % For mathematical symbols and environments
\usepackage{amsthm} % For theorem-like environments
\usepackage{lipsum} % just for sample text
\usepackage{natbib}
\usepackage{graphicx}
\usepackage{indentfirst}
\usepackage{bashful}
% for figures
\usepackage[margin=10pt,font=small,labelfont=bf,labelsep=endash]{caption}
\usepackage{graphicx}
\usepackage{calc}
% for tables
\usepackage{xlsx2csv}
\usepackage{csv2latex}
\usepackage[T1]{fontenc} % [REVISED]
\usepackage[utf8]{inputenc} % [REVISED]
\usepackage{hyperref}
\usepackage{accsupp}

%% Line numbers
\linespread{1.1}
% \linenumbers

% Use accsupp package to make line numbers non-selectable/non-copyable in PDF
\renewcommand\LineNumber{%
  \BeginAccSupp{method=escape,ActualText={}}%
  \thelinenumber\ %
  \EndAccSupp{}%
}

% Ensure that listings package is loaded and configured if you want to include code listings
\usepackage{listings}
\lstset{
  numbers=left,
  numberstyle=\tiny,
  stepnumber=1,
  numbersep=5pt,
  basicstyle=\ttfamily,
  frame=tb,
  framesep=5pt,
  framexleftmargin=15pt,
  backgroundcolor=\color{gray!10}
}

\makeatletter
\def\lst@PlaceNumber{%
  \llap{\normalfont
    \pdfliteral direct{/Span<</ActualText()>>BDC}%
    \lst@numberstyle{\thelstnumber}%
    \pdfliteral direct{EMC}%
    \kern\lst@numbersep%
  }%
}
\makeatother