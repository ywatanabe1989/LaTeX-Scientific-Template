\begin{abstract}
\pdfbookmark[1]{Abstract}{abstract}
Working memory (WM) is integral to numerous cognitive functions, yet the precise neural mechanisms involved are yet to be fully understood. While the hippocampus and sharp-wave ripple complexes (SWRs) --- brief, synchronous neural events that occur in the hippocampus --- are recognized for their role in memory consolidation and retrieval, the nature of their connection with WM tasks remains to be clarified. Our present study posits that the interaction between hippocampal multiunit activity patterns and SWRs may produce distinct dynamics during WM tasks. We analyzed a dataset that included intracranial electroencephalogram recordings from the medial temporal lobe (MTL) of nine epilepsy patients during an eight-second Sternberg task. We used Gaussian-process factor analysis to extract low-dimensional neural representations, referred to as 'trajectories', in MTL regions during the task. Our data indicate that the hippocampus displays the greatest variability in neural trajectory relative to the entorhinal cortex and amygdala. Moreover, the divergence in trajectories, as measured between encoding and retrieval phases, appears to be contingent on memory load. Remarkably, hippocampal trajectories oscillate during the retrieval phase, showing task-dependent shifts between encoding and retrieval states both during baseline and SWR events. We observed this fluctuation transition from encoding to retrieval states coinciding with the occurrence of SWRs. These findings underscore the fundamental role of the hippocampus in WM tasks and suggest a new hypothesis: the hippocampus transitions from an encoding to a retrieval state during SWRs.
\end{abstract}