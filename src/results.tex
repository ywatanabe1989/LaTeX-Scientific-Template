
\section{Results}
\subsection{iEEG recording and neural trajectory in MTL regions during a Sternberg task}
We utilized a publicly available dataset \cite{boran_dataset_2020} for our analysis, which comprises LFP signals (Figure 1A) from MTL regions (Table~\ref{tab:01}1) recorded during a revised Sternberg task. SWR$^+$ candidates were detected in all hippocampal regions derived from the LFP signals, filtered by the ripple band (80--140 Hz) (Figure 1B). SWR$^-$ candidates were outlined at the same timestamps as SWR$^+$ candidates but scrambled across various trials (Figure 1). The dataset also encompasses the multiunit spikes (Figure 1C), which were pinpointed through the utilization of a spike sorting algorithm \cite{niediek_reliable_2016}. Using the 50-ms binned multiunit activity without overlaps, we applied GPFA \cite{yu_gaussian-process_2009} to reveal the neural trajectory (or factors) of MTL regions per session and region (Figure 1D). Each factor was z-normalized by session and region (an example is session \#2 in AHL of subject \#1). The Euclidean distance from the origin ($O$) was subsequently calculated (Figure 1E).

\subsection{Hippocampal neural trajectory correlation with a Sternberg task}
Figure 2A illustrates the median neural trajectories of 50 trials as point clouds within the three principal factor spaces. By using the elbow method, we determined that the optimal embedding dimension for the GPFA model was three (Figure 2B). The trajectory distance from the origin ($O$) ($\mathrm{\lVert g_{F} \rVert}$, $\mathrm{\lVert g_{E} \rVert}$, $\mathrm{\lVert g_{M} \rVert}$, and $\mathrm{\lVert g_{R} \rVert}$) was found to be greater in the hippocampus than in the EC and amygdala (Figure 2C \& D).

Similarly, distances between geometric medians of the four phases, $\mathrm{\lVert g_{F}g_{E} \rVert}$, $\mathrm{\lVert g_{F}g_{M} \rVert}$, $\mathrm{\lVert g_{F}g_{R} \rVert}$, $\mathrm{\lVert g_{E}g_{M} \rVert}$, $\mathrm{\lVert g_{E}g_{R} \rVert}$, and $\mathrm{\lVert g_{M}g_{R} \rVert}$, were computed. It was observed that the hippocampus exhibited larger distances among the phases compared to the EC and amygdala. 

\subsection{Memory load-dependent neural trajectory distance between the encoding and retrieval states in the hippocampus}
Considering the memory load of the Sternberg task, we noted that the correct trial rate and set size (equal to the number of alphabet letters to be encoded) shared a negative correlation (Figure 3A). Likewise, a positive correlation was found between response time and set size (Figure 3B). Additionally, the set size and the trajectory distance between the encoding and retrieval phases ($\mathrm{log_{10}\lVert g_{E}g_{R} \rVert}$) demonstrated a positive relationship (Figure 3C). However, distances between other phase combinations showed no significant correlations (Figures 3D \& S2).

\subsection{Detection of hippocampal SWR from putative CA1 regions}
To improve the accuracy of recording sites and SWR detection, we endeavored to estimate electrodes in the CA1 regions of the hippocampus by observing distinct multiunit spike patterns during SWR events. For each session and hippocampal region, SWR$^+$/SWR$^-$ candidates were embedded into a two-dimensional space using UMAP (Figure 4A). The silhouette score was computed as a measure of the quality of clustering (Figure 4B \& Table~\ref{tab:02}). Recording sites with an average silhouette score across sessions more substantial than 0.6 were categorised as putative CA1 regions \cite{mcinnes_umap_2018, rousseeuw_silhouettes_1987}. As such, we identified five putative CA1 regions, four of which were not previously labeled seizure onset zones (Table~\ref{tab:01}).

Next, we labelled SWR$^+$/SWR$^-$ candidates within these putative CA1 regions as SWR$^+$ and SWR$^-$, respectively. Both SWR$^+$ and SWR$^-$ exhibited the same duration. A significant increase in SWR$^+$ incidence emerged during the first 400 ms of the retrieval phase. Moreover, the peak ripple band amplitude of SWR$^+$ was higher than that of SWR$^-$.

\subsection{Transient change in neural trajectory in the hippocampus during SWR}
We analyzed the \textit{distances} of the trajectory from origin ($O$) during SWR events in both encoding and retrieval phases (Figure 5A). Noting the increment in distance during SWR (Figure 5A), we grouped each SWR into three states: pre-, mid-, and post-SWR. 

\subsection{Visualization of hippocampal neural trajectory during SWR in two-dimensional spaces}
Based on our observations of the neural trajectory 'jump' during SWR (Figure 5), we visualized the three-dimensional trajectories of pre-, mid-, and post-SWR events during the encoding and retrieval phases (Figure 6). For this visualization, we positioned $\mathrm{g_{E}}$ at the origin (0, 0) and $\mathrm{g_{R}}$ at the coordinate ($\mathrm{\lVert g_{E}g_{R} \rVert}$, 0) in two-dimensional spaces by linearly aligning peri-SWR trajectories. 

\subsection{Fluctuating hippocampal neural trajectories between encoding and retrieval states}
Subsequently, we examined the trajectory \textit{directions} based on $\overrightarrow{\mathrm{g_{E}g_{R}}}$. Directions of SWRs were identified by the neural trajectory at $-250$ ms and $+250$ ms from their center (i.e., $\overrightarrow{\mathrm{eSWR^+}}$). From these data, we computed the density of $\overrightarrow{\mathrm{eSWR}} \cdot \overrightarrow{\mathrm{g_{E}g_{R}}}$, $\overrightarrow{\mathrm{rSWR}} \cdot \overrightarrow{\mathrm{g_{E}g_{R}}}$, and $\overrightarrow{\mathrm{eSWR}} \cdot \overrightarrow{\mathrm{rSWR}}$ (Figure 7A--D).