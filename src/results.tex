\section{Results}
\subsection{iEEG Recording and Neural Trajectory in MTL Regions during a Sternberg Task}
We conducted our analysis on a publicly available dataset \cite{boran_dataset_2020} consisting of LFP signals (Figure 1A) from MTL regions (Table~\ref{tab:01}1), recorded during an adapted Sternberg task. Ripple wave candidates with and without sharp wave ripples (SWR$^+$ and SWR$^-$, respectively) were detected in all hippocampal regions yielded from the filtered LFP signals in the ripple band (80--140 Hz) (Figure 1B). The SWR$^-$ candidates were identified at the same timestamps as their SWR$^+$ counterparts but scrambled across various trials (Figure 1). The dataset also captured multiunit spikes (Figure 1C), which were identified using a spike sorting algorithm \cite{niediek_reliable_2016}. By employing the 50-ms binned multiunit activity, and excluding overlaps, we applied Gaussian-process factor analysis (GPFA) \cite{yu_gaussian-process_2009} to reveal the neural trajectory (or factors) of MTL regions per session and region (Figure 1D). Each factor was z-normalized by session and region (as illustrated in session \#2 of AHL in subject \#1). Following this, the Euclidean distance from the origin ($O$) was computed (Figure 1E).

\subsection{Correlation of Hippocampal Neural Trajectory with a Sternberg Task}
Figure 2A features the median neural trajectories of 50 trials as point clouds within the three key factor spaces. With the elbow method, we deduced that the optimal embedding dimension for the GPFA model was three (Figure 2B). The trajectory distance from the origin ($O$) ($\mathrm{\lVert g_{F} \rVert}$, $\mathrm{\lVert g_{E} \rVert}$, $\mathrm{\lVert g_{M} \rVert}$, and $\mathrm{\lVert g_{R} \rVert}$) was discovered to be greater in the hippocampus than in the EC and amygdala (Figure 2C \& D).

Additionally, distances among geometric medians of the four phases, $\mathrm{\lVert g_{F}g_{E} \rVert}$, $\mathrm{\lVert g_{F}g_{M} \rVert}$, $\mathrm{\lVert g_{F}g_{R} \rVert}$, $\mathrm{\lVert g_{E}g_{M} \rVert}$, $\mathrm{\lVert g_{E}g_{R} \rVert}$, and $\mathrm{\lVert g_{M}g_{R} \rVert}$, were calculated, with the hippocampus displaying larger distances compared to the EC and amygdala.

\subsection{Memory-load-dependent Neural Trajectory Distance between the Encoding and Retrieval States in the Hippocampus}
Given the memory load of the Sternberg task, we found a negative correlation between the correct trial rate and set size (equivalent to the number of alphabetical letters to be encoded) (Figure 3A). A positive correlation was also observed between response time and set size (Figure 3B), as well as between set size and the trajectory distance between the encoding and retrieval phases ($\mathrm{log_{10}\lVert g_{E}g_{R} \rVert}$) (Figure 3C). However, no significant correlations were found between distances of other phase combinations (Figures 3D \& S2).

\subsection{Detection of Hippocampal SWR from Putative CA1 Regions}
To enhance the precision of recording sites and SWR detection, we aimed to estimate electrodes in the CA1 regions of the hippocampus, using distinct multiunit spike patterns during SWR events. For each session and specific hippocampal region, SWR$^+$/SWR$^-$ candidates were embedded into a two-dimensional space using UMAP (Figure 4A). We then computed the silhouette score to measure the quality of clustering (Figure 4B \& Table~\ref{tab:02}). Recording sites with an average silhouette score across sessions greater than 0.6 were defined as putative CA1 regions \cite{mcinnes_umap_2018, rousseeuw_silhouettes_1987}. Consequently, we identified five putative CA1 regions, four of which were not previously marked as seizure onset zones (Table~\ref{tab:01}).

Subsequently, we classified SWR$^+$/SWR$^-$ candidates within these putative CA1 regions as SWR$^+$ and SWR$^-$, respectively. Both SWR$^+$ and SWR$^-$ exhibited the same duration. However, SWR$^+$ incidence significantly increased during the initial 400 ms of the retrieval phase, and the peak ripple band amplitude of SWR$^+$ was also higher than that of SWR$^-$.

\subsection{Transient Change in Neural Trajectory in the Hippocampus during SWR}
We analyzed the \textit{distances} of the trajectory from the origin ($O$) during SWR events in both encoding and retrieval phases (Figure 5A). Observing the increase in distance during SWR (Figure 5A), we classified each SWR into three states: pre-, mid-, and post-SWR. 

\subsection{Visualization of Hippocampal Neural Trajectory during SWR in Two-Dimensional Spaces}
Observing the neural trajectory 'jump' during SWR (Figure 5), we visualized the three-dimensional trajectories of pre-, mid-, and post-SWR events during the encoding and retrieval phases (Figure 6). For this visualization, we arranged $\mathrm{g_{E}}$ at the origin (0, 0) and $\mathrm{g_{R}}$ at the coordinate ($\mathrm{\lVert g_{E}g_{R} \rVert}$, 0) in two-dimensional spaces by linearly aligning the trajectories surrounding SWR events. 

\subsection{Fluctuating Hippocampal Neural Trajectories between Encoding and Retrieval States}
Finally, we examined the trajectory \textit{directions} based on $\overrightarrow{\mathrm{g_{E}g_{R}}}$, identifying directions of SWRs by the neural trajectory at $-250$ ms and $+250$ ms from their center, denoted as $\overrightarrow{\mathrm{eSWR^+}}$. From these data, we computed the density of $\overrightarrow{\mathrm{eSWR}} \cdot \overrightarrow{\mathrm{g_{E}g_{R}}}$, $\overrightarrow{\mathrm{rSWR}} \cdot \overrightarrow{\mathrm{g_{E}g_{R}}}$, and $\overrightarrow{\mathrm{eSWR}} \cdot \overrightarrow{\mathrm{rSWR}}$ (Figure 7A--D).