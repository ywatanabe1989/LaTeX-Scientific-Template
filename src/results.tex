\section{Results}
\subsection{iEEG recording and neural trajectory in MTL regions during a Sternberg task}
Our analysis employed a publicly accessible dataset \cite{boran_dataset_2020}, which comprises LFP signals (Figure~\ref{fig:01}A) from MTL regions (Table~\ref{tab:01}) recorded during the execution of a modified Sternberg task. We extracted SWR$^+$ candidates from LFP signals that were filtered in the 80--140 Hz ripple band (Figure~\ref{fig:01}B), originating in all hippocampal regions (refer to Methods section). Meanwhile, SWR$^-$ candidates, control events for SWR$^+$ candidates, were defined at the same timestamps but distributed across different trials (Figure~\ref{fig:01}). The dataset also encompassed multiunit spikes (Figure~\ref{fig:01}C), recognized via a spike sorting algorithm \cite{niediek_reliable_2016}. Employing GPFA \cite{yu_gaussian-process_2009}, we applied this to 50-ms windows of binned multiunit activity without overlaps to determine the neural trajectories, or factors, of MTL regions by session and region (Figure~\ref{fig:01}D). We normalized each factor per session and region, for instance, session \#2 in AHL of subject \#1. We then calculated the Euclidean distance from the origin ($O$) (Figure~\ref{fig:01}E).

\subsection{Hippocampal neural trajectory correlation with a Sternberg task}
Figure~\ref{fig:02}A exhibits a distribution of median neural trajectories, comprising 50 trials, within the three main factor spaces. Utilizing the elbow method, we established the optimal embedding dimension for the GPFA model as three (Figure~\ref{fig:02}B). The trajectory distance from the origin ($O$) --- represented as $\mathrm{\lVert g_{F} \rVert}$, $\mathrm{\lVert g_{E} \rVert}$, $\mathrm{\lVert g_{M} \rVert}$, and $\mathrm{\lVert g_{R} \rVert}$ --- in the hippocampus surpassed the corresponding distances in the EC and amygdala (Figure~\ref{fig:02}C \& D).\footnote{Hippocampus: Distance = 1.11 [1.01], median [IQR], \textit{n} = 195,681 timepoints; EC: Distance = 0.94 [1.10], median [IQR], \textit{n} = 133,761 timepoints; Amygdala: Distance = 0.78 [0.88], median [IQR], \textit{n} = 165,281 timepoints.}
\\
\indent
Similarly, we computed the distances between the geometric medians of four phases, namely $\mathrm{\lVert g_{F}g_{E} \rVert}$, $\mathrm{\lVert g_{F}g_{M} \rVert}$, $\mathrm{\lVert g_{F}g_{R} \rVert}$, $\mathrm{\lVert g_{E}g_{M} \rVert}$, $\mathrm{\lVert g_{E}g_{R} \rVert}$, and $\mathrm{\lVert g_{M}g_{R} \rVert}$. The hippocampus showed larger distances between phases than those in the EC and amygdala. \footnote{Hippocampus: Distance = 0.60 [0.70], median [IQR], \textit{n} = 8,772 combinations; EC: Distance = 0.28 [0.52], median [IQR], \textit{n} = 5,017 combinations (\textit{p} $<$ 0.01; Brunner--Munzel test); Amygdala: Distance = 0.24 [0.42], median [IQR], \textit{n} = 7,466 combinations (\textit{p} $<$ 0.01; Brunner--Munzel test).}

\subsection{Memory-load-dependent neural trajectory distance between encoding and retrieval states in the hippocampus}
Regarding memory load in the Sternberg task, we observed a negative correlation between the correct rate of trials and the set size, which denotes the number of letters to be encoded (Figure~\ref{fig:03}A).\footnote{Correct rate: set size four (0.99 \textpm 0.11, mean \textpm SD; \textit{n} = 333 trials) vs. set size six (0.93 \textpm 0.26; \textit{n} = 278 trials; \textit{p} $<$ 0.001, Brunner--Munzel test with Bonferroni correction) and set size eight (0.87 \textpm 0.34; \textit{n} = 275 trials; \textit{p} $<$ 0.05; Brunner--Munzel test with Bonferroni correction). Generally, \textit{p} $<$ 0.001 for Kruskal--Wallis test; correlation coefficient = - 0.20, \textit{p} $<$ 0.001.} Concomitantly, a positive correlation was noted between the response time and set size (Figure~\ref{fig:03}B).\footnote{Response time: set size four (1.26 \textpm 0.45 s; \textit{n} = 333 trials) vs. set size six (1.53 \textpm 0.91 s; \textit{n} = 278 trials) and set size eight (1.66 \textpm 0.80 s; \textit{n} = 275 trials). All comparisons \textit{p} $<$ 0.001, Brunner--Munzel test with Bonferroni correction; \textit{p} $<$ 0.001 for Kruskal--Wallis test; correlation coefficient = 0.22, \textit{p} $<$ 0.001}.
\\
\indent
Next, we discovered a positive correlation between set size and the trajectory distance separating the encoding and retrieval phases ($\mathrm{log_{10}\lVert g_{E}g_{R} \rVert}$) (Figure~\ref{fig:03}C).\footnote{Correlation between set size and $\mathrm{log_{10}(\lVert g_{E}g_{R} \rVert}$): correlation coefficient = 0.05, \textit{p} $<$ 0.001. Specific values: $\mathrm{\lVert g_{E}g_{R} \rVert}$ = 0.54 [0.70] for set size four, \textit{n} = 447; $\mathrm{\lVert g_{E}g_{R} \rVert}$ = 0.58 [0.66] for set size six, \textit{n} = 381; $\mathrm{\lVert g_{E}g_{R} \rVert}$ = 0.61 [0.63] for set size eight, \textit{n} = 395.}. However, distances between other phase combinations did not highlight statistically significant correlations (Figures~\ref{fig:03}D and \ref{fig:s02}).

\subsection{Detection of hippocampal SWR from putative CA1 regions}
To enhance the precision of recording sites and SWR detection, we approximated the electrode placements in the CA1 regions of the hippocampus using distinguished multiunit spike patterns during SWR events. SWR$^+$/SWR$^-$ candidates from each session and hippocampal region were embedded in two-dimensional space using UMAP (Figure~\ref{fig:04}A).\footnote{Consider the AHL in session \#1 of subject \#1 as a case in point.} With the silhouette score as a quality metric for clustering (Figure~\ref{fig:04}B and Table~\ref{tab:02}), recording sites demonstrating an average silhouette score exceeding 0.6 across all sessions were identified as putative CA1 regions.\footnote{The identified regions were the AHL of subject \#1, AHR of subject \#3, PHL of subject \#4, AHL of subject \#6, and AHR of subject \#9.} (Tables~\ref{tab:02} and \ref{tab:03}). We identified five putative CA1 regions, four of which were not indicated as seizure onset zones (Table~\ref{tab:01}).
\\
\indent
Subsequently, SWR$^+$/SWR$^-$ candidates within these putative CA1 regions were labeled as SWR$^+$ and SWR$^-$, respectively\footnote{These definitions produced equal counts for both categories: SWR$^+$ (\textit{n} = 1,170) and SWR$^-$ (\textit{n} = 1,170).} (Table~\ref{tab:03}). Both SWR$^+$ and SWR$^-$ manifested identical durations \footnote{These definitions result in equal durations for both categories: SWR$^+$ (93.0 [65.4] ms) and SWR$^-$ (93.0 [65.4] ms).} due to their definitions and followed a log-normal distribution (Figure~\ref{fig:04}C). During the initial 400 ms of the retrieval phase, an increase in SWR$^+$ incidence was found\footnote{SWR$^+$ increased against the bootstrap sample; 95th percentile = 0.42 [Hz]; \textit{p} $<$ 0.05.} (Figure~\ref{fig:04}D). The peak ripple band amplitude of SWR$^+$ surpassed that of SWR$^-$ and followed a log-normal distribution (Figure~\ref{fig:04}E).\footnote{SWR$^+$ (3.05 [0.85] SD of baseline, median [IQR]; \textit{n} = 1,170) vs. SWR$^-$ (2.37 [0.33] SD of baseline, median [IQR]; \textit{n} = 1,170; \textit{p} $<$ 0.001; Brunner--Munzel test).}.

\subsection{Transient changes in hippocampal neural trajectory during SWR}
We assessed the 'distance' of the neural trajectory from the origin ($O$) during SWR events in both encoding and retrieval phases (Figure~\ref{fig:05}A). Observing the increase in distance during SWR, as illustrated in Figure~\ref{fig:05}A, we categorized each SWR into three stages: pre-, mid-, and post-SWR. Hence, the distances from $O$ during those SWR intervals are identified as $\mathrm{\lVert \text{pre-eSWR}^+ \rVert}$, $\mathrm{\lVert \text{mid-eSWR}^+ \rVert}$ and others.
\\
\indent
As a result, $\mathrm{\lVert \text{mid-eSWR}^+ \rVert}$\footnote{1.25 [1.30], median [IQR], \textit{n} = 1,281 in Match IN task; 1.12 [1.35], median [IQR], \textit{n} = 1,163 in Mismatch OUT task} exceeded $\mathrm{\lVert \text{pre-eSWR}^+ \rVert}$\footnote{1.08 [1.07], median [IQR], \textit{n} = 1,149 in Match IN task; 0.90 [1.12], median [IQR], \textit{n} = 1,088 in Mismatch OUT task}, and $\mathrm{\lVert \text{mid-rSWR}^+ \rVert}$\footnote{1.32 [1.24], median [IQR], \textit{n} = 935 in Match IN task; 1.15 [1.26], median [IQR], \textit{n} = 891 in Mismatch OUT task} was larger than $\mathrm{\lVert \text{pre-rSWR}^+ \rVert}$ in both the Match IN and Mismatch OUT tasks.\footnote{1.19 [0.96], median [IQR], \textit{n} = 673 in Match IN task; 0.94 [0.88], median [IQR], \textit{n} = 664 in Mismatch OUT task}.

\subsection{Visualization of hippocampal neural trajectory during SWR in two-dimensional spaces}
Having observed neural trajectory 'jumping' during SWR (Figure~\ref{fig:05}), we visualized the three-dimensional trajectories of pre-, mid-, and post-SWR events during the encoding and retrieval phases (Figure~\ref{fig:06}). The distance between these was found to be memory-load dependent (Figure~\ref{fig:03}). 
\\
\indent
To provide two-dimensional visualization, we linearly aligned peri-SWR trajectories by setting $\mathrm{g_{E}}$ at the origin (0, 0) and $\mathrm{g_{R}}$ at ($\mathrm{\lVert g_{E}g_{R} \rVert}$, 0). Subsequently, we rotated these aligned trajectories around the $\mathrm{g_{E}g_{R}}$ axis (the x axis), ensuring that the distances from the origin in the original three-dimensional spaces and angles from $\overrightarrow{\mathrm{g_{E}g_{R}}}$ are retained in the two-dimensional equivalent.
\\
\indent
Scatter plot visualization of neural trajectories within these two-dimensional spaces revealed distinct distributions of peri-SWR trajectories based on phases and task types. A notable example of this is the observation that the magnitude of  $\mathrm{\lVert \text{mid-eSWR}^+ \rVert}$ exceeds that of $\mathrm{\lVert \text{pre-eSWR}^+ \rVert}$ (Figure~\ref{fig:06}B), which is consistent with our previous observations (Figure~\ref{fig:05}).

\subsection{Fluctuations of hippocampal neural trajectories between encoding and retrieval states}
Subsequently, we investigated the 'direction' of the trajectory in relation to $\overrightarrow{\mathrm{g_{E}g_{R}}}$, which was found to be dependent on memory load (Figure~\ref{fig:03}). The directions of the SWRs were determined by the neural trajectory at $-250$ ms and $+250$ ms from their center, denoted as, for example, $\overrightarrow{\mathrm{eSWR^+}}$. We calculated the cosine similarities between $\overrightarrow{\mathrm{g_{E}g_{R}}}$, $\overrightarrow{\mathrm{eSWR}}$, and $\overrightarrow{\mathrm{rSWR}}$ in both SWR (SWR^+) and baseline periods (SWR^-) (Figure~\ref{fig:07}A--D).
\\
\indent
$\overrightarrow{\mathrm{rSWR^-}} \cdot \overrightarrow{\mathrm{g_{E}g_{R}}}$ exhibited a biphasic distribution. By computing the difference between the distribution of $\overrightarrow{\mathrm{rSWR^+}} \cdot \overrightarrow{\mathrm{g_{E}g_{R}}}$ (Figure~\ref{fig:07}A \& B) and that of $\overrightarrow{\mathrm{rSWR^-}} \cdot \overrightarrow{\mathrm{g_{E}g_{R}}}$ (Figure~\ref{fig:07}C \& D), we were able to determine the contributions of SWR (Figure~\ref{fig:07}E \& F), which indicated a shift in the direction of $\overrightarrow{\mathrm{g_{E}g_{R}}}$ (Figure~\ref{fig:07}E \& F: \textit{red rectangles}). 
\\
\indent
Furthermore, $\overrightarrow{\mathrm{eSWR^+}} \cdot \overrightarrow{\mathrm{rSWR^+}}$ was less than $\overrightarrow{\mathrm{eSWR^-}} \cdot \overrightarrow{\mathrm{rSWR^-}}$ strictly in Mismatch OUT task (Figure~\ref{fig:07}F: \textit{pink circles}). In other words, eSWR and rSWR pointed in the opposite direction exclusively in Mismatch OUT task but not Match IN task (Figure~\ref{fig:07}E: \textit{pink circles}).
\label{sec:results}