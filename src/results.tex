\section{Results}
\subsection{iEEG recording and neural trajectory in MTL regions during a Sternberg task}
Our analysis utilized a publicly accessible dataset \cite{boran_dataset_2020}, comprised of LFP signals (Figure~\ref{fig:01}A) from MTL regions (Table~\ref{tab:01}), recorded during the execution of a modified Sternberg task. From these LFP signals, we extracted SWR$^+$ candidates filtered in the 80--140 Hz ripple band (Figure~\ref{fig:01}B), originating from all hippocampal regions (refer to the Methods section). Meanwhile, we defined SWR$^-$ candidates, control events for SWR$^+$ candidates, at the same timestamps but distributed across different trials (Figure~\ref{fig:01}). The dataset also included multiunit spikes (Figure~\ref{fig:01}C), identified using a spike sorting algorithm \cite{niediek_reliable_2016}. Applying GPFA \cite{yu_gaussian-process_2009} to 50-ms windows of binned multiunit activity without overlaps, we determined the neural trajectories, or factors, of MTL regions by session and region (Figure~\ref{fig:01}D). We normalized each factor per session and region, for example, session \#2 in AHL of subject \#1, then calculated the Euclidean distance from the origin ($O$) (Figure~\ref{fig:01}E).

\subsection{Hippocampal neural trajectory correlation with a Sternberg task}
Figure~\ref{fig:02}A displays the distribution of median neural trajectories, which composed of 50 trials, within the three main factor spaces. Using the elbow method, we determined three as the optimal embedding dimension for the GPFA model (Figure~\ref{fig:02}B). The trajectory distance from the origin ($O$) --- represented as $\mathrm{\lVert g_{F} \rVert}$, $\mathrm{\lVert g_{E} \rVert}$, $\mathrm{\lVert g_{M} \rVert}$, and $\mathrm{\lVert g_{R} \rVert}$ --- in the hippocampus exceeded the corresponding distances in the EC and amygdala (Figure~\ref{fig:02}C \& D).\footnote{Hippocampus: Distance = 1.11 [1.01], median [IQR], \textit{n} = 195,681 timepoints; EC: Distance = 0.94 [1.10], median [IQR], \textit{n} = 133,761 timepoints; Amygdala: Distance = 0.78 [0.88], median [IQR], \textit{n} = 165,281 timepoints.}

We also calculated the distances between the geometric medians of four phases, namely $\mathrm{\lVert g_{F}g_{E} \rVert}$, $\mathrm{\lVert g_{F}g_{M} \rVert}$, $\mathrm{\lVert g_{F}g_{R} \rVert}$, $\mathrm{\lVert g_{E}g_{M} \rVert}$, $\mathrm{\lVert g_{E}g_{R} \rVert}$, and $\mathrm{\lVert g_{M}g_{R} \rVert}$. The hippocampus exhibited larger distances between phases compared to the EC and the amygdala.\footnote{Hippocampus: Distance = 0.60 [0.70], median [IQR], \textit{n} = 8,772 combinations; EC: Distance = 0.28 [0.52], median [IQR], \textit{n} = 5,017 combinations (\textit{p} $<$ 0.01; Brunner--Munzel test); Amygdala: Distance = 0.24 [0.42], median [IQR], \textit{n} = 7,466 combinations (\textit{p} $<$ 0.01; Brunner--Munzel test).}

\subsection{Memory-load-dependent neural trajectory distance between encoding and retrieval states in the hippocampus}
We observed a negative correlation between the correct rate of trials and the set size, indicating the number of letters to be encoded, during the Sternberg task (Figure~\ref{fig:03}A).\footnote{Correct rate: set size four (0.99 \textpm 0.11, mean \textpm SD; \textit{n} = 333 trials) vs. set size six (0.93 \textpm 0.26; \textit{n} = 278 trials; \textit{p} $<$ 0.001, Brunner--Munzel test with Bonferroni correction) and set size eight (0.87 \textpm 0.34; \textit{n} = 275 trials; \textit{p} $<$ 0.05; Brunner--Munzel test with Bonferroni correction). Generally, \textit{p} $<$ 0.001 for Kruskal--Wallis test; correlation coefficient = - 0.20, \textit{p} $<$ 0.001.} Simultaneously, we found a positive correlation between the response time and set size (Figure~\ref{fig:03}B).\footnote{Response time: set size four (1.26 \textpm 0.45 s; \textit{n} = 333 trials) vs. set size six (1.53 \textpm 0.91 s; \textit{n} = 278 trials) and set size eight (1.66 \textpm 0.80 s; \textit{n} = 275 trials). All comparisons \textit{p} $<$ 0.001, Brunner--Munzel test with Bonferroni correction; \textit{p} $<$ 0.001 for Kruskal--Wallis test; correlation coefficient = 0.22, \textit{p} $<$ 0.001}.

Further, we identified a positive correlation between the set size and the trajectory distance separating the encoding and retrieval phases ($\mathrm{log_{10}\lVert g_{E}g_{R} \rVert}$) (Figure~\ref{fig:03}C).\footnote{Correlation between set size and $\mathrm{log_{10}(\lVert g_{E}g_{R} \rVert}$): correlation coefficient = 0.05, \textit{p} $<$ 0.001. Specific values: $\mathrm{\lVert g_{E}g_{R} \rVert}$ = 0.54 [0.70] for set size four, \textit{n} = 447; $\mathrm{\lVert g_{E}g_{R} \rVert}$ = 0.58 [0.66] for set size six, \textit{n} = 381; $\mathrm{\lVert g_{E}g_{R} \rVert}$ = 0.61 [0.63] for set size eight, \textit{n} = 395.}. However, distances between other phase combinations did not show statistically significant correlations (Figures~\ref{fig:03}D and \ref{fig:s02}).

\subsection{Detection of hippocampal SWR from putative CA1 regions}
To improve the accuracy of the recording sites and SWR detection, we estimated the electrode placements in the CA1 regions of the hippocampus using distinctive multiunit spike patterns during SWR events. We embedded SWR$^+$/SWR$^-$ candidates from each session and hippocampal region in a two-dimensional space using UMAP (Figure~\ref{fig:04}A).\footnote{Consider the AHL in session \#1 of subject \#1 as an example.} Using the silhouette score as a clustering quality metric (Figure~\ref{fig:04}B and Table~\ref{tab:02}), we identified recording sites that showed an average silhouette score exceeding 0.6 across all sessions as putative CA1 regions.\footnote{The identified regions were the AHL of subject \#1, AHR of subject \#3, PHL of subject \#4, AHL of subject \#6, and AHR of subject \#9.} (Tables~\ref{tab:02} and \ref{tab:03}). From these, we identified five putative CA1 regions, four of which were not identified as seizure onset zones (Table~\ref{tab:01}).

We labeled SWR$^+$/SWR$^-$ candidates from these putative CA1 regions as SWR$^+$ and SWR$^-$, respectively\footnote{These definitions resulted in equal counts for both categories: SWR$^+$ (\textit{n} = 1,170) and SWR$^-$ (\textit{n} = 1,170).} (Table~\ref{tab:03}). Both SWR$^+$ and SWR$^-$, due to their definitions, presented equivalent durations \footnote{These definitions resulted in equal durations for both categories: SWR$^+$ (93.0 [65.4] ms) and SWR$^-$ (93.0 [65.4] ms).}. They followed a log-normal distribution (Figure~\ref{fig:04}C). An increase in SWR$^+$ incidence was detected during the initial 400 ms of the retrieval phase\footnote{SWR$^+$ increased against the bootstrap sample; 95th percentile = 0.42 [Hz]; \textit{p} $<$ 0.05.} (Figure~\ref{fig:04}D). The peak ripple band amplitude of SWR$^+$ was higher than that of SWR$^-$, following a log-normal distribution (Figure~\ref{fig:04}E).\footnote{SWR$^+$ (3.05 [0.85] SD of baseline, median [IQR]; \textit{n} = 1,170) vs. SWR$^-$ (2.37 [0.33] SD of baseline, median [IQR]; \textit{n} = 1,170; \textit{p} $<$ 0.001; Brunner--Munzel test).}. 

\subsection{Transient changes in hippocampal neural trajectory during SWR}
We examined the 'distance' of the neural trajectory from the origin ($O$) during SWR events in both encoding and retrieval phases (Figure~\ref{fig:05}A). Upon observing an increase in distance during SWR, as shown in Figure~\ref{fig:05}A, we categorized each SWR into three stages: pre-, mid-, and post-SWR. Thus, the distances from $O$ during such SWR intervals are represented as $\mathrm{\lVert \text{pre-eSWR}^+ \rVert}$, $\mathrm{\lVert \text{mid-eSWR}^+ \rVert}$, and others.

Consequently, $\mathrm{\lVert \text{mid-eSWR}^+ \rVert}$\footnote{1.25 [1.30], median [IQR], \textit{n} = 1,281 in Match IN task; 1.12 [1.35], median [IQR], \textit{n} = 1,163 in Mismatch OUT task} exceeded $\mathrm{\lVert \text{pre-eSWR}^+ \rVert}$\footnote{1.08 [1.07], median [IQR], \textit{n} = 1,149 in Match IN task; 0.90 [1.12], median [IQR], \textit{n} = 1,088 in Mismatch OUT task}, and $\mathrm{\lVert \text{mid-rSWR}^+ \rVert}$\footnote{1.32 [1.24], median [IQR], \textit{n} = 935 in Match IN task; 1.15 [1.26], median [IQR], \textit{n} = 891 in Mismatch OUT task} was larger than $\mathrm{\lVert \text{pre-rSWR}^+ \rVert}$ in both the Match IN and Mismatch OUT tasks.\footnote{1.19 [0.96], median [IQR], \textit{n} = 673 in Match IN task; 0.94 [0.88], median [IQR], \textit{n} = 664 in Mismatch OUT task}.

\subsection{Visualization of hippocampal neural trajectory during SWR in two-dimensional spaces}
Observing 'jumping' of the neural trajectory during SWR (Figure~\ref{fig:05}), we visualized three-dimensional trajectories of pre-, mid-, and post-SWR events during the encoding and retrieval phases (Figure~\ref{fig:06}). The distance between these was found to be memory-load dependent (Figure~\ref{fig:03}). 

To provide a two-dimensional visualization, we linearly aligned peri-SWR trajectories by setting $\mathrm{g_{E}}$ at the origin (0, 0) and $\mathrm{g_{R}}$ at ($\mathrm{\lVert g_{E}g_{R} \rVert}$, 0). We then rotated these aligned trajectories around the $\mathrm{g_{E}g_{R}}$ axis (the x axis), ensuring the preservation of distances from the origin in original three-dimensional spaces and angles from $\overrightarrow{\mathrm{g_{E}g_{R}}}$ in two-dimensional correlates.

In two-dimensional spaces, scatter plot visualization revealed distinct distributions of peri-SWR trajectories based on phases and task types. For instance, the magnitude of $\mathrm{\lVert \text{mid-eSWR}^+ \rVert}$ exceeded $\mathrm{\lVert \text{pre-eSWR}^+ \rVert}$ (Figure~\ref{fig:06}B), as consistent with our previous findings (Figure~\ref{fig:05}).

\subsection{Fluctuations of hippocampal neural trajectories between encoding and retrieval states}
Next, we investigated the 'direction' of the trajectory in relation to $\overrightarrow{\mathrm{g_{E}g_{R}}}$, found to be dependent on memory load (Figure~\ref{fig:03}). We defined the directions of the SWRs by the neural trajectory at $-250$ ms and $+250$ ms from their center, labeled as, for example, $\overrightarrow{\mathrm{eSWR^+}}$. We computed the cosine similarities between $\overrightarrow{\mathrm{g_{E}g_{R}}}$, $\overrightarrow{\mathrm{eSWR}}$, and $\overrightarrow{\mathrm{rSWR}}$ during both SWR (SWR^+) and baseline periods (SWR^-) (Figure~\ref{fig:07}A--D).

$\overrightarrow{\mathrm{rSWR^-}} \cdot \overrightarrow{\mathrm{g_{E}g_{R}}}$ manifested a biphasic distribution. By computing the difference between the distribution of $\overrightarrow{\mathrm{rSWR^+}} \cdot \overrightarrow{\mathrm{g_{E}g_{R}}}$ (Figure~\ref{fig:07}A \& B) and that of $\overrightarrow{\mathrm{rSWR^-}} \cdot \overrightarrow{\mathrm{g_{E}g_{R}}}$ (Figure~\ref{fig:07}C \& D), it was possible to discern the contributions of SWR (Figure~\ref{fig:07}E \& F). This analysis indicated a shift in the direction of $\overrightarrow{\mathrm{g_{E}g_{R}}}$ (Figure~\ref{fig:07}E \& F: \textit{red rectangles}). 

Moreover, $\overrightarrow{\mathrm{eSWR^+}} \cdot \overrightarrow{\mathrm{rSWR^+}}$ was less than $\overrightarrow{\mathrm{eSWR^-}} \cdot \overrightarrow{\mathrm{rSWR^-}}$ strictly in the Mismatch OUT task (Figure~\ref{fig:07}F: \textit{pink circles}). Therefore, eSWR and rSWR pointed in opposite directions exclusively in Mismatch OUT task but didn't do so in Match IN task (Figure~\ref{fig:07}E: \textit{pink circles}).
\label{sec:results}