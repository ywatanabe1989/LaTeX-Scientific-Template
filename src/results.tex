\section{Results}
\subsection{Intracranial EEG Recording and Neural Trajectories in Medial Temporal Lobe Regions During the Sternberg Task}

The analysis utilized a publicly available dataset \cite{boran_dataset_2020}, comprising local field potential signals (Figure~\ref{fig:01}A) from medial temporal lobe regions (Table~\ref{tab:01}) recorded during a modified Sternberg task. Sharp wave-ripple (SWR$^+$) candidates were extracted from local field potential signals and filtered in the 80--140 Hz ripple band (Figure~\ref{fig:01}B), deriving from all hippocampal regions (see Methods section for reference). SWR$^-$ candidates, control events for SWR$^+$ candidates, were defined at the same timestamps but distributed across different trials (Figure~\ref{fig:01}). The dataset encompassed multi-unit spike events (Figure~\ref{fig:01}C), acknowledged by a spike sorting algorithm \cite{niediek_reliable_2016}. Applying Gaussian-process factor analysis (GPFA) \cite{yu_gaussian-process_2009}, this was employed to 50-ms windows of binned multi-unit activity without overlaps to determine the neural trajectories or factors, of medial temporal lobe regions by session and region (Figure~\ref{fig:01}D). Each factor was normalized per session and region, for example, session \#2 in the anterior hippocampal left region of subject \#1. Then the Euclidean distance from the origin ($O$) was calculated (Figure~\ref{fig:01}E).

\subsection{Correlation of Hippocampal Neural Trajectories with the Sternberg Task}
Figure~\ref{fig:02}A displays a distribution of median neural trajectories, consisting of 50 trials, within the three main factor spaces. Using the elbow method, we determined the optimal embedding dimension for the GPFA model to be three (Figure~\ref{fig:02}B). The neural trajectory distance from the origin ($O$)— represented as $\mathrm{\lVert g_{F} \rVert}$, $\mathrm{\lVert g_{E} \rVert}$, $\mathrm{\lVert g_{M} \rVert}$, and $\mathrm{\lVert g_{R} \rVert}$— in the hippocampus exceeded the corresponding distances in the entorhinal cortex and amygdala (Figure~\ref{fig:02}C \& D).

We computed the distances between the geometric medians of four phases, namely $\mathrm{\lVert g_{F}g_{E} \rVert}$, $\mathrm{\lVert g_{F}g_{M} \rVert}$, $\mathrm{\lVert g_{F}g_{R} \rVert}$, $\mathrm{\lVert g_{E}g_{M} \rVert}$, $\mathrm{\lVert g_{E}g_{R} \rVert}$, and $\mathrm{\lVert g_{M}g_{R} \rVert}$. The hippocampus displayed larger distances between phases than those in the entorhinal cortex and amygdala.

\subsection{Memory Load-Dependent Neural Trajectory Distance between Encoding and Retrieval States in the Hippocampus}

Regarding the memory load in the Sternberg task, we noted a negative correlation between the correct rate of trials and the set size, which represents the number of letters to encode (Figure~\ref{fig:03}A). Simultaneously, a positive correlation existed between the response time and set size (Figure~\ref{fig:03}B).

We noted a positive correlation between set size and the neural trajectory distance between the encoding and retrieval phases ($\mathrm{log_{10}\lVert g_{E}g_{R} \rVert}$) (Figure~\ref{fig:03}C). However, distances between other phase combos did not yield any significant correlations (Figures~\ref{fig:03}D and \ref{fig:s02}).

\subsection{Detection of Hippocampal Sharp Wave-Ripples from Putative CA1 Regions}
To enhance recording site accuracy and sharp wave-ripple detection, we approximated the electrode positions in the CA1 regions of the hippocampus based on distinct multi-unit spike patterns during sharp wave-ripple events. SWR$^+$/SWR$^-$ candidates from each session and hippocampal region were embedded in a two-dimensional space using uniform manifold approximation and projection (UMAP) (Figure~\ref{fig:04}A). The silhouette score served as a clustering quality metric (Figure~\ref{fig:04}B and Table~\ref{tab:02}), and recording sites demonstrating an average silhouette score over 0.6 across all sessions were identified as putative CA1 regions (Tables~\ref{tab:02} and \ref{tab:03}). We identified five putative CA1 regions, four of which did not display as seizure onset zones (Table~\ref{tab:01}).

Subsequently, SWR$^+$/SWR$^-$ candidates within these putative CA1 regions were labeled as SWR$^+$ and SWR$^-$, respectively (Table~\ref{tab:03}). Both SWR$^+$ and SWR$^-$ demonstrated identical durations due to their definitions and followed a log-normal distribution (Figure~\ref{fig:04}C). During the initial 400 ms of the retrieval phase, an increase in SWR$^+$ occurrence was noted (Figure~\ref{fig:04}D). The peak ripple band amplitude of SWR$^+$ exceeded that of SWR$^-$ and followed a log-normal distribution (Figure~\ref{fig:04}E).

\subsection{Transient Shift in Hippocampal Neural Trajectory During Sharp Wave-Ripple}
We evaluated the 'distance' of the neural trajectory from the origin ($O$) during sharp wave-ripple events in both the encoding and retrieval phases (Figure~\ref{fig:05}A). Observing the distance increase during sharp wave-ripple, we categorized each sharp wave-ripple into three stages: pre-, mid-, and post-sharp wave-ripple. Thus, the distances from the origin $O$ during those sharp wave-ripple intervals are denoted as $\mathrm{\lVert \text{pre-eSWR}^+ \rVert}$, $\mathrm{\lVert \text{mid-eSWR}^+ \rVert}$, and others.

As a result, $\mathrm{\lVert \text{mid-eSWR}^+ \rVert}$ exceeded $\mathrm{\lVert \text{pre-eSWR}^+ \rVert}$, and $\mathrm{\lVert \text{mid-rSWR}^+ \rVert}$ was larger than $\mathrm{\lVert \text{pre-rSWR}^+ \rVert}$ in both the Match IN and Mismatch OUT tasks.

\subsection{Visualization of Hippocampal Neural Trajectory During Sharp Wave-Ripple in Two-Dimensional Spaces}
Having observed neural trajectory 'jumping' during sharp wave-ripple (Figure~\ref{fig:05}), we visualized the three-dimensional neural trajectories of pre-, mid-, and post-sharp wave-ripple events during the encoding and retrieval phases (Figure~\ref{fig:06}). The distance between these was found to be dependent on memory load (Figure~\ref{fig:03}).

To provide a two-dimensional visualization, we linearly aligned peri-sharp wave-ripple neural trajectories by setting $\mathrm{g_{E}}$ at the origin (0, 0) and $\mathrm{g_{R}}$ at ($\mathrm{\lVert g_{E}g_{R} \rVert}$, 0). We then rotated these aligned neural trajectories around the $\mathrm{g_{E}g_{R}}$ axis (the x axis), ensuring that the distances from the origin $O$ in the original three-dimensional spaces and angles from $\overrightarrow{\mathrm{g_{E}g_{R}}}$ were maintained in the two-dimensional equivalent.

A scatter plot visualization of neural trajectories within these two-dimensional spaces revealed distinct distributions of peri-sharp wave-ripple neural trajectories based on phases and task types. A notable example of this is the observation that the magnitude of $\mathrm{\lVert \text{mid-eSWR}^+ \rVert}$ exceeds that of $\mathrm{\lVert \text{pre-eSWR}^+ \rVert}$ (Figure~\ref{fig:06}B), consistent with our previous observations (Figure~\ref{fig:05}).

\subsection{Fluctuation of Hippocampal Neural Trajectories between Encoding and Retrieval States}
Subsequently, we examined the 'direction' of the neural trajectory in relation to $\overrightarrow{\mathrm{g_{E}g_{R}}}$, which was found to be memory load-dependent (Figure~\ref{fig:03}). The directions of the sharp wave-ripples were determined by the neural trajectory at $-250$ ms and $+250$ ms from their center, denoted as, for example, $\overrightarrow{\mathrm{eSWR^+}}$. We calculated the cosine similarity among $\overrightarrow{\mathrm{g_{E}g_{R}}}$, $\overrightarrow{\mathrm{eSWR}}$, and $\overrightarrow{\mathrm{rSWR}}$ in both sharp wave-ripple (SWR$^+$) and baseline periods (SWR$^-$) (Figure~\ref{fig:07}A--D).

$\overrightarrow{\mathrm{rSWR^-}} \cdot \overrightarrow{\mathrm{g_{E}g_{R}}}$ revealed a biphasic distribution. We determined the contributions of sharp wave-ripple by calculating the difference between the distribution of $\overrightarrow{\mathrm{rSWR^+}} \cdot \overrightarrow{\mathrm{g_{E}g_{R}}}$ (Figure~\ref{fig:07}A \& B) and that of $\overrightarrow{\mathrm{rSWR^-}} \cdot \overrightarrow{\mathrm{g_{E}g_{R}}}$ (Figure~\ref{fig:07}C \& D), which indicated a shift in the direction of $\overrightarrow{\mathrm{g_{E}g_{R}}}$ (Figure~\ref{fig:07}E \& F: \textit{red rectangles}).

In addition, $\overrightarrow{\mathrm{eSWR^+}} \cdot \overrightarrow{\mathrm{rSWR^+}}$ was less than $\overrightarrow{\mathrm{eSWR^-}} \cdot \overrightarrow{\mathrm{rSWR^-}}$ strictly in the Mismatch OUT task (Figure~\ref{fig:07}F: \textit{pink circles}). In other words, encoding sharp wave-ripple and retrieval sharp wave-ripple pointed in the opposite direction, but only in the Mismatch OUT task, and not in the Match IN task (Figure~\ref{fig:07}E: \textit{pink circles}).
\label{sec:results}