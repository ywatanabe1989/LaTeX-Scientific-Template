%%%%%%%%%%%%%%%%%%%%%%%%%%%%%%%%%%%%%%%%%%%%%%%%%%%%%%%%%%%%%%%%%%%%%%%%%%%%%%%%
%% Results
%%%%%%%%%%%%%%%%%%%%%%%%%%%%%%%%%%%%%%%%%%%%%%%%%%%%%%%%%%%%%%%%%%%%%%%%%%%%%%%%
\section{Results}
\subsection{iEEG recording and neural trajectory in MTL regions during a Sternberg task}
We utilized a publicly accessible dataset \cite{boran_dataset_2020} for this analysis, which consists of LFP signals (Figure 1A) from MTL regions (Table~\ref{tab:01}1) obtained during a modified Sternberg task. SWR$^+$ candidates were detected within all hippocampal regions from LFP signals, filtered by the ripple band (80--140 Hz) (Figure 1B), whereas SWR$^-$ candidates were defined at identical timestamps as SWR$^+$ candidates but shuffled across separate trials (Figure 1). The multiunit spikes (Figure 1C) were also included in the dataset, identified by implementation of a spike sorting algorithm \cite{niediek_reliable_2016}. Relying on the 50-ms binned multiunit activity devoid of overlaps, we applied GPFA \cite{yu_gaussian-process_2009} to elucidate the neural trajectory (or factors) of MTL regions by session and region (Figure 1D). Each factor was z-normalized by session and region (an instance being session \#2 in AHL of subject \#1). The Euclidean distance from the origin ($O$) was subsequently calculated (Figure 1E).

\subsection{Hippocampal neural trajectory correlation with a Sternberg task}
Figure 2A delineates the median neural trajectories of 50 trials as point clouds within the three primary factor space. The optimal embedding dimension for the GPFA model, determined by employing the elbow method, was found to be three (Figure 2B). The trajectory distance from the origin ($O$) ($\mathrm{\lVert g_{F} \rVert}$, $\mathrm{\lVert g_{E} \rVert}$, $\mathrm{\lVert g_{M} \rVert}$, and $\mathrm{\lVert g_{R} \rVert}$) was larger in the hippocampus than in the EC and amygdala (Figure 2C \& D).\footnote{Hippocampus: Distance = 1.11 [1.01], median [IQR], \textit{n} = 195,681 timepoints; EC: Distance = 0.94 [1.10], median [IQR], \textit{n} = 133,761 timepoints; Amygdala: Distance = 0.78 [0.88], median [IQR], \textit{n} = 165,281 timepoints.}
\\
\indent
Similarly, the distances among geometric medians of the four phases: $\mathrm{\lVert g_{F}g_{E} \rVert}$, $\mathrm{\lVert g_{F}g_{M} \rVert}$, $\mathrm{\lVert g_{F}g_{R} \rVert}$, $\mathrm{\lVert g_{E}g_{M} \rVert}$, $\mathrm{\lVert g_{E}g_{R} \rVert}$, and $\mathrm{\lVert g_{M}g_{R} \rVert}$ were calculated, displaying that the hippocampus demonstrated larger distances among the phases compared to both the EC and amygdala. \footnote{Hippocampus: Distance = 0.60 [0.70], median [IQR], \textit{n} = 8,772 combinations; EC: Distance = 0.28 [0.52], median [IQR], \textit{n} = 5,017 combinations (\textit{p} $<$ 0.01; Brunner--Munzel test); Amygdala: Distance = 0.24 [0.42], median [IQR], \textit{n} = 7,466 combinations (\textit{p} $<$ 0.01; Brunner--Munzel test).}

\subsection{Memory load-dependent neural trajectory distance between the encoding and retrieval states in the hippocampus}
Considering the memory load of the Stenberg task, the correct trial rate and set size (= the number of alphabet letters to encode) were negatively correlated (Figure 3A). \footnote{Correct rate: set size four (0.99 \textpm 0.11, mean \textpm SD; \textit{n} = 333 trials) vs. set size six (0.93 \textpm 0.26; \textit{n} = 278 trials) and set size eight (0.87 \textpm 0.34; \textit{n} = 275 trials; \textit{p} $<$ 0.05; Brunner--Munzel test with Bonferroni correction). Overall, \textit{p} $<$ 0.001 for Kruskal--Wallis test; correlation coefficient = - 0.20, \textit{p} $<$ 0.001.} Likewise, response time and set size displayed a positive correlation (Figure 3B).\footnote{Response time: set size four (1.26 \textpm 0.45 s; \textit{n} = 333 trials) vs. set size six (1.53 \textpm 0.91 s; \textit{n} = 278 trials) and set size eight (1.66 \textpm 0.80 s; \textit{n} = 275 trials). All comparisons \textit{p} $<$ 0.001, Brunner--Munzel test with Bonferroni correction; \textit{p} $<$ 0.001 for Kruskal--Wallis test; correlation coefficient = 0.22, \textit{p} $<$ 0.001}
\\
\indent
Further, the set size and trajectory distance between the encoding and retrieval phases ($\mathrm{log_{10}\lVert g_{E}g_{R} \rVert}$) exhibited a positive correlation (Figure 3C).\footnote{Correlation between set size and $\mathrm{log_{10}(\lVert g_{E}g_{R} \rVert}$): correlation coefficient = 0.05, \textit{p} $<$ 0.001. Specific values: $\mathrm{\lVert g_{E}g_{R} \rVert}$ = 0.54 [0.70] for set size four trials, \textit{n} = 447; $\mathrm{\lVert g_{E}g_{R} \rVert}$ = 0.58 [0.66] for set size six trials, \textit{n} = 381; $\mathrm{\lVert g_{E}g_{R} \rVert}$ = 0.61 [0.63] for set size eight trials, \textit{n} = 395.}, while distances between other phase combinations did not note any significant correlations (Figures 3D \& S2).

\subsection{Detection of hippocampal SWR from putative CA1 regions}
With the intent of improving the accuracy of recording sites and SWR detection, we attempted to estimate electrodes in CA1 regions of the hippocampus by observing distinct multiunit spike patterns during SWR occurrences. For each session and hippocampal region, SWR$^+$/SWR$^-$ candidates were embedded into a two-dimensional space using UMAP (Figure 4A).\footnote{For illustrative purposes, consider the AHL in session \#1 of subject \#1.} The silhouette score was calculated as a measure of clustering quality (Figure 4B \& Table~\ref{tab:02}). Recording sites with an average silhouette score across sessions exceeding 0.6 were recognized as putative CA1 regions \cite{mcinnes_umap_2018, rousseeuw_silhouettes_1987} \footnote{The regions identified were: AHL of subject \#1, AHR of subject \#3, PHL of subject \#4, AHL of subject \#6, and AHR of subject \#9.}  (Tables~\ref{tab:02} \& \ref{tab:03}). Hence, we identified five putative CA1 regions, four of which were not previously labeled as seizure onset zones (Table~\ref{tab:01}).
\\
\indent
Subsequently, we labeled SWR$^+$/SWR$^-$ candidates within these putative CA1 regions as SWR$^+$ and SWR$^-$, respectively\footnote{Defining them resulted in equal counts for both categories: SWR$^+$ (\textit{n} = 1,170) and SWR$^-$ (\textit{n} = 1,170).}  (Table~\ref{tab:03}). Both SWR$^+$ and SWR$^-$ exhibited the same duration\footnote{Defining them resulted in identical duration for both categories: SWR$^+$ (93.0 [65.4] ms) and SWR$^-$ (93.0 [65.4] ms).}  (Figure 4C) as per their definitions, following a log-distribution profile. A notable uptick in SWR$^+$ incidence appeared during the initial 400 ms of the retrieval phase \footnote{SWR$^+$ increased against the bootstrap sample; 95th percentile = 0.42 [Hz]; \textit{p} $<$ 0.05.}  (Figure 4D). Besides, the peak ripple band amplitude of SWR$^+$ exceeded that of SWR$^-$, following a log-normal distribution (Figure 4E).\footnote{SWR$^+$ (3.05 [0.85] SD of baseline, median [IQR]; \textit{n} = 1,170) vs. SWR$^-$ (2.37 [0.33] SD of baseline, median [IQR]; \textit{n} = 1,170; \textit{p} $<$ 0.001; Brunner--Munzel test).}.

\subsection{Transient change in neural trajectory in the hippocampus during SWR}
We analyzed the \textit{distances} of the trajectory from origin ($O$) during SWR events in both encoding and retrieval phases (Figure 5A). Given the increase in distance during SWR, as shown in Figure 5A, we classified each SWR into three stages: pre-, mid-, and post-SWR. Thereafter, the distances from $O$ during these SWR periods are represented as $\mathrm{\lVert \text{pre-eSWR}^+ \rVert}$, $\mathrm{\lVert \text{mid-eSWR}^+ \rVert}$, and so forth.
\\
\indent
$\mathrm{\lVert \text{mid-eSWR}^+ \rVert}$
\footnote{1.25 [1.30], median [IQR], \textit{n} = 1,281, in Match IN task; 1.12 [1.35], median [IQR], \textit{n} = 1,163, in Mismatch OUT task}
was larger than $\mathrm{\lVert \text{pre-eSWR}^+ \rVert}$
\footnote{1.08 [1.07], median [IQR], \textit{n} = 1,149, in Match IN task; 0.90 [1.12], median [IQR], \textit{n} = 1,088, in Mismatch OUT task}, and $\mathrm{\lVert \text{mid-rSWR}^+ \rVert}$ 
\footnote{1.32 [1.24], median [IQR], \textit{n} = 935, in Match IN task; 1.15 [1.26], median [IQR], \textit{n} = 891, in Mismatch OUT task}
was larger than $\mathrm{\lVert \text{pre-rSWR}^+ \rVert}$ in both Match IN and Mismatch OUT tasks.
\footnote{1.19 [0.96], median [IQR], \textit{n} = 673, in Match IN task; 0.94 [0.88], median [IQR], \textit{n} = 664, in Mismatch OUT task}

\subsection{Visualization of hippocampal neural trajectory during SWR in two-dimensional spaces}
Given our observations of neural trajectory 'jump' during SWR (Figure 5), we visualized the three-dimensional trajectories of pre-, mid-, and post-SWR events during the encoding and retrieval phases (Figure 6), the distance between which was dependent on memory-load (Figure 3).
\\
\indent
We accomplished the visualization in two-dimensional spaces by linearly aligning peri-SWR trajectories, positioning $\mathrm{g_{E}}$ at the origin (0, 0) and $\mathrm{g_{R}}$ at the coordinate ($\mathrm{\lVert g_{E}g_{R} \rVert}$, 0). These aligned trajectories were then rotated around the $\mathrm{g_{E}g_{R}}$ axis (= x-axis), ensuring conservation of distances from the origin $O$ and angles between $\mathrm{g_{E}g_{R}}$ in the original three-dimensional spaces in the two-dimensional spaces.
\\
\indent
The scatter plot in these two-dimensional spaces portrays the distinctive distributions of peri-SWR trajectories based on phases and task types. For instance, one can distinguish that $\mathrm{\lVert \text{mid-eSWR}^+ \rVert}$ is larger than $\mathrm{\lVert \text{pre-eSWR}^+ \rVert}$ (Figure 6B), consistent with our earlier findings (Figure 5).

\subsection{Fluctuating hippocampal neural trajectories between encoding and retrieval states}
Following this, we inspected the trajectory \textit{directions} based on $\overrightarrow{\mathrm{g_{E}g_{R}}}$. Directions of SWRs were identified by the neural trajectory at $-250$ ms and $+250$ ms from their center (i.e., $\overrightarrow{\mathrm{eSWR^+}}$).
\\
\indent
We computed the density of $\overrightarrow{\mathrm{eSWR}} \cdot \overrightarrow{\mathrm{g_{E}g_{R}}}$, $\overrightarrow{\mathrm{rSWR}} \cdot \overrightarrow{\mathrm{g_{E}g_{R}}}$, and $\overrightarrow{\mathrm{eSWR}} \cdot \overrightarrow{\mathrm{rSWR}}$ (Figure 7A--D). $\overrightarrow{\mathrm{rSWR^-}} \cdot \overrightarrow{\mathrm{g_{E}g_{R}}}$ exhibited biphasic distributions.
\\
\indent
By comparing the distribution of $\overrightarrow{\mathrm{rSWR^+}} \cdot \overrightarrow{\mathrm{g_{E}g_{R}}}$ (Figures 7A \& B) with those of $\overrightarrow{\mathrm{rSWR^-}} \cdot \overrightarrow{\mathrm{g_{E}g_{R}}}$ (Figures 7C \& D), we calculated the contributions of SWR (Figures 7E \& F), which unveiled a shift in the direction of $\overrightarrow{\mathrm{g_{E}g_{R}}}$ (Figures 7E \& F; see \textit{red rectangles}).
\\
\indent
Additionally, only in the Mismatch OUT task was $\overrightarrow{\mathrm{eSWR^+}} \cdot \overrightarrow{\mathrm{rSWR^+}}$ less than that of $\overrightarrow{\mathrm{eSWR^-}} \cdot \overrightarrow{\mathrm{rSWR^-}}$ (baseline periods) (Figure 7F; see \textit{pink circles}); stated differently, eSWRs and rSWRs directed in the opposing direction exclusively in the Mismatch OUT task but not in the Match IN task (Figure 7E; see \textit{pink circles}).
\label{sec:results}