\clearpage
\pdfbookmark[2]{ID 02}{id_02}
\begin{table*}[ht]
\centering
\rowcolors{3}{gray!25}{white}
\rowcolor{white}
\begin{tabular}{|l|l|l|l|l|}
\hline
&Hipp. head&&Hipp. body&\\
\hline
Subject&AHL&AHR&PHL&PHR\\
\hline
\#1&0.60 ± 0.14&n.a.&n.a.&0.1 ± 0\\
\hline
\#2&0.21 ± 0.16&0.17 ± 0.21&0.18 ± 0.22&0.20 ± 0.15\\
\hline
\#3&0.40 ± 0.42&0.83 ± 0.12&n.a.&n.a.\\
\hline
\#4&0.10 ± 0.00&0.10 ± 0.00&0.90 ± 0.00&0.10 ± 0.14\\
\hline
\#5&n.a.&n.a.&n.a.&n.a.\\
\hline
\#6&0.63 ± 0.06&n.a.&n.a.&0.27 ± 0.06\\
\hline
\#7&0.10 ± 0.00&0.35 ± 0.35&0.37 ± 0.47&0.10 ± 0.00\\
\hline
\#8&0.13 ± 0.10&n.a.&0.28 ± 0.49&n.a.\\
\hline
\#9&n.a.&0.85 ± 0.07&0.15 ± 0.07&n.a.\\
\hline
\bottomrule
\end{tabular}
\caption{\textbf{
The silhouette score of UMAP clustering between $SWR^+$ candidates and $SWR^-$ candidates
}
\smallskip
\\
The silhouette scores (mean ± SD for sessions by subject) of UMAP clustering on SWR+ candidates and SWR− candidates (Figure 4A) were based on their underlying multiunit spike patterns (mean values were 0.205 [0.285], median [IQR]; Figure 4B).
}
% width=1\textwidth\label{tab:02}
\end{table*}
