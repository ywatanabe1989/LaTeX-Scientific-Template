\caption{\textbf{
State-Dependent Trajectories of Hippocampal Neurons
}
\smallskip
\\
\textbf{\textit{A.}} Neural trajectories within the initial three-dimensional factors derived from the Gaussian Process Factor Analysis (GPFA) are displayed. The smaller dots correspond to coordinates of 50-ms neural trajectory bins, while the larger dots with \textit{black} edges signify the geometric medians for respective stages in the Sternberg working memory task: fixation (\textit{gray}), encoding (\textit{blue}), maintenance (\textit{green}), and retrieval (\textit{red}). \textbf{\textit{B.}} The figure conveys the log-likelihood of the GPFA models versus the count of dimensions used to embed multiunit spikes found in the medial temporal lobe (MTL) territories. In specific, the elbow method pinpointed the optimal dimension to be three. \textbf{\textit{C.}} This panel illustrates the distance of the neural trajectories from the origin ($O$) for the hippocampus (Hipp.), entorhinal cortex (EC), and amygdala (Amy.), against the time elapsed from the probe onset. \textbf{\textit{D.}} The distance of the trajectory from $O$ within MTL regions is displayed. The hippocampus shows the farthest distance, followed by the EC and the Amygdala. \textbf{\textit{E.}} The plot represents inter-phase trajectory distances within the MTL regions.
Abbreviations:
}
% width=0.5\textwidth
