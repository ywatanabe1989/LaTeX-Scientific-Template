\caption{\textbf{Detection of Sharp Wave Ripples in Presumed CA1 Regions}\\
\textbf{\textit{A.}} A two-dimensional UMAP \cite{mcinnes_umap_2018} projection illustrates multi-unit spikes during positive sharp wave ripples (SWR$^+$) candidates (\textit{purple}) and negative sharp wave ripples (SWR$^-$) candidates (\textit{yellow}). \textbf{\textit{B.}} A cumulative density plot shows silhouette scores, which reflect the quality of UMAP clustering (refer to Table~\ref{tab:02}). Presumed CA1 regions are identified by silhouette scores greater than 0.60 (equivalent to the 75th percentile). Hits and misses recorded from these regions are classified as SWR$^+$ and SWR$^-$ accordingly (\textit{n}s = 1,170). \textbf{\textit{C.}} Consistent distributions of durations for SWR$^+$ (\textit{purple}) and SWR$^-$ (\textit{yellow}) are demonstrated, as defined by the median interquartile range (IQR) of 93.0 [65.4] ms. \textbf{\textit{D.}} The occurrence of SWR$^+$ (\textit{purple}) and SWR$^-$ (\textit{yellow}), in relation to the probe's timing, is depicted as a mean \textpm 95\% confidence interval. Please note that the intervals may not be visibly discernible due to their confined ranges, despite a significant surge in SWR incidence during the initial 400 ms of the retrieval phase (0.421 [Hz], *\textit{p} $<$ 0.05, bootstrap test). \textbf{\textit{E.}} The distributions of ripple band peak amplitudes for SWR$^-$ (\textit{yellow}; 2.37 [0.33] SD of baseline, median [IQR]) and SWR$^+$ (\textit{purple}; 3.05 [0.85] SD of baseline, median [IQR]) are depicted (***\textit{p} $<$ 0.001, the Brunner--Munzel test).} % width=1\textwidth