\caption{\textbf{
Detection of SWRs in Prospective CA1 Regions
}
\smallskip
\\
\textbf{\textit{A.}} A two-dimensional UMAP (Uniform Manifold Approximation and Projection) \cite{mcinnes_umap_2018} projection that utilizes multiunit spikes is presented during SWR$^+$ candidates (\textit{purple}) and SWR$^-$ candidates (\textit{yellow}). \textbf{\textit{B.}} A cumulative density plot displaying silhouette scores, indicative of the UMAP clustering quality, is provided for hippocampal regions (Table~\ref{tab:02} as reference). Hippocampal regions that exhibit silhouette scores greater than 0.60, equivalent to the $75^{th}$ percentile, are identified as probable CA1 regions. SWR$^+$ and SWR$^-$ candidates obtained from these hypothetical CA1 regions are respectively categorized as SWR$^+$ and SWR$^-$ (with \textit{n}s = 1,170). \textbf{\textit{C.}} The same distributions of durations are presented for both SWR$^+$ (\textit{purple}) and SWR$^-$ (\textit{yellow}) due to their similar natures (93.0 [65.4] ms, expressed as median [IQR]). \textbf{\textit{D.}} An illustration of the SWR incidence for both the SWR$^+$ (\textit{purple}) and SWR$^-$ (\textit{yellow}) relative to the probe's timing is conveyed as a mean with a 95\% confidence interval. However, given that the intervals may not be noticeable due to their slender range, it should be highlighted that a significant rise in SWR incidence was discovered during the beginning 400 ms of the retrieval phase (0.421 [Hz], *\textit{p} $<$ 0.05, verified by a bootstrap test). \textbf{\textit{E.}} The distributions of ripple band peak amplitudes for SWR$^-$ (\textit{yellow}; 2.37 [0.33] SD of baseline, median [IQR]) and SWR$^+$ (\textit{purple}; 3.05 [0.85] SD of baseline, median [IQR]) are outlined (***\textit{p} $<$ 0.001, confirmed by the Brunner--Munzel test).
}
% width=1\textwidth
