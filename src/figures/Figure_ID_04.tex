\caption{\textbf{
Detection of SWRs in Assumed CA1 Regions}
\smallskip
\\
\textbf{\textit{A.}} Two-dimensional Uniform Manifold Approximation and Projection (UMAP) projection of multi-unit spikes during possible SWRs (\textit{purple}) and non-SWRs (\textit{yellow}) periods is presented\cite{mcinnes_umap_2018}. \textbf{\textit{B.}} The cumulative density plot of silhouette scores, which measure the quality of UMAP clustering, for the various hippocampal regions is plotted (see Table~\ref{tab:02}). Areas with a silhouette score above 0.60 (corresponding to the $75^{th}$ percentile) were identified as probable CA1 regions. SWR and non-SWR periods in these potential CA1 regions were defined as SWRs and non-SWRs, respectively (\textit{n}s = 1,170)\cite{rousseeuw_silhouettes_1987}. \textbf{\textit{C.}} The distribution of durations for both SWRs (\textit{purple}) and non-SWRs (\textit{yellow}) are shown, considering their respective definitions (93.0 [65.4] ms, median [IQR])\cite{girardeau_selective_2009}\cite{norman_hippocampal_2021}. \textbf{\textit{D.}} The occurrence rate of SWRs (\textit{purple}) and non-SWRs (\textit{yellow}) over time since stimulation initiation is illustrated as a mean value \textpm 95\% confidence interval. It is important to note that due to the close intervals, visualization may be difficult. Also, a significant increase in SWR occurrence was detected during the first 400 ms of the retrieval phase (0.421 [Hz], *\textit{p} < 0.05, bootstrap test)\cite{buzsaki_hippocampal_2015}\cite{ego-stengel_disruption_2010}\cite{fernandez-ruiz_long-duration_2019}. \textbf{\textit{E.}} Distributions of ripple band peak amplitudes are provided for non-SWRs (\textit{yellow}; 2.37 [0.33] SD of baseline, median [IQR]) and SWRs (\textit{purple}; 3.05 [0.85] SD of baseline, median [IQR]). Significant differences were observed (***\textit{p} < 0.001, using the Brunner--Munzel test)\cite{norman_hippocampal_2019}\cite{diba_forward_2007}\cite{liu_consensus_2022}.
}
% width=1\textwidth