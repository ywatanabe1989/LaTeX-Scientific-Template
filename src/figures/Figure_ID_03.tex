\caption{\textbf{
Dependency of Trajectory Distance on Memory Load: Encoding and Retrieval States in the Hippocampus.
}
\smallskip
\\
\textbf{\textit{A.}} Illustrates the relationship between set size (the number of letters that require encoding) and the correct rate in the working memory task (coefficient = $-0.20$, ***\textit{p} $<$ 0.001). \textbf{\textit{B.}} Showcases the correlation between set size and response time (coefficient = 0.23, ***\textit{p} $<$ 0.001). \textbf{\textit{C.}} Depicts the influence of set size on the inter-phase distances between the encoding and retrieval phases ($\lVert \mathrm{g_{E}g_{R}} \rVert$) (correlation coefficient = 0.05). \textbf{\textit{D.}} \textit{Red} dots represent experimental observations of correlations between set size and the ensuing parameters: correct rate, response time, $\log_{10}{\lVert \mathrm{g_{F}g_{E}} \rVert}$, $\log_{10}{\lVert \mathrm{g_{F}g_{M}} \rVert}$, $\log_{10}{\lVert \mathrm{g_{F}g_{R}} \rVert}$, $\log_{10}{\lVert \mathrm{g_{E}g_{M}} \rVert}$, $\log_{10}{\lVert \mathrm{g_{E}g_{R}} \rVert}$, and $\log_{10}{\lVert \mathrm{g_{M}g_{R}} \rVert}$. The \textit{gray} kernel density plot demonstrates the corresponding set-size-shuffled surrogate (\textit{n} = 1,000) (***\textit{p}s $<$ 0.001).
}
% width=1\textwidth