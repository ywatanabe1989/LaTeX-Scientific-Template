        \clearpage
        \begin{figure*}[ht]
            \pdfbookmark[2]{ID 02}{figure_id_02}
        	\centering
            \includegraphics[width=]{./src/figures/.png/Figure_ID_02.png}
        	\caption{\textbf{
State-dependent hippocampal neural trajectory
}
\smallskip
\\
\textbf{\textit{A.}} The neural trajectory in the first three-dimensional factors computed using GPFA is illustrated. Smaller dots represent the coordinates of 50-ms neural trajectory bins, while larger dots outlined in \textit{black} denote the geometric medians for subsequent phases in the Sternberg working memory task: fixation (\textit{gray}), encoding (\textit{blue}), maintenance (\textit{green}), and retrieval (\textit{red})\cite{yu_gaussian-process_2009}. \textbf{\textit{B.}} This figure demonstrates the log-likelihood of GPFA models in conjunction with the number of dimensions employed to embed multi-unit spikes in MTL regions. Significantly, the optimal dimension was determined to be three, ascertained through the use of the elbow method\cite{virtanen_scipy_2020}. \textbf{\textit{C.}} In this segment, the distance of neural trajectory is mapped from the origin ($O$) for the hippocampus (Hipp.), entorhinal cortex (EC), and amygdala (Amy.), and is plotted in relation to the time from the probe's commencement\cite{boran_dataset_2020}. \textbf{\textit{D.}} The following graph illustrates the trajectory distance from $O$ within MTL regions, whereby the hippocampus exhibits the greatest distance, succeeded by the EC and the Amygdala\cite{fernandez-ruiz_long-duration_2019}. \textbf{\textit{E.}} The subsequent representation indicates inter-phase trajectory distances within the MTL regions\cite{liu_consensus_2022}.
Abbreviations:
}
        	\label{fig:02}
        \end{figure*}
