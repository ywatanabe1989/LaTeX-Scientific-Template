        \clearpage
        \begin{figure*}[ht]
            \pdfbookmark[2]{ID 02}{figure_id_02}
        	\centering
            \includegraphics[width=0.5\textwidth]{./src/figures/.png/Figure_ID_02.png}
        	\caption{\textbf{
State-Dependent Trajectories of Hippocampal Neurons
}
\smallskip
\\
\textbf{\textit{A.}} This panel depicts neural trajectories within the first three-dimensional factors derived from Gaussian Process Factor Analysis (GPFA). Smaller dots represent coordinates corresponding to 50-ms neural trajectory bins. Meanwhile, larger dots with \textit{black} borders denote geometric medians for different stages in the Sternberg working memory task, as follows: fixation (\textit{gray}), encoding (\textit{blue}), maintenance (\textit{green}), and retrieval (\textit{red}). \textbf{\textit{B.}} The graph presents the log-likelihood of the GPFA models compared with the number of dimensions used to embed multiunit spikes from the medial temporal lobe (MTL) territories. Specifically, the optimal dimension, as indicated by the elbow method, was found to be three. \textbf{\textit{C.}} This figure illustrates the distance of the neural trajectories from the origin ($O$) for the hippocampus (Hipp.), entorhinal cortex (EC), and amygdala (Amy.), over time following the probe onset. \textbf{\textit{D.}} The panel displays distances of the trajectories from $O$ within the MTL areas. The hippocampus demonstrates the greatest distance, followed by the EC and the Amygdala. \textbf{\textit{E.}} This graph displays the distances between phases of the trajectories within the MTL regions.
Abbreviations:
}
% width=0.5\textwidth
        	\label{fig:02}
        \end{figure*}
