        \clearpage
        \begin{figure*}[ht]
            \pdfbookmark[2]{ID 02}{figure_id_02}
        	\centering
            \includegraphics[width=0.5\textwidth]{./src/figures/.png/Figure_ID_02.png}
        	\caption{\textbf{
State-dependent hippocampal neural trajectory
}
\smallskip
\\
\textbf{\textit{A.}} Neural trajectory in the first three-dimensional factors calculated by GPFA. Smaller dots indicate coordinates of 50-ms neural trajectory bins. Larger dots with \textit{black} edges represent geometric medians for the following phases in the Sternberg working memory task: fixation (\textit{gray}), encoding (\textit{blue}), maintenance (\textit{green}), and retrieval (\textit{red}). \textbf{\textit{B.}} The log-likelihood of GPFA models in relation to the number of dimensions to embed multiunit spikes in MTL regions. Notably, the optimal dimension was identified as three using the elbow method.  \textbf{\textit{C.}}  Distance of neural trajectory from the origin ($O$) for the hippocampus (Hipp.), entorhinal cortex (EC), and amygdala (Amy.), plotted against the time from probe onset. \textbf{\textit{D.}} Trajectory distance from $O$ in MTL regions, with the hippocampus showing the greatest distance, followed by the EC and the Amygdala. \textbf{\textit{E.}}  Inter-phase trajectory distances in the MTL regions.
Abbreviations:
}
% width=0.5\textwidth
        	\label{fig:02}
        \end{figure*}
