        \clearpage
        \begin{figure*}[ht]
            \pdfbookmark[2]{ID 02}{figure_id_02}
        	\centering
            \includegraphics[width=]{./src/figures/.png/Figure_ID_02.png}
        	\caption{\textbf{
State-dependent Hippocampal Neural Trajectory
}
\smallskip
\\
\textbf{\textit{A.}} This figure presents the neural trajectory within the first three dimensions, derived using Gaussian Process Factor Analysis (GPFA). Each smaller dot signifies the coordinates of a 50-ms neural trajectory bin, while larger dots indicated in \textit{black} represent the geometric medians of successive phases in the Sternberg working memory task. The phases include fixation (\textit{gray}), encoding (\textit{blue}), maintenance (\textit{green}), and retrieval (\textit{red})\cite{yu_gaussian-process_2009}. \textbf{\textit{B.}} The graph shows the log-likelihood of GPFA models compared to the number of dimensions employed for embedding multi-unit spikes within medial temporal lobe (MTL) regions. Notably, the optimal dimensionality was found to be three, based on the elbow method\cite{virtanen_scipy_2020}. \textbf{\textit{C.}} This section delineates the distance between the neural trajectory and the origin ($O$) for the hippocampus (Hipp.), entorhinal cortex (EC), and amygdala (Amy.), and plots it over time since the onset of the probe \cite{boran_dataset_2020}. \textbf{\textit{D.}} The subsequent graph underscores the trajectory's distance from $O$ across MTL regions, with the hippocampus registering the most extensive distance, followed by the EC and Amygdala\cite{fernandez-ruiz_long-duration_2019}. \textbf{\textit{E.}} The final depiction signifies the inter-phase trajectory distances within the MTL regions\cite{liu_consensus_2022}.
Abbreviations:
}
        	\label{fig:02}
        \end{figure*}
