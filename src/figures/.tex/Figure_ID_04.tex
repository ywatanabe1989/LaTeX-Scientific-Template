        \clearpage
        \begin{figure*}[ht]
            \pdfbookmark[2]{ID 04}{figure_id_04}
        	\centering
            \includegraphics[width=1\textwidth]{./src/figures/.png/Figure_ID_04.png}
        	\caption{\textbf{Detection of SWRs in Presumed CA1 Regions}
\smallskip
\\
\textbf{\textit{A.}} A two-dimensional Uniform Manifold Approximation and Projection (UMAP) projection of multi-unit spikes during potential SWRs (\textit{purple}) and non-SWRs (\textit{yellow}) periods is given\cite{mcinnes_umap_2018}. \textbf{\textit{B.}} The cumulative density plot of silhouette scores, measuring the quality of UMAP clustering across diverse hippocampal regions, is shown (refer to Table~\ref{tab:02}). Regions that attained a silhouette score above 0.60 (corresponding to the $75^{th}$ percentile), are identified as probable CA1 areas. Within these potential CA1 regions, the SWR and non-SWR periods were respectively categorized as SWRs and non-SWRs (\textit{n} = 1,170)\cite{rousseeuw_silhouettes_1987}. \textbf{\textit{C.}} The distributions of durations for both SWRs (\textit{purple}) and non-SWRs (\textit{yellow}) are depicted, based on their respective definitions (93.0 [65.4] ms, median [IQR])\cite{girardeau_selective_2009}\cite{norman_hippocampal_2021}. \textbf{\textit{D.}} An illustration of the frequency of SWRs (\textit{purple}) and non-SWRs (\textit{yellow}) over time from the start of stimulation, represented by mean value \textpm 95\% confidence interval is given. It should be noted that due to close intervals, visual differentiation can be difficult. Additionally, there was a discernible increase in SWR frequency during the initial 400 ms of the retrieval phase (0.421 [Hz], *\textit{p} < 0.05, bootstrap test)\cite{buzsaki_hippocampal_2015}\cite{ego-stengel_disruption_2010}\cite{fernandez-ruiz_long-duration_2019}. \textbf{\textit{E.}} Distributions of ripple band peak amplitudes for non-SWRs (\textit{yellow}; 2.37 [0.33] times the standard deviation (SD) of the baseline, median [IQR]) and SWRs (\textit{purple}; 3.05 [0.85] times the SD of the baseline, median [IQR]) are exhibited. Considerable differences were observed (***\textit{p} < 0.001, by the Brunner--Munzel test)\cite{norman_hippocampal_2019}\cite{diba_forward_2007}\cite{liu_consensus_2022}.
}
% width=1\textwidth
        	\label{fig:04}
        \end{figure*}
