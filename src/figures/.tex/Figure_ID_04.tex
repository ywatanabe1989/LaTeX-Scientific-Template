        \clearpage
        \begin{figure*}[ht]
            \pdfbookmark[2]{ID 04}{figure_id_04}
        	\centering
            \includegraphics[width=1\textwidth]{./src/figures/.png/Figure_ID_04.png}
        	\caption{\textbf{
Detection of SWRs in Presumed CA1 Regions
}
\smallskip
\\
\textbf{\textit{A.}} Two-dimensional UMAP (uniform manifold approximation and projection) projection of multiunit spikes during candidates for SWRs (\textit{purple}) and non-SWRs (\textit{yellow})\cite{mcinnes_umap_2018}. \textbf{\textit{B.}}  Cumulative density plot of silhouette scores, utilized as a gauge for UMAP clustering quality, for hippocampal regions (refer to Table~\ref{tab:02}). Lands exceeding a silhouette score of 0.60 (equating to the $75^{th}$ percentile) were classified as probable CA1 regions. Candidates for SWRs and non-SWRs identified in these hypothetical CA1 regions were defined as SWRs and non-SWRs (\textit{n}s = 1,170), respectively\cite{rousseeuw_silhouettes_1987}. \textbf{\textit{C.}}  The distribution of durations for both SWRs (\textit{purple}) and non-SWRs (\textit{yellow}) are congruent, given their definitions (93.0 [65.4] ms, median [IQR])\cite{girardeau_selective_2009}\cite{norman_hippocampal_2021}. \textbf{\textit{D.}}  SWR incidence for SWRs (\textit{purple}) and non-SWRs (\textit{yellow}) over time from probing, represented as a mean value \textpm 95\% confidence interval. Despite their singular character, the intervals may not be visible due to their narrowness. Note that a significant uptake in SWR incidence was perceived during the first 400 ms of the retrieval phase (0.421 [Hz], *\textit{p} < 0.05, bootstrap test)\cite{buzsaki_hippocampal_2015}\cite{ego-stengel_disruption_2010}\cite{fernandez-ruiz_long-duration_2019}. \textbf{\textit{E.}}  The distributions of ripple band peak amplitudes are presented for non-SWRs (\textit{yellow}; 2.37 [0.33] SD of baseline, median [IQR]) and SWRs (\textit{purple}; 3.05 [0.85] SD of baseline, median [IQR]) (***\textit{p} < 0.001, utilizing the Brunner--Munzel test)\cite{norman_hippocampal_2019}\cite{diba_forward_2007}\cite{liu_consensus_2022}.
}
% width=1\textwidth
        	\label{fig:04}
        \end{figure*}
