        \clearpage
        \begin{figure*}[ht]
            \pdfbookmark[2]{ID 04}{figure_id_04}
        	\centering
            \includegraphics[width=1\textwidth]{./src/figures/.png/Figure_ID_04.png}
        	\caption{\textbf{
Detection of SWRs in Presumptive CA1 Regions
}
\smallskip
\\
\textbf{\textit{A.}} Two-dimensional UMAP (Uniform Manifold Approximation and Projection) \cite{mcinnes_umap_2018} projection of multiunit spikes during SWR$^+$ candidates (\textit{purple}) and SWR$^-$ candidates (\textit{yellow}). \textbf{\textit{B.}} Cumulative density plot shows silhouette scores, indicative of UMAP clustering quality, for hippocampal regions (see Table~\ref{tab:02} for reference). Note that hippocampal regions with silhouette scores greater than 0.60 (equivalent to the $75^{th}$ percentile) were identified as possible CA1 regions. SWR$^+$ and SWR$^-$ candidates recorded from these speculative CA1 regions were respectively classified as SWR$^+$ and SWR$^-$ (\textit{n}s = 1,170). \textbf{\textit{C.}} The identical distributions of durations are presented for SWR$^+$ (\textit{purple}) and SWR$^-$ (\textit{yellow}), owing to their definitions (93.0 [65.4] ms, median [IQR]). \textbf{\textit{D.}} SWR incidence for both SWR$^+$ (\textit{purple}) and SWR$^-$ (\textit{yellow}) obtained relative to the probe's timing is illustrated as a mean \textpm 95\% confidence interval. However, as the intervals may not be visible due to their narrow range, note that a significant increase in SWR incidence was detected during the initial 400 ms of the retrieval phase (0.421 [Hz], *\textit{p} $<$ 0.05, bootstrap test). \textbf{\textit{E.}} The distributions of ripple band peak amplitudes for SWR$^-$ (\textit{yellow}; 2.37 [0.33] SD of baseline, median [IQR]) and SWR$^+$ (\textit{purple}; 3.05 [0.85] SD of baseline, median [IQR]) are delineated (***\textit{p} $<$ 0.001, the Brunner--Munzel test).
}
% width=1\textwidth
        	\label{fig:04}
        \end{figure*}
