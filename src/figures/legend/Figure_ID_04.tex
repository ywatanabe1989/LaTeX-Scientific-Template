\caption{\textbf{
SWR detection in putative CA1 regions
}
\smallskip
\\
\textbf{\textit{A.}} Two-dimensional UMAP (uniform manifold approximation and projection)\cite{mcinnes_umap_2018} projection of multiunit spikes during SWR$^+$ candidates (\textit{purple}) and SWR$^-$ candidates (\textit{yellow}). \textbf{\textit{B.}}  Cumulative density plot of silhouette scores, a barometer for UMAP clustering quality, for hippocampal regions (refer to Table 2). Note that hippocampal regions with silhouette scores exceeding 0.60 (= $75^{th}$ percentile) were defined as putative CA1 regions. SWR$^+$ and SWR$^-$ candidates recorded in these putative CA1 regions were defined as SWR$^+$ and SWR$^-$ (\textit{n}s = 1,170), respectively. \textbf{\textit{C.}}  The distributions of durations for SWR$^+$ (\textit{purple}) and SWR$^-$ (\textit{yellow}), which are identical due to their definitions (93.0 [65.4] ms, median [IQR]). \textbf{\textit{D.}}  SWR incidence for both SWR$^+$ (\textit{purple}) and SWR$^-$ (\textit{yellow}) relative to time from probe, represented as mean \textpm 95\% confidence interval, although the intervals might not be visible due to their narrow range. Note the significant elevation in SWR incidence was detected during the first 400 ms of the retrieval phase (0.421 [Hz], *\textit{p} $<$ 0.05, bootstrap test). \textbf{\textit{E.}}  The distributions of ripple band peak amplitude for SWR$^-$ (\textit{yellow}; 2.37 [0.33] SD of baseline, median [IQR]) and SWR$^+$ (\textit{purple}; 3.05 [0.85] SD of baseline, median [IQR]) (***\textit{p} $<$ 0.001, the Brunner--Munzel test).
}
% width=1\textwidth