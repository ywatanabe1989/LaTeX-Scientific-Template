\caption{\textbf{Visualizing Neural Trajectories During Sharp-Wave Ripple Events in a Two-Dimensional Space}

\smallskip

\\
This figure demonstrates the association of neural trajectories with hippocampal activity during Sharp-Wave Ripple (SWR) events in a two-dimensional context. \textbf{\textit{A.}} It depicts example trajectories of the pre- (\textit{gray}), mid- (\textit{yellow}), and post-SWR$^-$ (\textit{black}) phases of an SWR event~\cite{buzsaki_hippocampal_2015}. \textbf{\textit{B.}} The trajectories that correspond with SWR$^+$ conditions are presented, contrasting with the SWR$^-$ backdrop~\cite{fernandez-ruiz_long-duration_2019}. Variations in the magnitude of $\lVert \mathrm{g_{E}g_{R}} \rVert$ are evident across sessions~\cite{liu_consensus_2022}. The projection protocol is outlined as follows: initially, $\mathrm{g_{E}}$ was located at the origin $O$ (0,0) and $\mathrm{g_{R}}$ at ($\lVert \mathrm{g_{E}g_{R}} \rVert$, 0), realized through linear transformation~\cite{kim_corticalhippocampal_2022}. Subsequently, rotation of the point cloud around the $\mathrm{g_{E}g_{R}}$ axis (the x-axis) was conducted to accommodate a two-dimensional space~\cite{yu_gaussian-process_2009}. As a result, the distances from $O$ and the angles relative to the $\mathrm{g_{E}g_{R}}$ axis remained consistent with their original three-dimensional configuration~\cite{mcinnes_umap_2018}. Key terms used in this context: SWR pertains to Sharp-Wave Ripple events; eSWR means SWR during the encoding phase; rSWR signifies SWR during the retrieval phase; SWR$^+$ defines an SWR event; SWR$^-$ represents the control event for SWR$^+$; pre-SWR, mid-SWR, and post-SWR indicate time intervals from $-800$ to $-250$ ms, from $-250$ to $+250$ ms, and from $+250$ to $+800$ ms from the center of an SWR event, respectively~\cite{zhang_hippocampal_2022}.
}
% width=1\textwidth
