%% preamble
\usepackage[english]{babel}
\usepackage[table]{xcolor} % For coloring tables
\usepackage{booktabs} % For professional quality tables
\usepackage{colortbl} % For coloring cells in tables
\usepackage{amsmath, amssymb} % For mathematical symbols and environments
\usepackage{amsthm} % For theorem-like environments
\usepackage{lipsum} % just for sample text
\usepackage{natbib}
\usepackage{graphicx}
\usepackage{indentfirst}
\usepackage{bashful}
\usepackage[margin=10pt,font=small,labelfont=bf,labelsep=endash]{caption}
\usepackage{graphicx}
\usepackage{calc}
\usepackage[T1]{fontenc} % [REVISED]
\usepackage[utf8]{inputenc} % [REVISED]
\usepackage{hyperref}
\usepackage{accsupp}
%% Line numbers
\linespread{1.1}
% \linenumbers
% Tables
\usepackage[pass]{geometry}
\usepackage{pdflscape}
\usepackage{csvsimple}
\usepackage{xltabular}
\usepackage{booktabs}
\usepackage{siunitx}
\usepackage{makecell}
\sisetup{round-mode=figures,round-precision=3}
\renewcommand\theadfont{\bfseries}
\renewcommand\theadalign{c}
\newcolumntype{C}[1]{>{\centering\arraybackslash}m{#1}}
\renewcommand{\arraystretch}{1.5}
\definecolor{lightgray}{gray}{0.95}

%% Diff
\usepackage{xcolor}
% Define commands for highlighting
% diff
\usepackage[most]{tcolorbox} % for boxes with transparency
% Define colors with transparency (opacity value)
\definecolor{GreenBG}{rgb}{0,1,0}
\definecolor{RedBG}{rgb}{1,0,0}
% Define tcolorbox environments for highlighting
\newtcbox{\greenhighlight}[1][]{%
  on line,
  colframe=GreenBG,
  colback=GreenBG!50!white, % 50% transparent green
  boxrule=0pt,
  arc=0pt,
  boxsep=0pt,
  left=1pt,
  right=1pt,
  top=2pt,
  bottom=2pt,
  tcbox raise base
}
\newtcbox{\redhighlight}[1][]{%
  on line,
  colframe=RedBG,
  colback=RedBG!50!white, % 50% transparent red
  boxrule=0pt,
  arc=0pt,
  boxsep=0pt,
  left=1pt,
  right=1pt,
  top=2pt,
  bottom=2pt,
  tcbox raise base
}
\newcommand{\REDSTARTS}{\color{red}}
\newcommand{\REDENDS}{\color{black}}
\newcommand{\GREENSTARTS}{\color{green}}
\newcommand{\GREENENDS}{\color{black}}