%%%%%%%%%%%%%%%%%%%%%%%%%%%%%%%%%%%%%%%%%%%%%%%%%%%%%%%%%%%%%%%%%%%%%%%%%%%%%%%%
%% Results
%%%%%%%%%%%%%%%%%%%%%%%%%%%%%%%%%%%%%%%%%%%%%%%%%%%%%%%%%%%%%%%%%%%%%%%%%%%%%%%%
\section{Results}
\subsection{iEEG recording and neural trajectory in MTL regions during a Sternberg task}
We employed a publicly available dataset \cite{boran_dataset_2020} for this analysis. This dataset recorded LFP signals (Figure 1A) within the MTL regions (Table 1) during a modified Sternberg task. SWR$^+$ candidates were identified from LFP signals passed through the ripple band (Figure 1B) within all hippocampal regions (refer to Methods), while SWR$^-$ candidates were designated at identical timestamps of the SWR$^+$ candidates but shuffling them across different trials (Figure 1). The multiunit spikes (Figure 1C) were included in the dataset, being established using a spike sorting algorithm \cite{niediek_reliable_2016}. Using the 50-ms binned multiunit activity without overlaps, we employed GPFA \cite{yu_gaussian-process_2009} to determine the neural trajectory (or factors) of the MTL regions by session and region (Figure 1D). Each factor was z-normalized by session and region (for example, session \#2 in AHL of subject \#1). Subsequently, the Euclidean distance from the origin ($O$) was calculated (Figure 1E).

\subsection{Hippocampal neural trajectory correlated with a WM task}
In Figure 2A, the median neural trajectories of 50 trials are depicted as points within the three major factor space. The optimal embedding dimension for the GPFA model was determined to be three using the elbow method (Figure 2B). The trajectory distance from the origin ($O$) for the hippocampus was larger compared to the EC and amygdala (Figure C \& D).\footnote{Hippocampus: Distance = 1.11 [1.01], median [IQR], \textit{n} = 195,681 timepoints; EC: Distance = 0.94 [1.10], median [IQR], \textit{n} = 133,761 timepoints; Amygdala: Distance = 0.78 [0.88], median [IQR], \textit{n} = 165,281 timepoints.}
\\
\indent
Similarly, the distance among geometric medians of the four phases were calculated: \textit{i.e.}, $\mathrm{\lVert g_{F}g_{E} \rVert}$) for the distance between fixation and encoding states. Again, the hippocampus showed larger distances among phases compared to both the EC and amygdala. \footnote{Hippocampus: Distance = 0.60 [0.70], median [IQR], \textit{n} = 8,772 combinations; EC: Distance = 0.28 [0.52], median [IQR], \textit{n} = 5,017 combinations (\textit{p} $<$ 0.01; Brunner--Munzel test); Amygdala: Distance = 0.24 [0.42], median [IQR], \textit{n} = 7,466 combinations (\textit{p} $<$ 0.01; Brunner--Munzel test).}

\subsection{Memory load-dependent neural trajectory distance between the encoding and retrieval states in the hippocampus}
Correct rate of trials and set size (the number of alphabetical letters to encode) were negatively correlated (Figure 3A). \footnote{Correct rate: set size four (0.99 \textpm 0.11, mean \textpm SD; \textit{n} = 333 trials) vs. set size six (0.93 \textpm 0.26; \textit{n} = 278 trials; \textit{p} $<$ 0.001, Brunner--Munzel test with Bonferroni correction) and set size eight (0.87 \textpm 0.34; \textit{n} = 275 trials; \textit{p} $<$ 0.05; Brunner--Munzel test with Bonferroni correction). Overall, \textit{p} $<$ 0.001 for Kruskal--Wallis test; correlation coefficient = - 0.20, \textit{p} $<$ 0.001.} Similarly, response time and set size were positively correlated (Figure 3B).\footnote{Response time: set size four (1.26 \textpm 0.45 s; \textit{n} = 333 trials) vs. set size six (1.53 \textpm 0.91 s; \textit{n} = 278 trials) and set size eight (1.66 \textpm 0.80 s; \textit{n} = 275 trials). All comparisons \textit{p} $<$ 0.001, Brunner--Munzel test with Bonferroni correction; \textit{p} $<$ 0.001 for Kruskal--Wallis test; correlation coefficient = 0.22, \textit{p} $<$ 0.001}
\\
\indent
Set size and the trajectory distance between the encoding and retrieval phases ($\mathrm{log_{10}\lVert g_{E}g_{R} \rVert}$) were positively correlated (Figure 3C).\footnote{Correlation between set size and $\mathrm{log_{10}(\lVert g_{E}g_{R} \rVert}$): correlation coefficient = 0.05, \textit{p} $<$ 0.001. Specific values: $\mathrm{\lVert g_{E}g_{R} \rVert}$ = 0.54 [0.70] for set size four trials, \textit{n} = 447; $\mathrm{\lVert g_{E}g_{R} \rVert}$ = 0.58 [0.66] for set size six trials, \textit{n} = 381; $\mathrm{\lVert g_{E}g_{R} \rVert}$ = 0.61 [0.63] for set size eight trials, \textit{n} = 395.}. However, no significant correlations were found between set size and distance among other phase combinations (Figures 3D \& S2).

\subsection{Detection of hippocampal SWR from putative CA1 regions}
Under the aim to improve the precision of recording sites and the detection of SWRs, we identified electrodes in putative CA1 regions of the hippocampus based on observing distinct multiunit spike patterns during SWR events. For each session and hippocampal region, SWR$^+$/SWR$^-$ candidates were embedded into a two-dimensional space via UMAP (Figure 4A).\footnote{For illustrative purposes, consider the AHL in session \#1 of subject \#1.} We calculated the silhouette score as a measure of clustering quality (Figure 4B \& Table 2). Recording sites with an average silhouette score across sessions exceeding 0.6 were designated as putative CA1 regions\footnote{The identified regions were: AHL of subject \#1, AHR of subject \#3, PHL of subject \#4, AHL of subject \#6, and AHR of subject \#9.}  (Tables 2 \& 3). Consequently, four regions out of the five putative CA1 areas had not been identified as seizure onset zones, while one had been designated (Table 1).
\\
\indent
Subsequently, SWR$^+$/SWR$^-$ candidates within these putative CA1 regions were labeled SWR$^+$ and SWR$^-$, respectively\footnote{Definitions lead to equal counts for both categories: SWR$^+$ (\textit{n} = 1,170) and SWR$^-$ (\textit{n} = 1,170).}  (Table 3). Both SWR$^+$ and SWR$^-$ exhibited an identical duration\footnote{Definitions lead to equal duration for both categories: SWR$^+$ (93.0 [65.4] ms) and SWR$^-$ (93.0 [65.4] ms).}  (Figure 4C) due to their definitions, following a log-distribution profile. A marked increase in SWR$^+$ incidence was detected during the initial 400 ms from the onset of the retrieval phase \footnote{SWR$^+$ increased against the bootstrap sample; 95th percentile = 0.42 [Hz]; \textit{p} $<$ 0.05.}  (Figure 4D). Additionally, the peak ripple band amplitude for SWR$^+$ exceeded that of SWR$^-$ and followed a log-normal distribution (Figure 4E).\footnote{SWR$^+$ (3.05 [0.85] SD of baseline, median [IQR]; \textit{n} = 1,170) vs. SWR$^-$ (2.37 [0.33] SD of baseline, median [IQR]; \textit{n} = 1,170; \textit{p} $<$ 0.001; Brunner--Munzel test).}.

\subsection{Transient neural trajectory change in the hippocampus during SWR}
The distances of trajectory from the origin ($O$) during SWR events in both the encoding and retrieval phases (\textit{\textit{i.e.}}, eSWR$^+$, eSWR$^-$, rSWR$^+$, and rSWR$^-$) were calculated (Figure 5A). Given the pronounced peak in them, we categorized each SWR into three stages: pre-SWR, mid-SWR, and post-SWR. Subsequently, the distances from $O$ during these SWR periods are represented as $\mathrm{\lVert \text{pre-eSWR}^+ \rVert}$, $\mathrm{\lVert \text{mid-eSWR}^+ \rVert}$, and so on.
\\
\indent
$\mathrm{\lVert \text{mid-eSWR}^+ \rVert}$
\footnote{1.25 [1.30], median [IQR], \textit{n} = 1,281, in Match IN task; 1.12 [1.35], median [IQR], \textit{n} = 1,163, in Mismatch OUT task}
was larger than $\mathrm{\lVert \text{pre-eSWR}^+ \rVert}$
\footnote{1.08 [1.07], median [IQR], \textit{n} = 1,149, in Match IN task; 0.90 [1.12], median [IQR], \textit{n} = 1,088, in Mismatch OUT task}
(Figure 5B). Similarly, $\mathrm{\lVert \text{mid-rSWR}^+ \rVert}$
\footnote{1.32 [1.24], median [IQR], \textit{n} = 935, in Match IN task; 1.15 [1.26], median [IQR], \textit{n} = 891, in Mismatch OUT task}
was larger than $\mathrm{\lVert \text{pre-rSWR}^+ \rVert}$.
\footnote{1.19 [0.96], median [IQR], \textit{n} = 673, in Match IN task; 0.94 [0.88], median [IQR], \textit{n} = 664, in Mismatch OUT task}

\subsection{Visualization of hippocampal neural trajectory during SWR in two-dimensional spaces}
Based on our observations of neural trajectory 'jump' during SWR (Figure 5), we visualized the trajectories of pre-, mid-, and post-SWR events during the encoding and retrieval phases (Figure 6), the distance of which was memory-load dependendent (Figure 3).
\\
\indent
To achieve the visualization in two dimension spaces, the peri-SWR trajectories were aligned linearly by positioning $\mathrm{g_{E}}$ at the origin (0, 0) and placing $\mathrm{g_{R}}$ at ($\mathrm{\lVert g_{E}g_{R} \rVert}$, 0). These aligned trajectories were rotated around the x axis for visualization purposes. Importantly, distances and angles in the original three-dimensional spaces are preserved in these two-dimensional ones.
\\
\indent
The scatter plot in these two-dimensional spaces illustrates characteristic distributions of peri-SWR trajectories based on phases and task types. For instance, longer distance of mid-eSWR$^+$, compaired to pre-eSWR$^-$, from $\mathrm{g_{E}}$ can be observed (Figure 6B), consistent with our earlier findings (Figure 5).

\subsection{Fluctuations of hippocampal neural trajectories between encoding and retrieval states}
Subsequently, we checked trajectory directions based on $\overrightarrow{\mathrm{g_{E}g_{R}}}$. SWR directions were defined by neural trajectory at $-250$ ms and $+250$ ms from their center (\textit{i.e.}, $\overrightarrow{\mathrm{eSWR^+}}$).
\\
\indent
$\overrightarrow{\mathrm{eSWR^+}}\cdot\overrightarrow{\mathrm{rSWR^+}}$ showed a bias towards $+1$ in Match IN task (Figure 7A) but towards $-1$ in Mismatch OUT task (Figure 7B). These tendencies were also observed in $\overrightarrow{\mathrm{eSWR^-}}\cdot\overrightarrow{\mathrm{rSWR^-}}$ (Figure 7C--F).
\\
\indent
Moreover, $\overrightarrow{\mathrm{rSWR^+}}\cdot\overrightarrow{\mathrm{g_{E}g_{R}}}$ showed a biphasic distribution in Match In task (Figure 7A) in contrast to a monophasic distribution in Mismatch In task (Figure 7B).
\\
\indent
Both in Match IN and Mismatch OUT tasks, by taking the differences between the distribution of $\overrightarrow{\mathrm{rSWR^+}}\cdot\overrightarrow{\mathrm{g_{E}g_{R}}}$ (Figure 7A \& B) and that of $\overrightarrow{\mathrm{rSWR^-}}\cdot\overrightarrow{\mathrm{g_{E}g_{R}}}$ (Figure 7C \& D), the effect of SWR was revealed as a shift from $\mathrm{g_{E}}$ to $\mathrm{g_{R}}$ (Figure 7E \& F).
\label{sec:results}
