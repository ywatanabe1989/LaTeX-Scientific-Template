\begin{abstract}
\pdfbookmark[1]{Abstract}{abstract}
Working memory (WM) is critical for various cognitive functions, yet the underlying neural mechanisms remain obscure. Although the hippocampus and sharp-wave ripple complexes (SWRs) --- transient and synchronous neural bursts in the hippocampus --- are recognized for their roles in memory consolidation and retrieval, their relationship with WM tasks is poorly understood. This study hypothesizes that multiunit activity patterns in the human hippocampus, in conjunction with SWRs, exhibit distinct behaviors during WM tasks. To test this hypothesis, we employed a dataset consisting of intracranial electroencephalogram recordings from the medial temporal lobe (MTL) of nine epilepsy patients during an eight-second Sternberg test. Gaussian-process factor analysis was utilized to derive low-dimensional neural representations, or 'trajectories', specific to WM tasks within MTL areas. Our findings demonstrate that the hippocampus displays the most significant variations in neural trajectory compared to the entorhinal cortex and amygdala. Moreover, the distance of trajectories between encoding and retrieval phases was memory-load dependent. Notably, the hippocampal trajectory fluctuated between the encoding and retrieval stats in a task-dependent manner, with a shift from encoding to retrieval states during SWRs. These observations underscore the hippocampus's pivotal role in WM tasks and provide novel insights into the functional contributions of the hippocampus to WM demands.
\end{abstract}