\section{Methods}
\subsection{Dataset}
A publicly available dataset \cite{boran_dataset_2020} was used, which consists of nine epilepsy patients performing a modified Sternberg task. This task involves four phases: fixation (1s), encoding (2s), maintenance (3s), and retrieval (2s) \cite{boran_dataset_2020}. During the encoding phase, participants were exposed to four, six, or eight alphabet letters, referred to as the set size. Subsequently, they had to decide whether a probe letter presented during the retrieval phase was previously displayed (the correct choice for the Match IN task) or not (the correct choice for the Mismatch OUT task). iEEG signals were recorded at a sampling rate of 32 kHz, within a frequency range of 0.5--5,000 Hz, using depth electrodes implanted in the medial temporal lobe (MTL) regions: the anterior head of the left and the right hippocampus (AHL and AHR), the posterior body of the hippocampus (PHL and PHR), the entorhinal cortex (ECL and ECR), and the amygdala (AL and AR), as illustrated in Figure~\ref{fig:01}A and Table~\ref{tab:01}. The iEEG signals were subsequently downsampled to a rate of 2 kHz. Correlations among variables such as set size and correct rate were investigated (Figure~\ref{fig:s01}S1). The timings of multiunit spikes were determined by a spike sorting algorithm \cite{niediek_reliable_2016} using the Combinato package (\url{https://github.com/jniediek/combinato})(Figure~\ref{fig:01}C).

\subsection{Calculation of neural trajectories using GPFA}
Neural trajectories, also termed 'factors' (Figure~\ref{fig:01}D), in the hippocampus, EC, and amygdala (Figure~\ref{fig:01}D), were computed using GPFA \cite{yu_gaussian-process_2009} applied to the multiunit activity data for each session. GPFA was performed with the elephant package (\url{https://elephant.readthedocs.io/en/latest/reference/gpfa.html}). The bin size was set to 50 ms, with no overlaps. Each factor was z-normalized across all sessions. The Euclidean distance from the origin ($O$) was then calculated (Figure~\ref{fig:01}E).
\\
\indent
For each trajectory within a region, for instance, AHL, \textit{geometric medians} (i.e., $\mathrm{g_{F}}$ for fixation, $\mathrm{g_{E}}$ for encoding, $\mathrm{g_{M}}$ for maintenance, and $\mathrm{g_{R}}$ for retrieval phase) were determined by calculating the median coordinates of the trajectory during the four phases (Figure~\ref{fig:01}D). An optimal dimensionality for GPFA was identified as three using the elbow method, which was derived by investigating the log-likelihood values through a three-fold cross-validation approach (Figure~\ref{fig:02}B).

\subsection{Identifying SWR candidates from hippocampal regions}
Potential SWR events within the hippocampus were detected using a widely accepted method \cite{liu_consensus_2022}. LFP signals from a region of interest (ROI), such as AHL, were re-referenced by subtracting an averaged signal from locations outside the ROI (\textit{e.g.}, AHR, PHL, PHR, ECL, ECR, AL, and AR) (see Figure~\ref{fig:01}A). The re-referenced LFP signals were then filtered with a ripple-band filter (80--140 Hz) to identify SWR candidates (=$\textrm{SWR}^+$ candidates) (see Figure~\ref{fig:01}B). SWR detection was conducted using a published tool (\url{https://github.com/Eden-Kramer-Lab/ripple_detection}) \cite{kay_hippocampal_2016}, with the bandpass range adjusted to 80--140 Hz for humans \cite{norman_hippocampal_2019} \cite{norman_hippocampal_2021}, different from the original 150--250 Hz range typically applied to rodents.
\\
\indent
Control events for $\textrm{SWR}^+$ candidates, labeled as $\textrm{SWR}^-$ candidates, were identified by randomly shuffling the timestamps of $\textrm{SWR}^+$ candidates across all trials and subjects. The resulting $\textrm{SWR}^+/\textrm{SWR}^-$ candidates were then subjected to visual inspection, as shown in Figure~\ref{fig:01}.

\subsection{Defining SWRs from putative hippocampal CA1 regions}
SWRs were distinguished from SWR candidates in presumptive CA1 regions. Initially, these regions were defined as follows: $\textrm{SWR}^+/\textrm{SWR}^-$ candidates in the hippocampus were projected into a two-dimensional space based on overlapping spike counts per unit employing a supervised method using UMAP (Uniform Manifold Approximation and Projection) \cite{mcinnes_umap_2018} (Figure~\ref{fig:04}A). Clustering validation was performed by computing the silhouette score \cite{rousseeuw_silhouettes_1987} from clustered samples (Table~\ref{tab:02}). Regions in the hippocampus, which scored above 0.6 on average across sessions (75th percentile) (Figure~\ref{fig:04}B), were characterized as presumed CA1 regions, identifying five electrode positions from five patients (Table~\ref{tab:03}).
\\
\indent
$\textrm{SWR}^+/\textrm{SWR}^-$ candidates in the assumed CA1 regions were classified as $\textrm{SWR}^+/\textrm{SWR}^-$, thus relinquishing their candidate status. The duration and ripple band peak amplitude of SWRs were observed to follow log-normal distributions (Figure~\ref{fig:04}4C \& E). Each time period of SWR was partitioned relative to the time from the SWR center into pre- (at $-800$ to $-300$ ms from SWR center), mid- (at $-250$ to $+250$ ms), and post-SWR (at $+300$ to $+800$ ms) times.

\subsection{Statistical evaluation}
The Brunner--Munzel test and the Kruskal-Wallis test were performed using the SciPy package in Python \cite{virtanen_scipy_2020}. Correlational analysis was performed by determining the rank of the observed correlation coefficient in its associated set-size-shuffled surrogate using a custom Python script. The bootstrap test was implemented using an in-house Python script.

\label{sec:methods}