```tex
%%%%%%%%%%%%%%%%%%%%%%%%%%%%%%%%%%%%%%%%%%%%%%%%%%%%%%%%%%%%%%%%%%%%%%%%%%%%%%%%
%% Methods
%%%%%%%%%%%%%%%%%%%%%%%%%%%%%%%%%%%%%%%%%%%%%%%%%%%%%%%%%%%%%%%%%%%%%%%%%%%%%%%%
\section{Methods}
\subsection{Dataset}
A publicly accessible dataset was utilized \cite{boran_dataset_2020}, which included nine epilepsy patients performing a modified Sternberg task comprising the subsequent four phases: fixation (1 s), encoding (2 s), maintenance (3 s), and retrieval (2 s) \cite{boran_dataset_2020}. During the encoding phase, participants were shown sets of four, six, or eight alphabetical letters, denoted herein as the set size. During the retrieval phase, participants were required to ascertain whether a probe letter had been previously shown (the correct choice for the Match IN task) or not (the correct choice for the Mismatch OUT task). Intracranial EEG (iEEG) signals were collected using depth electrodes implanted within the medial temporal lobe (MTL) regions: left and right hippocampal head (AHL and AHR), body (PHL and PHR), entorhinal cortex (ECL and ECR), and amygdala (AL and AR). These signals were recorded at a sampling rate of 32 kHz and within a frequency range of 0.5--5,000 Hz (Figure~\ref{fig:01}A and Table~\ref{tab:01}). The iEEG signals were then resampled at a rate of 2 kHz. Correlations between experimental variables such as set size and accuracy rate were uncovered (Figure~\ref{fig:s01}S1). Times of multiunit spikes were estimated using a spike sorting algorithm \cite{niediek_reliable_2016} from the Combinato package (\url{https://github.com/jniediek/combinato})(Figure~\ref{fig:01}C).

\subsection{Calculation of neural trajectories using GPFA}
To extract the neural trajectories (referred to as factors; Figure~\ref{fig:01}D) within the hippocampus, entorhinal cortex (EC), and amygdala, we employed GPFA \cite{yu_gaussian-process_2009} on multiunit activity data for each session (Figure~\ref{fig:01}D). GPFA was implemented using the elephant package (\url{https://elephant.readthedocs.io/en/latest/reference/gpfa.html}). The bin size was configured as 50 ms, with no overlaps. Each factor was z-normalized across all sessions. The Euclidean distance from the origin ($O$) was calculated based on these trajectories (Figure~\ref{fig:01}E).
\\
\indent
Within each trajectory for a region such as AHL, the \textit{geometric medians} (i.e., $\mathrm{g_{F}}$ for fixation, $\mathrm{g_{E}}$ for encoding, $\mathrm{g_{M}}$ for maintenance, and $\mathrm{g_{R}}$ for retrieval phase) were calculated by establishing the median coordinates of the trajectory during the four phases (Figure~\ref{fig:01}D). The optimal dimensionality for GPFA was defined to be three, as determined using the elbow method via log-likelihood values in a three-fold cross-validation approach (Figure~\ref{fig:02}B).

\subsection{Defining SWR candidates from hippocampal regions}
A detection method in line with the field consensus \cite{liu_consensus_2022} was used to identify potential SWR events in the hippocampus. The local field potential (LFP) signals from a region of interest (ROI), such as AHL, were re-referenced by subtracting the average signal outside the ROI (e.g., AHR, PHL, PHR, ECL, ECR, AL, AR) (see Figure~\ref{fig:01}A). These re-referenced LFP signals were used to apply a ripple-band filter (80--140 Hz) and isolate SWR-positive (SWR$^+$) candidates (see Figure~\ref{fig:01}B). SWR detection was performed using a publicly available tool (\url{https://github.com/Eden-Kramer-Lab/ripple_detection}) \cite{kay_hippocampal_2016}, with modifications such as a revised bandpass range of 80--140 Hz for human applications \cite{norman_hippocampal_2019,norman_hippocampal_2021} instead of the original 150--250 Hz range primarily used for rodents.
\\
\indent
For SWR$^+$ candidates, SWR-negative (SWR$^-$) candidates were defined as control events by shuffling the timestamps of SWR$^+$ candidates across all trials and subjects. The defined SWR$^+$/SWR$^-$ candidates were visually inspected (see Figure~\ref{fig:01}).

\subsection{Defining SWRs from putative hippocampal CA1 regions}
We confined SWR candidates within putative CA1 regions to define SWRs. First, putative CA1 regions were identified as follows. SWR$^+$/SWR$^-$ candidates within the hippocampus were embedded into a two-dimensional space based on superimposed spike counts per unit using UMAP (uniform manifold approximation and projection) \cite{mcinnes_umap_2018} in a supervised manner (Figure~\ref{fig:04}A). The silhouette score \cite{rousseeuw_silhouettes_1987}, a measure for validating clustering, was calculated from clustered samples (Table~\ref{tab:02}). Hippocampal regions showcasing an average silhouette score across sessions exceeding the 75th percentile were defined as putative CA1 regions, which resulted in five electrode locations identified from five patients (Table~\ref{tab:03}).
\\
\indent
SWR$^+$/SWR$^-$ candidates within putative CA1 regions were marked as SWRs, thereby they were no longer candidates. SWR duration and ripple band peak amplitude followed a log-normal distribution (Figure~\ref{fig:04}C \& E). As seen in Figure~\ref{fig:01}, SWR$^+$/SWR$^-$ were visually inspected. Each SWR period was divided into pre-SWR (from $-800$ to $-300$ ms from SWR center), mid-SWR (from $-250$ to $+250$ ms), and post-SWR (from $+300$ to $+800$ ms) according to the time from the center of the SWR.

\subsection{Statistical Evaluation}
The Brunner--Munzel and Kruskal-Wallis tests were conducted using the scipy package in Python \cite{virtanen_scipy_2020}. A correlation analysis was performed by determining the rank of the observed correlation coefficient in the set-size-shuffled surrogate dataset, using a custom Python script. Moreover, a bootstrap test was executed using a homemade Python script.

\label{sec:methods}
```