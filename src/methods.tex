```tex
%%%%%%%%%%%%%%%%%%%%%%%%%%%%%%%%%%%%%%%%%%%%%%%%%%%%%%%%%%%%%%%%%%%%%%%%%%%%%%%%
%% Methods
%%%%%%%%%%%%%%%%%%%%%%%%%%%%%%%%%%%%%%%%%%%%%%%%%%%%%%%%%%%%%%%%%%%%%%%%%%%%%%%%
\section{Methods}
\subsection{Dataset}
We employed a publicly accessible dataset \cite{boran_dataset_2020} that incorporates nine epilepsy patients undertaking a modified Sternberg task with four sequential phases: fixation (1 s), encoding (2 s), maintenance (3 s), and retrieval (2 s) \cite{boran_dataset_2020}. Sets of four, six, or eight letters were shown to the participants during the encoding phase, referred to here as the set size. During retrieval, participants identified whether a probe letter was present (the Match IN task) or absent in the previously shown set (the Mismatch OUT task). Intracranial EEG (iEEG) recordings were obtained via depth electrodes positioned in the medial temporal lobe (MTL) regions: left and right hippocampal head (AHL and AHR), body (PHL and PHR), entorhinal cortex (ECL and ECR), and amygdala (AL and AR). These signals were recorded at 32 kHz sampling rate and resampled at 2 kHz, and covered a frequency range of 0.5--5,000 Hz (Figure~\ref{fig:01}A and Table~\ref{tab:01}). Correlations were determined between experimental variables, such as set size and accuracy rate (Figure~\ref{fig:s01}S1). Multiunit spike times were estimated using the Combinato package's spike sorting algorithm \cite{niediek_reliable_2016} (\url{https://github.com/jniediek/combinato}) (Figure~\ref{fig:01}C).

\subsection{Extraction of Neural Trajectories Using GPFA}
Using GPFA \cite{yu_gaussian-process_2009}, we extracted neural trajectories (also referred to as factors; Figure~\ref{fig:01}D) within the hippocampus, entorhinal cortex (EC), and amygdala from the multiunit activity data of each session. This was possible with the elephant package (\url{https://elephant.readthedocs.io/en/latest/reference/gpfa.html}); a bin size of 50 ms was implemented, excluding overlaps. Each factor was normalized (z-normalization) across sessions, and the Euclidean distance from the origin ($O$) was attributed to these trajectories (Figure~\ref{fig:01}E).

In the trajectory of each region (such as AHL), so-called \textit{geometric medians} ($\mathrm{g_{F}}$ for fixation, $\mathrm{g_{E}}$ for encoding, $\mathrm{g_{M}}$ for maintenance, and $\mathrm{g_{R}}$ for retrieval phase) were obtained by determining the median coordinates during the four phases (Figure~\ref{fig:01}D). Using the elbow method on three-fold cross-validated log-likelihood values, optimal dimensionality for GPFA was determined as three (Figure~\ref{fig:02}B).

\subsection{Identifying SWR Candidates from Hippocampal Regions}
An accepted detection method \cite{liu_consensus_2022} aided in identifying potential SWR occurrences within the hippocampus. The regional local field potential (LFP) signals, such as those from AHL, were re-referenced by subtracting the regional mean signal outside the area of interest (e.g., AHR, PHL, PHR, ECL, ECR, AL, AR) (see Figure~\ref{fig:01}A). This re-referenced LFP signal together with a ripple-band filter (80--140 Hz) facilitated the identification of SWR-positive (SWR$^+$) candidates (Figure~\ref{fig:01}B). Public tool-based SWR detection (\url{https://github.com/Eden-Kramer-Lab/ripple_detection}) \cite{kay_hippocampal_2016} with changes like a revised bandpass range of 80--140 Hz for human applications \cite{norman_hippocampal_2019,norman_hippocampal_2021} was employed in place of the typically used rodent range (150--250 Hz). For SWR$^+$, SWR-negative (SWR$^-$) were defined as control events by shuffling timestamps of SWR$^+$ across all trials and subjects; these SWR$^+$ and SWR$^-$ underwent visual inspection (Figure~\ref{fig:01}).

\subsection{Defining SWRs from Putative Hippocampal CA1 Regions}
Potential SWR events were determined within the putative CA1 regions. The possible CA1 regions were identified by embedding SWR$^+$/SWR$^-$ from the hippocampus into a two-dimensional space, supervised using UMAP based on superposed spike counts per unit \cite{mcinnes_umap_2018} (Figure~\ref{fig:04}A). The silhouette score \cite{rousseeuw_silhouettes_1987}, computed from clustered samples (Table~\ref{tab:02}), validated the effectiveness of clustering. Regions exceeding the 75th percentile in average silhouette scores across sessions were labeled as putative CA1 areas, leading to the discovery of five electrode positions in five patients (Table~\ref{tab:03}). 

Thereafter, SWR$^+$/SWR$^-$ within these putative CA1 areas were designated as SWRs, no longer merely candidates. The duration of SWRs and ripple band peak amplitude displayed a log-normal distribution (Figure~\ref{fig:04}C \& E). As shown in Figure~\ref{fig:01}, each SWR, as well as SWR$^+$/SWR$^-$, underwent visual scrutiny. SWR periods were classified into pre-SWR (from $-800$ to $-300$ ms from the SWR center), mid-SWR (from $-250$ to $+250$ ms), and post-SWR (from $+300$ to $+800$ ms), with reference to the time from the SWR's center.

\subsection{Statistical Analysis}
We performed Brunner--Munzel and Kruskal-Wallis tests using the scipy package in Python \cite{virtanen_scipy_2020}. A correlation analysis was conducted to establish the rank of the observed correlation coefficient in the set-size-shuffled surrogate dataset, employing custom Python code. Additionally, a custom Python script facilitated the execution of a bootstrap test.

\label{sec:methods}
```
