```tex
%%%%%%%%%%%%%%%%%%%%%%%%%%%%%%%%%%%%%%%%%%%%%%%%%%%%%%%%%%%%%%%%%%%%%%%%%%%%%%%%
%% Methods
%%%%%%%%%%%%%%%%%%%%%%%%%%%%%%%%%%%%%%%%%%%%%%%%%%%%%%%%%%%%%%%%%%%%%%%%%%%%%%%%
\section{Methods}
\subsection{Dataset}
We utilized a publicly accessible dataset \cite{boran_dataset_2020} wherein nine epilepsy patients performed a modified Sternberg task comprising the following four phases: fixation (1 s), encoding (2 s), maintenance (3 s), and retrieval (2 s) \cite{boran_dataset_2020}. During the encoding phase, participants were presented with sets of four, six, or eight alphabetical letters, which we refer to as the set size. Subsequently, participants' task was to determine whether a probe letter presented during the retrieval phase had been previously displayed (the correct choice for the Match IN task) or not (the correct choice for the Mismatch OUT task). Intracranial EEG (iEEG) signals were captured using depth electrodes implanted within the medial temporal lobe (MTL) regions: the left and right hippocampal head (AHL and AHR), hippocampal body (PHL and PHR), entorhinal cortex (ECL and ECR), and amygdala (AL and AR). These signals were recorded at a sampling rate of 32 kHz and within the frequency range of 0.5--5,000 Hz (Figure~\ref{fig:01}A and Table~\ref{tab:01}). The iEEG signals were then resampled at a rate of 2 kHz. We uncovered correlations between the experimental variables such as set size and accuracy rate (Figure~\ref{fig:s01}S1). The times of multiunit spikes were estimated using a spike sorting algorithm \cite{niediek_reliable_2016} from the Combinato package ((\url{https://github.com/jniediek/combinato})(Figure~\ref{fig:01}C).

\subsection{Calculation of neural trajectories using GPFA}
To derive the neural trajectories (referred to as factors; Figure~\ref{fig:01}D) within the hippocampus, entorhinal cortex (EC), and amygdala, we used GPFA \cite{yu_gaussian-process_2009} on multiunit activity data for each session (Figure~\ref{fig:01}D). We implemented GPFA using the elephant package ((\url{https://elephant.readthedocs.io/en/latest/reference/gpfa.html}). We configured the bin size as 50 ms, with no overlaps. Each factor was z-normalized across all sessions. We calculated the Euclidean distance from the origin ($O$) using these trajectories (Figure~\ref{fig:01}E).
\\
\indent
Within each trajectory for a region such as AHL, we calculated the \textit{geometric medians} (i.e., $\mathrm{g_{F}}$ for fixation, $\mathrm{g_{E}}$ for encoding, $\mathrm{g_{M}}$ for maintenance, and $\mathrm{g_{R}}$ for retrieval phase) by establishing the median coordinates of the trajectory during the four phases (Figure~\ref{fig:01}D). We defined the optimal dimensionality for GPFA as three, as determined via the elbow method using log-likelihood values in a three-fold cross-validation approach (Figure~\ref{fig:02}B).

\subsection{Defining SWR candidates from hippocampal regions}
To pinpoint potential SWR events in the hippocampus, we used a detection method consistent with the consensus in the field \cite{liu_consensus_2022}. We re-referenced local field potential (LFP) signals from a region of interest (ROI), such as AHL, by subtracting the average signal outside the ROI (e.g., AHR, PHL, PHR, ECL, ECR, AL, AR) (see Figure~\ref{fig:01}A). Using these re-referenced LFP signals, we applied a ripple-band filter (80--140 Hz) to isolate SWR-positive (SWR$^+$) candidates (see Figure~\ref{fig:01}B). We performed SWR detection using a publicly available tool ((\url{https://github.com/Eden-Kramer-Lab/ripple_detection}) \cite{kay_hippocampal_2016}, with some modifications such as a revised bandpass range of 80--140 Hz for human applications \cite{norman_hippocampal_2019,norman_hippocampal_2021} as opposed to the original 150--250 Hz range used primarily for rodents.
\\
\indent
For SWR$^+$ candidates, we defined SWR-negative (SWR$^-$) candidates as control events by shuffling the timestamps of SWR$^+$ candidates across all trials and subjects. We visually inspected the defined SWR$^+$/SWR$^-$ candidates (see Figure~\ref{fig:01}).

\subsection{Defining SWRs from putative hippocampal CA1 regions}
We narrowed down SWR candidates within putative CA1 regions to define SWRs. We first identified putative CA1 regions as follows. We embedded SWR$^+$/SWR$^-$ candidates within the hippocampus into a two-dimensional space based on their superimposed spike counts per unit using UMAP (uniform manifold approximation and projection) \cite{mcinnes_umap_2018} in a supervised manner (Figure~\ref{fig:04}A). The silhouette score \cite{rousseeuw_silhouettes_1987}, a validation metric for clustering, was calculated from clustered samples (Table~\ref{tab:02}). We defined hippocampal regions with an average silhouette score across sessions greater than the 75th percentile as putative CA1 regions, resulting in the identification of five electrode locations from five patients (Table~\ref{tab:03}).
\\
\indent
We defined SWR$^+$/SWR$^-$ candidates within putative CA1 regions as SWRs, meaning they were no longer candidates. The duration and ripple band peak amplitude of SWRs followed log-normal distributions (Figure~\ref{fig:04}C \& E). We visually inspected SWR$^+$/SWR$^-$ as shown in Figure~\ref{fig:01}. We divided each SWR period into pre-SWR (at $-800$ to $-300$ ms from SWR center), mid-SWR (at $-250$ to $+250$ ms), and post-SWR (at $+300$ to $+800$ ms) based on the time from the SWR's center.

\subsection{Statistical Evaluation}
We conducted the Brunner--Munzel and Kruskal-Wallis tests using the scipy package in Python \cite{virtanen_scipy_2020}. We performed a correlation analysis by determining the rank of the observed correlation coefficient within the set-size-shuffled surrogate dataset, using a custom Python script. Additionally, we executed a bootstrap test using a homemade Python script.

\label{sec:methods}
```