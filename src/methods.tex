\section{Methods}
\subsection{Dataset}
A publicly available dataset \cite{boran_dataset_2020} was used. This dataset consists of nine epilepsy patients performing a modified Sternberg task, a task incorporating four phases: fixation (1s), encoding (2s), maintenance (3s), and retrieval (2s) \cite{boran_dataset_2020}. Throughout the encoding phase, participants were introduced to four, six, or eight alphabet letters, known as the set size. These participants were then tasked to ascertain whether a probe letter, presented during the retrieval phase was previously shown (the correct choice for the Match IN task) or not (the correct choice for the Mismatch OUT task). iEEG signals were recorded at a sampling rate of 32 kHz, within a frequency range of 0.5--5,000 Hz, using depth electrodes implanted in various medial temporal lobe (MTL) regions; these include the anterior head of the left and right hippocampus (AHL and AHR), the posterior body of the hippocampus (PHL and PHR), the entorhinal cortex (ECL and ECR), and the amygdala (AL and AR), as depicted in Figure~\ref{fig:01}A and Table~\ref{tab:01}. Subsequently, these iEEG signals were downsampled to a rate of 2 kHz. Correlations among variables, such as set size and correct rate, were investigated (Figure~\ref{fig:s01}S1). The timings of multiunit spikes were determined using a spike sorting algorithm \cite{niediek_reliable_2016} via the Combinato package (\url{https://github.com/jniediek/combinato}) (Figure~\ref{fig:01}C).

\subsection{Calculation of neural trajectories using GPFA}
Neural trajectories, a.k.a. 'factors' (Figure~\ref{fig:01}D), in areas including the hippocampus, EC, and amygdala (Figure~\ref{fig:01}D), were calculated using GPFA \cite{yu_gaussian-process_2009} applied to the multiunit activity data for each session. GPFA was executed with the elephant package (\url{https://elephant.readthedocs.io/en/latest/reference/gpfa.html}). The bin size was set to 50 ms, with no overlaps. Each factor was z-normalized across all sessions. The Euclidean distance from the origin ($O$) was subsequently calculated (Figure~\ref{fig:01}E).
\\
\indent
For each trajectory within a region, e.g., AHL, \textit{geometric medians} ($\mathrm{g_{F}}$ for fixation, $\mathrm{g_{E}}$ for encoding, $\mathrm{g_{M}}$ for maintenance, and $\mathrm{g_{R}}$ for retrieval phase) were ascertained by calculating the median coordinates of the trajectory during the four phases (Figure~\ref{fig:01}D). An optimal dimensionality for GPFA was identified as three using the elbow method, derived by investigating the log-likelihood values through a three-fold cross-validation approach (Figure~\ref{fig:02}B).

\subsection{Identifying SWR candidates from hippocampal regions}
Potential SWR events within the hippocampus were detected using a widely adopted method \cite{liu_consensus_2022}. LFP signals from a specific region of interest (ROI), such as AHL, were re-referenced by subtracting an averaged signal from locations outside the ROI (\textit{e.g.}, AHR, PHL, PHR, ECL, ECR, AL, and AR) (see Figure~\ref{fig:01}A). The re-referenced LFP signals then underwent filtering with a ripple-band filter (80--140 Hz) to identify possible SWR candidates, dubbed $\textrm{SWR}^+$ candidates (see Figure~\ref{fig:01}B). SWR detection utilized a published tool (\url{https://github.com/Eden-Kramer-Lab/ripple_detection}) \cite{kay_hippocampal_2016}, with the bandpass range adjusted to 80--140 Hz for humans \cite{norman_hippocampal_2019} \cite{norman_hippocampal_2021}, deviating from the original 150--250 Hz range typically applied to rodents.
\\
\indent
Control events for $\textrm{SWR}^+$ candidates, labeled as $\textrm{SWR}^-$ candidates, were identified by randomly shuffling the timestamps of $\textrm{SWR}^+$ candidates across all trials and subjects. The resulting $\textrm{SWR}^+/\textrm{SWR}^-$ candidates were then subjected to visual inspection (Figure~\ref{fig:01}).

\subsection{Defining SWRs from putative hippocampal CA1 regions}
SWRs were distinguished from SWR candidates in presumptive CA1 regions. The regions were initially defined in the following manner: $\textrm{SWR}^+/\textrm{SWR}^-$ candidates in the hippocampus were projected into a two-dimensional space based on overlapping spike counts per unit using a supervised method called UMAP (Uniform Manifold Approximation and Projection) \cite{mcinnes_umap_2018} (Figure~\ref{fig:04}A). Clustering validation was performed by computing the silhouette score \cite{rousseeuw_silhouettes_1987} from clustered samples (Table~\ref{tab:02}). Regions in the hippocampus, which scored above 0.6 on average across sessions (75th percentile) (Figure~\ref{fig:04}B), were considered to be presumed CA1 regions, thereby identifying five electrode positions among five patients (Table~\ref{tab:03}).
\\
\indent
$\textrm{SWR}^+/\textrm{SWR}^-$ candidates in these assumed CA1 regions were reclassified as $\textrm{SWR}^+/\textrm{SWR}^-$, hence losing their candidate status. The duration and ripple band peak amplitude of SWRs were observed to conform to log-normal distributions (Figure~\ref{fig:04}4C \& E). Each time period of SWR was partitioned in relation to the time from the SWR center into pre- (at $-800$ to $-300$ ms from SWR center), mid- (at $-250$ to $+250$ ms), and post-SWR (at $+300$ to $+800$ ms) times.

\subsection{Statistical evaluation}
Statistical analyses were performed using the Brunner--Munzel and Kruskal-Wallis tests courtesy of the SciPy package in Python \cite{virtanen_scipy_2020}. The correlational analysis determined the rank of the observed correlation coefficient in its associated set-size-shuffled surrogate via a custom Python script. The bootstrap test, on the other hand, was implemented using an in-house Python script.

\label{sec:methods}