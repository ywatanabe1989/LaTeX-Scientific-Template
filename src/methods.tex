\section{Methods}
\subsection{Dataset}
The data for this study were obtained from a publicly available dataset involving nine patients with epilepsy performing a modified Sternberg task \cite{boran_dataset_2020}. This task encompasses four stages: fixation (1s), encoding (2s), maintenance (3s), and retrieval (2s). During the encoding phase, a set of four, six, or eight letters were presented to the participants. Subsequently, they were tasked with assessing whether a probe letter displayed during the retrieval phase was previously shown (indicating a correct response for the Match IN task) or not (indicating a correct response for the Mismatch OUT task). Intracranial electroencephalography (iEEG) signals were recorded at a 32 kHz sampling rate within a 0.5--5,000 Hz frequency range, with depth electrodes placed in medial temporal lobe (MTL) regions including the anterior head and posterior body of the left and right hippocampus (AHL, AHR, PHL, and PHR), the entorhinal cortex (ECL and ECR), and the amygdala (AL and AR), as shown in Figure~\ref{fig:01}A and Table~\ref{tab:01}. The iEEG signals were then downsampled to 2 kHz. Correlations between variables such as set size and accuracy rate were explored (Figure~\ref{fig:s01}S1). Multiunit spike timings were determined via a spike sorting algorithm \cite{niediek_reliable_2016} using the Combinato package (\url{https://github.com/jniediek/combinato})(Figure~\ref{fig:01}C).

\subsection{Calculation of NT using GPFA}
Neural trajectories (NTs), also referred to as 'factors', in the hippocampus, EC, and amygdala were identified using Gaussian-Process Factor Analysis (GPFA) \cite{yu_gaussian-process_2009} applied to multi-unit activity data for each session, using the Elephant package (\url{https://elephant.readthedocs.io/en/latest/reference/gpfa.html}). The bin size was set to 50 ms with no overlaps. Each factor was Z-normalized across all sessions, and the Euclidean distance from the origin (O) was computed.
\\
\indent
For each NT in respective regions like AHL, geometric medians ($\mathrm{g_{F}}$ for fixation, $\mathrm{g_{E}}$ for encoding, $\mathrm{g_{M}}$ for maintenance, and $\mathrm{g_{R}}$ for retrieval phase) were calculated by determining the NT's median coordinates during the four phases. An optimal GPFA dimensionality was identified as three using the elbow method drawn from investigating the log-likelihood values via a three-fold cross-validation approach (Figure~\ref{fig:02}B).

\subsection{Identifying SWR candidates from hippocampal regions}
Potential sharp wave-ripple (SWR) events within the hippocampus were identified using a widely accepted method \cite{liu_consensus_2022}. Local field potential (LFP) signals from a region of interest (ROI), such as AHL, were re-referenced by deducting the averaged signal from locations outside the ROI (including AHR, PHL, PHR, ECL, ECR, AL, and AR). The re-referenced LFP signals were then filtered using a ripple-band filter (80--140 Hz) to identify SWR candidates, denoted as $\textrm{SWR}^+$ candidates. SWR detection used a published tool (\url{https://github.com/Eden-Kramer-Lab/ripple_detection}) \cite{kay_hippocampal_2016}, with the bandpass range adjusted to 80--140 Hz for humans \cite{norman_hippocampal_2019, norman_hippocampal_2021}, unlike the initial 150--250 Hz range typically applied to rodents.
\\
\indent
Control events for $\textrm{SWR}^+$ candidates, labeled as $\textrm{SWR}^-$ candidates, were discovered by randomly shuffling the timestamps of $\textrm{SWR}^+$ candidates across all trials and subjects. The resulting $\textrm{SWR}^+/\textrm{SWR}^-$ candidates were then visually inspected.

\subsection{Defining SWRs from potential hippocampal CA1 regions}
Candidates for $\textrm{SWR}^+/\textrm{SWR}^-$ events in the hippocampus' putative CA1 regions were distinguished. Pyramidal neurons in these regions, referred to as cornu ammonis 1 (CA1), were initially defined as follows: $\textrm{SWR}^+/\textrm{SWR}^-$ candidates were projected into a two-dimensional space based on overlapping spike counts per unit using a supervised method, UMAP (Uniform Manifold Approximation and Projection) \cite{mcinnes_umap_2018}. Clustering validation was performed by calculating the silhouette score \cite{rousseeuw_silhouettes_1987} from clustered samples. Regions in the hippocampus, with scores above 0.6 on average across sessions (75th percentile), were classified as putative CA1 regions, resulting in the identification of five electrode positions from five patients.
\\
\indent
$\textrm{SWR}^+/\textrm{SWR}^-$ candidates in these predefined CA1 regions were marked as $\textrm{SWR}^+/\textrm{SWR}^-$, consequently losing their candidate status. The duration and ripple band peak amplitude of SWRs were found to follow log-normal distributions. Each SWR timeframe was partitioned relative to the time from the SWR center into pre- (at $-800$ ms to $-300$ ms from the SWR center), mid- (at $-250$ to $+250$ ms), and post-SWR (at $+300$ to $+800$ ms) periods.

\subsection{Statistical Evaluation}
Both the Brunner--Munzel test and the Kruskal-Wallis test were performed using the SciPy package in Python \cite{virtanen_scipy_2020}. Correlational analysis was conducted by determining the rank of the observed correlation coefficient within its associated set-size-shuffled surrogate using a custom Python script. The bootstrap test was implemented with an in-house Python script.
\label{sec:methods}