```tex
%%%%%%%%%%%%%%%%%%%%%%%%%%%%%%%%%%%%%%%%%%%%%%%%%%%%%%%%%%%%%%%%%%%%%%%%%%%%%%%%
%% Methods
%%%%%%%%%%%%%%%%%%%%%%%%%%%%%%%%%%%%%%%%%%%%%%%%%%%%%%%%%%%%%%%%%%%%%%%%%%%%%%%%
\section{Methods}
\subsection{Dataset}
We used a publicly available dataset \cite{boran_dataset_2020}. It included nine epilepsy patients performing a modified Sternberg task encompassing the four following phases: fixation (1 s), encoding (2 s), maintenance (3 s), and retrieval (2 s) \cite{boran_dataset_2020}. During the encoding phase, participants viewed sets containing four, six, or eight letters, denoted herein as the set size. During the retrieval phase, participants determined if a probe letter was present in the initial set (the correct response for the Match IN task) or not (the correct response for the Mismatch OUT task). Intracranial EEG (iEEG) signals were collected using depth electrodes placed within the medial temporal lobe (MTL) regions: left and right hippocampal head (AHL and AHR), body (PHL and PHR), entorhinal cortex (ECL and ECR), and amygdala (AL and AR). These signals were recorded at a sampling rate of 32 kHz and within a frequency range of 0.5--5,000 Hz (Figure~\ref{fig:01}A and Table~\ref{tab:01}). The iEEG signals were then resampled at 2 kHz. Correlations between experimental variables such as set size and accuracy rate were established (Figure~\ref{fig:s01}S1). Multiunit spike times were estimated using the spike sorting algorithm from the Combinato package \cite{niediek_reliable_2016} (\url{https://github.com/jniediek/combinato})(Figure~\ref{fig:01}C).

\subsection{Calculation of neural trajectories using GPFA}
We utilized GPFA \cite{yu_gaussian-process_2009} on the multiunit activity data for each session to extract neural trajectories (referred to as factors; Figure~\ref{fig:01}D) within the hippocampus, entorhinal cortex (EC), and amygdala. This was executed using the elephant package (\url{https://elephant.readthedocs.io/en/latest/reference/gpfa.html}). The bin size was 50 ms, without overlaps. Each factor was z-normalized across all sessions. The Euclidean distance from the origin ($O$) was derived from these trajectories (Figure~\ref{fig:01}E).
\\
\indent
Within each trajectory for a particular region like AHL, the \textit{geometric medians} (i.e., $\mathrm{g_{F}}$ for fixation, $\mathrm{g_{E}}$ for encoding, $\mathrm{g_{M}}$ for maintenance, and $\mathrm{g_{R}}$ for retrieval phase) were derived by establishing the median coordinates of the trajectories during the four phases (Figure~\ref{fig:01}D). Three was determined to be the optimal dimensionality for GPFA, as defined by the elbow method using log-likelihood values in a three-fold cross-validation approach (Figure~\ref{fig:02}B).

\subsection{Defining SWR candidates from hippocampal regions}
To identify potential SWR events in the hippocampus, a widely accepted detection method was used \cite{liu_consensus_2022}. The regional local field potential (LFP) signals, like those from AHL, were re-referenced by subtracting the mean signal outside the region of interest (e.g., AHR, PHL, PHR, ECL, ECR, AL, AR) (see Figure~\ref{fig:01}A). This re-referenced LFP signals were used with a ripple-band filter (80--140 Hz) to discern the SWR-positive (SWR$^+$) candidates (see Figure~\ref{fig:01}B). SWR detection was performed using a public tool (\url{https://github.com/Eden-Kramer-Lab/ripple_detection}) \cite{kay_hippocampal_2016}, with modifications such as a revised bandpass range of 80--140 Hz for human applications \cite{norman_hippocampal_2019,norman_hippocampal_2021} over the original 150--250 Hz range commonly used for rodents.
\\
\indent
For SWR$^+$, the control events were defined as SWR-negative (SWR$^-$) by shuffling the timestamps of SWR$^+$ across all trials and subjects. These SWR$^+$ and SWR$^-$  were visually inspected (see Figure~\ref{fig:01}).

\subsection{Defining SWRs from putative hippocampal CA1 regions}
SWR candidates were defined within the putative CA1 regions for SWRs. The potential CA1 regions were identified as follows: SWR$^+$/SWR$^-$ in the hippocampus were embedded into a two-dimensional space using UMAP based on superimposed spike counts per unit in a supervised manner \cite{mcinnes_umap_2018}(Figure~\ref{fig:04}A). The silhouette score \cite{rousseeuw_silhouettes_1987}, computed from clustered samples (Table~\ref{tab:02}), was used for validating clustering effectiveness. Regions with an average silhouette score across sessions surpassing the 75th percentile were labeled as putative CA1 territories, leading to the identification of five electrode locations in five patients (Table~\ref{tab:03}).
\\
\indent
Then, SWR$^+$/SWR$^-$ within putative CA1 sectors were defined as SWRs, no longer considered as candidates. SWR duration and ripple band peak amplitude exhibited a log-normal distribution (Figure~\ref{fig:04}C \& E). As shown in Figure~\ref{fig:01}, SWR$^+$/SWR$^-$ underwent visual scrutiny. Each SWR period was classified into pre-SWR (from $-800$ to $-300$ ms from SWR center), mid-SWR (from $-250$ to $+250$ ms), and post-SWR (from $+300$ to $+800$ ms), referenced to the time from the SWR's center.

\subsection{Statistical Evaluation}
The Brunner--Munzel and Kruskal-Wallis tests were conducted using the scipy package in Python \cite{virtanen_scipy_2020}. We determined the rank of the observed correlation coefficient in the set-size-shuffled surrogate dataset using a custom Python code for a correlation analysis. Additionally, we executed a bootstrap test with a custom Python script.

\label{sec:methods}
```
