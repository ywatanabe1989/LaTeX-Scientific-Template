\section{Introduction}
Working memory (WM) significantly influences daily life, and its neural bases continue to be intensively researched. A primary focus of this research is the hippocampus, a structure integral to memory functions \cite{scoville_loss_1957} \cite{squire_legacy_2009}  \cite{boran_persistent_2019} \cite{kaminski_persistently_2017} \cite{kornblith_persistent_2017} \cite{faraut_dataset_2018} \cite{borders_hippocampus_2022} \cite{li_functional_2023} \cite{dimakopoulos_information_2022}. Deepening our understanding of the hippocampus's role in working memory is critical not only for knowledge advancement but also potentially for cognitive abilities enhancement.
\\
\indent
Current evidence suggests a transient, synchronized oscillation known as sharp-wave ripple (SWR) \cite{buzsaki_hippocampal_2015} is associated with several cognitive functions. These comprise memory replay \cite{wilson_reactivation_1994} \cite{nadasdy_replay_1999} \cite{lee_memory_2002} \cite{diba_forward_2007} \cite{davidson_hippocampal_2009}, memory consolidation \cite{girardeau_selective_2009} \cite{ego-stengel_disruption_2010} \cite{fernandez-ruiz_long-duration_2019} \cite{kim_corticalhippocampal_2022}, memory recall \cite{wu_hippocampal_2017} \cite{norman_hippocampal_2019} \cite{norman_hippocampal_2021}, and neural plasticity \cite{behrens_induction_2005} \cite{norimoto_hippocampal_2018}. These associations posit that SWR could be a core computational feature of hippocampal processing, contributing to working memory performance. However, research on SWR's effects on working memory is relatively sparse \cite{jadhav_awake_2012}, being mainly restricted to rodent models engaged in navigation tasks with indefinable memory acquisition and recall timing.
\\
\indent
Recent research proposes that low-dimensional representations in hippocampal neurons can elucidate WM task performance. Specifically, the firing patterns of place cells \cite{okeefe_hippocampus_1971} \cite{okeefe_place_1976} \cite{ekstrom_cellular_2003} \cite{kjelstrup_finite_2008} \cite{harvey_intracellular_2009} found in the hippocampus, display within a dynamic, nonlinear three-dimensional hyperbolic space in rats \cite{zhang_hippocampal_2022}. Additionally, grid cells in the entorhinal cortex (EC)—a primary route to the hippocampus \cite{naber_reciprocal_2001} \cite{van_strien_anatomy_2009} \cite{strange_functional_2014}—showed a toroidal geometry during exploration in rats \cite{gardner_toroidal_2022}. These studies, however, are limited by their emphasis on spatial navigation tasks in rodents, which affect the temporal resolution of WM tasks. For instance, the timing of information acquisition by an animal is unclear in these contexts. Therefore, the generalizability of these findings to humans and tasks beyond navigation still requires verification.
\\
\indent
Considering these factors, this study explores the hypothesis that hippocampal neurons exhibit unique 'neural trajectories' in low-dimensional spaces, particularly during SWR episodes, when responding to WM tasks. We tested this hypothesis using a high-temporal-resolution dataset of patients performing an eight-second Sternberg task (1 s for fixation, 2 s for encoding, 3 s for maintenance, and 2 s for retrieval). The patients' medial temporal lobe (MTL) intracranial electroencephalography (iEEG) signals were recorded \cite{boran_dataset_2020}. To analyze low-dimensional neural trajectories, we employed Gaussian-process factor analysis (GPFA), a recognized method for examining neural population dynamics \cite{yu_gaussian-process_2009}.
\label{sec:introduction}