Working memory (WM) plays a crucial role in everyday life, and its neural underpinnings remain an area of ongoing research. The hippocampus, notably integral to memory, continues to be a primary focus of this investigation \cite{scoville_loss_1957} \cite{squire_legacy_2009}  \cite{boran_persistent_2019} \cite{kaminski_persistently_2017} \cite{kornblith_persistent_2017} \cite{faraut_dataset_2018} \cite{borders_hippocampus_2022} \cite{li_functional_2023} \cite{dimakopoulos_information_2022}. Gaining insights into the role of the hippocampus in working memory is vital to deepening our understanding of cognitive processes, hence fostering the progression of cognitive training and interventions.
\\
\indent
Current evidence suggests a transient, synchronized oscillation, referred to as sharp-wave ripple (SWR) \cite{buzsaki_hippocampal_2015}, is linked with several cognitive functions, such as memory replay \cite{wilson_reactivation_1994} \cite{nadasdy_replay_1999} \cite{lee_memory_2002} \cite{diba_forward_2007} \cite{davidson_hippocampal_2009}, memory consolidation \cite{girardeau_selective_2009} \cite{ego-stengel_disruption_2010} \cite{fernandez-ruiz_long-duration_2019} \cite{kim_corticalhippocampal_2022}, memory recall \cite{wu_hippocampal_2017} \cite{norman_hippocampal_2019} \cite{norman_hippocampal_2021}, and neural plasticity \cite{behrens_induction_2005} \cite{norimoto_hippocampal_2018}. This evidence indicates the likelihood that SWR could be a critical component of hippocampal processing, contributing to working memory performance. However, research investigating the effects of SWRs on working memory remains sparse \cite{jadhav_awake_2012}, and is largely limited to rodent models participating in navigation tasks where the timing of memory acquisition and recall is not explicitly distinguished.
\\
\indent
Recent studies indicate that hippocampal neurons exhibit low-dimensional representations during WM tasks. Notably, the firing patterns of place cells \cite{okeefe_hippocampus_1971} \cite{okeefe_place_1976} \cite{ekstrom_cellular_2003} \cite{kjelstrup_finite_2008} \cite{harvey_intracellular_2009}, located in the hippocampus, are observed to be encompassed within a dynamic, nonlinear three-dimensional hyperbolic geometry in rodents \cite{zhang_hippocampal_2022}. Moreover, grid cells in the entorhinal cortex (EC)—the dominant pathway to the hippocampus \cite{naber_reciprocal_2001} \cite{van_strien_anatomy_2009} \cite{strange_functional_2014}—displayed toroidal topology during exploration \cite{gardner_toroidal_2022}. Unfortunately, these investigations are confined to spatial navigation tasks in rodents, thus imposing limitations on the temporal resolution of WM tasks. The applicability of these findings to human subjects and their generalization beyond navigation tasks remains to be established.
\\
\indent
Given these considerations, the current study aims to validate the hypothesis that hippocampal neurons exhibit distinctive representations in low-dimensional spaces, designated as 'neural trajectory,' during WM tasks, most prominently within SWR periods. To evaluate this claim, we employed a dataset of patients performing an eight-second Sternberg task with high temporal resolution (1 s for fixation, 2 s for encoding, 3 s for maintenance, and 2 s for retrieval), while their intracranial electroencephalography signals (iEEG) within the medial temporal lobe (MTL) were being monitored \cite{boran_dataset_2020}. To investigate low-dimensional neural trajectories, we employed Gaussian-process factor analysis (GPFA), a method renowned for analyzing neural population dynamics \cite{yu_gaussian-process_2009}.
\label{sec:introduction}