\section{Introduction}
Working memory (WM) plays a critical role in our daily activities, but the neural mechanisms underlying it are not completely understood. Particularly, the hippocampus, a key brain region involved in memory, warrants ongoing investigation \cite{scoville_loss_1957,squire_legacy_2009,boran_persistent_2019,kaminski_persistently_2017,kornblith_persistent_2017,faraut_dataset_2018,borders_hippocampus_2022,li_functional_2023,dimakopoulos_information_2022}. Enhancing our understanding of the hippocampus's role in working memory can lead to deeper insights into cognitive processes, thereby promoting the development of cognitive training strategies and interventions. 

\indent
Sharp wave ripples (SWR), transient and synchronous oscillations generated by the hippocampus, are associated with key cognitive functions, including memory replay \cite{wilson_reactivation_1994,nadasdy_replay_1999,lee_memory_2002,davidson_hippocampal_2009}, memory consolidation \cite{girardeau_selective_2009,ego-stengel_disruption_2010,fernandez-ruiz_long-duration_2019,kim_corticalhippocampal_2022}, memory recall \cite{wu_hippocampal_2017,norman_hippocampal_2019,norman_hippocampal_2021}, and neural plasticity \cite{behrens_induction_2005,norimoto_hippocampal_2018}. Consequently, SWRs could be critical to hippocampal processing and influence working memory performance. However, studies examining the influence of SWRs on working memory are scarce \cite{jadhav_awake_2012} and primarily center on rodent models utilizing navigation tasks. Such research has not clearly differentiated between the specific timing of memory recall and acquisition.

\indent
Moreover, it has been found that hippocampal neurons present low-dimensional representations during WM tasks. The firing patterns of place cells \cite{okeefe_hippocampus_1971,okeefe_place_1976,ekstrom_cellular_2003,kjelstrup_finite_2008,harvey_intracellular_2009,royer_control_2012} in the hippocampus, for instance, align with a dynamic, nonlinear three-dimensional hyperbolic geometry in rodents \cite{zhang_hippocampal_2022}. Similarly, grid cells in the entorhinal cortex (EC)—the main entry point to the hippocampus \cite{naber_reciprocal_2001,van_strien_anatomy_2009,strange_functional_2014}—display a toroidal topology during exploration \cite{gardner_toroidal_2022}. However, these studies primarily relate to spatial navigation tasks in rodents and provide limited temporal resolution for WM tasks. Additionally, it is still uncertain whether these results apply to humans or to tasks beyond navigation.

\indent
In light of these considerations, this study tests the hypothesis that hippocampal neurons demonstrate distinct low-dimensional representations, or 'neural trajectories', during WM tasks, specifically during SWR periods. To investigate this, we used a dataset of patients performing an eight-second Sternberg task (with high temporal resolution: 1 s for fixation, 2 s for encoding, 3 s for maintenance, and 2 s for retrieval) while their intracranial electroencephalography signals (iEEG) in the medial temporal lobe (MTL) were recorded \cite{boran_dataset_2020}. We implemented Gaussian-process factor analysis (GPFA) on multichannel unit activity to explore low-dimensional neural trajectories, a proven method to analyze neural population dynamics \cite{yu_gaussian-process_2009}.
\label{sec:introduction}