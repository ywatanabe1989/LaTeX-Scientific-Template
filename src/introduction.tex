\section{Introduction}
Working memory (WM) significantly influences everyday life, and the neural bases of this cognitive process continue to be the subject of intensive research. One key focus of this research is the hippocampus, a structure integral to memory functions \cite{scoville_loss_1957} \cite{squire_legacy_2009}  \cite{boran_persistent_2019} \cite{kaminski_persistently_2017} \cite{kornblith_persistent_2017} \cite{faraut_dataset_2018} \cite{borders_hippocampus_2022} \cite{li_functional_2023} \cite{dimakopoulos_information_2022}. A thorough understanding of the hippocampus's role in working memory is essential for advancing our knowledge of cognitive processes [CHECKME>]and fostering advancements in cognitive training and interventions[<CHECKME ENDS].
\\
\indent
Current evidence suggests that a transient, synchronized oscillation, called sharp-wave ripple (SWR) \cite{buzsaki_hippocampal_2015}, is associated with several cognitive functions. These include memory replay \cite{wilson_reactivation_1994} \cite{nadasdy_replay_1999} \cite{lee_memory_2002} \cite{diba_forward_2007} \cite{davidson_hippocampal_2009}, memory consolidation \cite{girardeau_selective_2009} \cite{ego-stengel_disruption_2010} \cite{fernandez-ruiz_long-duration_2019} \cite{kim_corticalhippocampal_2022}, memory recall \cite{wu_hippocampal_2017} \cite{norman_hippocampal_2019} \cite{norman_hippocampal_2021}, and neural plasticity \cite{behrens_induction_2005} \cite{norimoto_hippocampal_2018}. This association suggests that SWR may be a crucial part of hippocampal processing that contributes to working memory performance. However, research on the effects of SWRs on working memory is relatively scarce \cite{jadhav_awake_2012}, and is predominantly limited to rodent models engaged in navigation tasks, where the timing of memory acquisition and recall is not clearly defined.
\\
\indent
Recent studies have found low-dimensional representations in the hippocampal neurons during WM tasks. Specifically, the firing patterns of place cells \cite{okeefe_hippocampus_1971} \cite{okeefe_place_1976} \cite{ekstrom_cellular_2003} \cite{kjelstrup_finite_2008} \cite{harvey_intracellular_2009}, found in the hippocampus, have been identified within a dynamic, nonlinear three-dimensional hyperbolic space in rats \cite{zhang_hippocampal_2022}. Additionally, grid cells in the entorhinal cortex (EC), which is the main pathway to the hippocampus \cite{naber_reciprocal_2001} \cite{van_strien_anatomy_2009} \cite{strange_functional_2014}, exhibited a toroidal geometry during exploration in rats \cite{gardner_toroidal_2022}. However, these studies are limited by their focus on spatial navigation tasks in rodents, affecting the temporal resolution of WM tasks [CHECKME>]such as the timing of information acquisition and recall[<CHECKME ENDS]. The applicability of these findings to human subjects and beyond navigation tasks still needs to be confirmed.
\\
\indent
Considering these aspects, the present study aims to investigate the hypothesis that hippocampal neurons display distinctive patterns of activity or 'neural trajectories' in low-dimensional spaces, particularly during SWR periods, in response to WM tasks. To test this hypothesis, we used a dataset of patients performing an eight-second Sternberg task (1 s for fixation, 2 s for encoding, 3 s for maintenance, and 2 s for retrieval) with high temporal resolution. Intracranial electroencephalography (iEEG) signals within the medial temporal lobe (MTL) were recorded for these patients \cite{boran_dataset_2020}. To examine low-dimensional neural trajectories, we utilized Gaussian-process factor analysis (GPFA), an established method for analyzing neural population dynamics \cite{yu_gaussian-process_2009}.
\label{sec:introduction}