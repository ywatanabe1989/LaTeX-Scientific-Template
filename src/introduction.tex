\section{Introduction}
Working memory (WM) is essential for various daily tasks, yet we still lack full comprehension of the associated neural mechanisms. In particular, the hippocampus, a crucial region for memory in the brain, demands continued investigation \cite{scoville_loss_1957,squire_legacy_2009,boran_persistent_2019,kaminski_persistently_2017,kornblith_persistent_2017,faraut_dataset_2018,borders_hippocampus_2022,li_functional_2023,dimakopoulos_information_2022}. Improving our understanding of the hippocampus's role in working memory could provide enhanced insights into cognitive processes and stimulate the advancement of cognitive training strategies and interventions. 

\indent
Hippocampally-generated sharp wave ripples (SWR), which are transient and synchronous oscillations, are linked to fundamental cognitive functions such as memory replay \cite{wilson_reactivation_1994,nadasdy_replay_1999,lee_memory_2002,davidson_hippocampal_2009}, memory consolidation \cite{girardeau_selective_2009,ego-stengel_disruption_2010,fernandez-ruiz_long-duration_2019,kim_corticalhippocampal_2022}, memory recall \cite{wu_hippocampal_2017,norman_hippocampal_2019,norman_hippocampal_2021}, and neural plasticity \cite{behrens_induction_2005,norimoto_hippocampal_2018}. Therefore, SWRs may play a key role in hippocampal processing and potentially influence working memory performance. Nonetheless, research investigating the impact of SWRs on working memory is limited \cite{jadhav_awake_2012}, mainly focusing on rodent models performing navigation tasks without clearly differentiating between specific timings of memory recall and acquisition.

\indent
Furthermore, it has been observed that hippocampal neurons exhibit low-dimensional representations during WM tasks. Notably, the firing patterns of hippocampal place cells \cite{okeefe_hippocampus_1971,okeefe_place_1976,ekstrom_cellular_2003,kjelstrup_finite_2008,harvey_intracellular_2009,royer_control_2012} align to a dynamic, nonlinear three-dimensional hyperbolic geometry in rodents \cite{zhang_hippocampal_2022}. Also, grid cells in the entorhinal cortex (EC)—the primary gateway to the hippocampus \cite{naber_reciprocal_2001,van_strien_anatomy_2009,strange_functional_2014}—exhibit a toroidal topology during exploration \cite{gardner_toroidal_2022}. Unfortunately, these studies mostly pertain to spatial navigation tasks in rodents and offer limited temporal resolution for WM tasks. Furthermore, they leave unanswered whether these findings apply to humans or tasks other than navigation.

\indent
Considering the above, this study aims to test the hypothesis that hippocampal neurons present distinct low-dimensional representations, or 'neural trajectories', during WM tasks, particularly during SWR episodes. To interrogate this, we used a patient dataset performing an eight-second Sternberg task (offering high temporal resolution: 1 s for fixation, 2 s for encoding, 3 s for maintenance, and 2 s for retrieval) while recording their medial temporal lobe (MTL) intracranial electroencephalography signals (iEEG) \cite{boran_dataset_2020}. We applied Gaussian-process factor analysis (GPFA) to multichannel unit activity to examine low-dimensional neural trajectories, a well-validated method for analyzing neural population dynamics \cite{yu_gaussian-process_2009}.
\label{sec:introduction}
