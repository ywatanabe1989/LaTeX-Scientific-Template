\section{Introduction}
Working memory (WM) is of crucial importance in our daily activities, yet the neural mechanisms underlying it are not yet entirely understood. In particular, the role of the hippocampus, a critical brain region involved in memory, warrants continuing study \cite{scoville_loss_1957,squire_legacy_2009,boran_persistent_2019,kaminski_persistently_2017,kornblith_persistent_2017,faraut_dataset_2018,borders_hippocampus_2022,li_functional_2023,dimakopoulos_information_2022}. An improved understanding of the hippocampus's role in working memory will provide deeper insights into cognitive processes, thus facilitating the development of cognitive training strategies and interventions.
\\
\indent
The hippocampus generates transient and synchronous oscillations, called sharp wave ripples (SWR), which are associated with various cognitive functions, including memory replay \cite{wilson_reactivation_1994,nadasdy_replay_1999,lee_memory_2002,davidson_hippocampal_2009}, memory consolidation \cite{girardeau_selective_2009,ego-stengel_disruption_2010,fernandez-ruiz_long-duration_2019,kim_corticalhippocampal_2022}, memory recall \cite{wu_hippocampal_2017,norman_hippocampal_2019,norman_hippocampal_2021}, and neural plasticity \cite{behrens_induction_2005,norimoto_hippocampal_2018}. As a result, SWRs could be a key aspect of processing in the hippocampus and impact working memory performance. However, studies exploring the effect of SWRs on working memory are relatively few \cite{jadhav_awake_2012} and mainly focus on rodent models using navigation tasks where the specific timing of memory recall and acquisition is not clearly differentiated.
\\
\indent
In addition, it has been revealed that hippocampal neurons exhibit low-dimensional representations during WM tasks. For example, the firing patterns of place cells \cite{okeefe_hippocampus_1971,okeefe_place_1976,ekstrom_cellular_2003,kjelstrup_finite_2008,harvey_intracellular_2009,royer_control_2012} in the hippocampus conform to a dynamic, nonlinear three-dimensional hyperbolic geometry in rodents \cite{zhang_hippocampal_2022}. Also, grid cells in the entorhinal cortex (EC) — the main entry point to the hippocampus \cite{naber_reciprocal_2001,van_strien_anatomy_2009,strange_functional_2014}— show a toroidal topology during exploration \cite{gardner_toroidal_2022}. Yet again, these studies are limited to spatial navigation tasks in rodents and yield limited temporal resolution for WM tasks. Moreover, it remains an open question whether these results can be extended to humans or to tasks beyond navigation.
\\
\indent
Considering these points, this study evaluates the hypothesis that hippocampal neurons exhibit distinct representations in low-dimensional spaces, referred to as 'neural trajectories', during WM tasks, especially during SWR periods. To examine this hypothesis, we used a dataset of patients performing an eight-second Sternberg task (with high temporal resolution: 1 s for fixation, 2 s for encoding, 3 s for maintenance, and 2 s for retrieval) while their intracranial electroencephalography signals (iEEG) in the medial temporal lobe (MTL) were recorded \cite{boran_dataset_2020}. We adopted the Gaussian-process factor analysis (GPFA) on multichannel unit activity to probe low-dimensional neural trajectories, a well-established approach for analyzing neural population dynamics \cite{yu_gaussian-process_2009}.
\label{sec:introduction}