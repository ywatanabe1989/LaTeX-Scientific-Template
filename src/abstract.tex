\begin{abstract}
\pdfbookmark[1]{Abstract}{abstract}
Working memory (WM) is crucial for numerous cognitive functions, but the neural mechanisms that underpin its operation are not fully comprehended. The hippocampus and its associated sharp-wave ripples (SWRs) --- brief, synchronized oscillations observed within --- have been studied due to their role in memory consolidation and retrieval, albeit, the full extent of their involvement with WM tasks is yet to be thoroughly explored. This study proves that WM task introduces unique dynamics in the multi-unit activity patterns of the hippocampus, particularly during SWR periods. We analyzed intracranial electroencephalography data, collected from the medial temporal lobe (MTL) of nine epilepsy patients during an eight-second Sternberg task. Using a Gaussian-process factor analysis, we extracted low-dimensional neural representations or neural trajectories (NT) within MTL regions throughout the Sternberg WM task. The analysis showed significant variances in the hippocampal NT when compared with that of the entorhinal cortex and the amygdala. Additionally, the NT distance between the encoding and retrieval phases was found to be dependent on the memory load. Significantly, the hippocampal NT varied during the retrieval phase between encoding and retrieval states in a task-dependent manner. These variations transitioned from encoding to retrieval states during SWR episodes. These findings stress the importance of the hippocampus in WM tasks, and propose a new hypothes that the hippocampus transitions its functional state from encoding to retrieval during SWRs in WM tasks.
\end{abstract}