\begin{abstract}
\pdfbookmark[1]{Abstract}{abstract}
Working memory (WM) is crucial to various cognitive functions, yet its neural mechanisms remain largely elusive. Emerging interest focuses on the roles of the hippocampus and sharp wave-ripple complexes (SWRs) – transient, synchronous neural events in the hippocampus – in memory consolidation and retrieval, however, their relationship to WM tasks remains ambiguous. Recent studies propose that multiunit activity patterns in the hippocampus may operate concurrently with SWRs, exhibiting distinct dynamics during WM tasks. We performed an analysis of an electroencephalogram dataset from the medial temporal lobe (MTL) in nine patients with epilepsy during an eight-second Sternberg task. Low-dimensional neural representations, or 'trajectories', within the MTL were extracted using Gaussian-process factor analysis as the WM task was performed. The results show that significant differences in neural trajectories exist in the hippocampus compared to the entorhinal cortex and amygdala. Additionally, the divergence in trajectories between the encoding and retrieval phases appears to be memory load-dependent. Interestingly, hippocampal trajectories fluctuate during the retrieval phase, indicating task-dependent shifts between encoding and retrieval states occurring during both baseline and SWR events. These shifts from encoding to retrieval states are synchronized with the presence of SWRs, highlighting the critical function of the hippocampus in WM tasks. This finding suggests a novel hypothesis: the hippocampus modulates its functional state from encoding to retrieval in the event of SWRs.
\end{abstract}