\begin{abstract}
\pdfbookmark[1]{Abstract}{abstract}
Working memory (WM) is vital for many cognitive functions, although the neural mechanisms underlying its operation are not yet fully understood. Although the hippocampus and associated sharp-wave ripples (SWRs) --- brief, synchronized oscillations within it --- have been explored owing to their role in memory consolidation and retrieval, a comprehensive understanding of their involvement in WM tasks is pending further investigation. This study reveals that a WM task introduces unique dynamics into the multi-unit activity patterns of the hippocampus, primarily during SWR episodes. We analyzed intracranial electroencephalography data collected from the medial temporal lobe (MTL) of nine epilepsy patients performing an eight-second Sternberg task. Using Gaussian-process factor analysis, we extracted low-dimensional neural representations or neural trajectories (NT) within MTL regions throughout the Sternberg WM task. The analysis demonstrated substantial differences in hippocampal NT compared to that in the entorhinal cortex and the amygdala. Additionally, the NT distance between the encoding and retrieval phases was dependent on the memory load. Notably, hippocampal NT distinguished encoding and retrieval states during the retrieval phase in a task-dependent manner. These differences transitioned from encoding to retrieval states in the presence of SWR episodes. These results emphasize the importance of the hippocampus in WM tasks and propose a new hypothesis: the hippocampus transitions its functional state from encoding to retrieval during SWRs in WM tasks.
\end{abstract}