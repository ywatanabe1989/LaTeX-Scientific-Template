\begin{abstract}
\pdfbookmark[1]{Abstract}{abstract}
Working memory (WM) is integral to many cognitive functions, but its neural mechanisms are not fully understood. Interest has emerged regarding the roles of the hippocampus and sharp-wave ripple complexes (SWRs) - brief, synchronous neural events in the hippocampus - in memory consolidation and retrieval. Yet, the relationship between these elements and WM tasks remains unclear. Recent studies suggest that multiunit activity patterns in the hippocampus might function in unison with SWRs, displaying unique dynamics during WM tasks. We conducted an analysis of an electroencephalogram dataset from the medial temporal lobe (MTL) in nine patients with epilepsy during an eight-second Sternberg task. Low-dimensional neural representations or 'trajectories' within the MTL regions were extracted using Gaussian-process factor analysis while the WM task was performed. The findings reveal that significant differences in neural trajectories are seen in the hippocampus in contrast to the entorhinal cortex and amygdala. Moreover, the dissimilarity in trajectories between the encoding and retrieval phases seems dependent on the memory load. Intriguingly, hippocampal trajectories vary during the retrieval phase, suggesting task-dependent shifts between encoding and retrieval states that occur during both baseline and SWR instances. These shifts from encoding to retrieval states coincide with the presence of SWRs, underscoring the pivotal role of the hippocampus in WM tasks. This suggests a novel hypothesis: the hippocampus adjusts its functional state from encoding to retrieval during SWRs.
\end{abstract}