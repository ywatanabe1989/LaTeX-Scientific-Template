\begin{abstract}
\pdfbookmark[1]{Abstract}{abstract}
Working memory (WM) is fundamental to a plethora of cognitive functions, but the intricate neural mechanisms crucial to its operation are not fully comprehended. In particular, the roles of the hippocampus and sharp-wave ripple complexes (SWRs) --- rapid, synchronised neural events within the hippocampus --- are known to facilitate memory consolidation and retrieval, yet their contributions to WM tasks remain somewhat ambiguous. We propose that the coordinated activity patterns in the hippocampus work in unison with SWRs, displaying distinct dynamics during WM tasks. Our study involved an extensive analysis of a dataset obtained from intracranial electroencephalogram recordings from the medial temporal lobe (MTL) of nine epileptic patients during an eight-second Sternberg task. We utilised Gaussian-process factor analysis to identify low-dimensional neural representations, or 'trajectories,' within the MTL areas during the WM task. We found that the neural trajectory displayed the most substantial variations in the hippocampus compared to the entorhinal cortex and amygdala. Moreover, we observed that the deviation in trajectories between encoding and retrieval phases was dependent on memory load. Interestingly, hippocampal trajectories oscillated during the retrieval phase, revealing task-dependent shifts between encoding and retrieval states, which encompassed baseline and SWR phases. These oscillations transitioned from encoding to retrieval states consistent with the occurrence of SWRs. These findings underline the significant role of the hippocampus during the performance of WM tasks and put forward a compelling hypothesis for further exploration: the hippocampus undergoes a functional transition from encoding to retrieval during SWRs.
\end{abstract}