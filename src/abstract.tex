\begin{abstract}
\pdfbookmark[1]{Abstract}{abstract}
Working memory (WM) is vital for numerous cognitive functions, yet the complexity of the neural mechanisms involved is not yet fully understood. Intriguingly, the hippocampus and sharp-wave ripple complexes (SWRs) --- brief, synchronous neural events in the hippocampus --- gain attention for their roles in memory consolidation and retrieval. However, their association with WM tasks remains unclear. Our current research indicates that patterns of multiunit activity in the hippocampus might operate in synergy with SWRs, thereby exhibiting unique dynamics during WM tasks. We analyzed a dataset consisting of intracranial electroencephalogram recordings taken from the medial temporal lobe (MTL) of nine epilepsy patients engaged in an eight-second Sternberg task. Gaussian-process factor analysis was utilized to extract low-dimensional neural representations, known as 'trajectories', within the MTL regions while performing the WM task. Our results show that the hippocampus presents the most significant variation in neural trajectory compared to the entorhinal cortex and amygdala. Furthermore, the dissimilarity in trajectories between encoding and retrieval phases appeared to rely on memory load. Importantly, the hippocampal trajectories fluctuate during the retrieval phase, indicating task-dependent shifts between encoding and retrieval states, both during baseline and SWR events. This fluctuation transitions from encoding to retrieval states synchronously with the occurrence of SWRs. These findings reaffirm the crucial role of the hippocampus in WM tasks and advance a new hypothesis: the hippocampus transitions its functional state from encoding to retrieval during SWRs.
\end{abstract}