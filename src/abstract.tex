\begin{abstract}
\pdfbookmark[1]{Abstract}{abstract}
Working memory (WM) plays a pivotal role in multiple cognitive functions, yet the complex neural mechanisms supporting its operation continue to remain unclear. In particular, despite the recognized roles of the hippocampus and sharp-wave ripple complexes (SWRs) -- brief, synchronous neural oscillations observed in the hippocampus -- in memory consolidation and retrieval, their contribution to WM tasks remains undefined. We demonstrate that during a WM task, multiunit activity patterns in the hippocampus exhibit unique dynamics, particularly during SWR periods. This study analyzed a dataset obtained from intracranial electroencephalogram recordings performed in the medial temporal lobe (MTL) of nine epilepsy patients during an eight-second Sternberg task. We employed Gaussian-process factor analysis to identify low-dimensional neural representations, referred to as 'trajectories,' within the MTL regions during the WM task. The results depicted significant variations in hippocampal neural trajectories as opposed to those in the entorhinal cortex and amygdala. Moreover, the trajectory distance between the encoding and retrieval phases was memory load-dependent. Notably, hippocampal trajectories during the retrieval phase showcased oscillations between encoding and retrieval states, contingent on the task type, particularly displaying a transient shift from encoding to retrieval states during SWRs. These findings emphasize the hippocampus's central role in executing WM tasks and suggest a future research hypothesis: the functional state of the hippocampus transitions from encoding to retrieval during SWRs.
\end{abstract}