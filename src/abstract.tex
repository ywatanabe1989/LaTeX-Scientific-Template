\begin{abstract}
\pdfbookmark[1]{Abstract}{abstract}
Working memory (WM) is crucial for a multitude of cognitive functions, yet the intricacies of the neural mechanisms at play remain to be fully comprehended. Interestingly, the hippocampus and sharp-wave ripple complexes (SWRs) --- succinct, synchronous neural events in the hippocampus --- garner recognition for their significance in memory consolidation and retrieval, albeit their correlation with WM tasks is yet to be clarified. Our current study suggests that multiunit activity patterns in the hippocampus may work in tandem with SWRs, thereby displaying unique dynamics during WM tasks. We examined a dataset comprising intracranial electroencephalogram recordings obtained from the medial temporal lobe (MTL) of nine epilepsy patients engaged in an eight-second Sternberg task. Gaussian-process factor analysis was employed to extract low-dimensional neural representations, termed 'trajectories', within the MTL regions during the WM task. Our findings indicate that the hippocampus exhibits the most pronounced variation in neural trajectory in comparison with the entorhinal cortex and amygdala. Moreover, the dissimilarity in trajectories measured between encoding and retrieval phases was found to be contingent on memory load. Significantly, the hippocampal trajectories oscillate during the retrieval phase, demonstrating task-dependent shifts between encoding and retrieval states, both during baseline and SWR events. This fluctuation transitions from encoding to retrieval states concurrent with the occurrence of SWRs. These discoveries reassert the pivotal role of the hippocampus in WM tasks and propose a fresh hypothesis: the hippocampus transitions its functional state from encoding to retrieval in the midst of SWRs.
\end{abstract}