\begin{abstract}
\pdfbookmark[1]{Abstract}{abstract}
Working memory (WM) is central to numerous cognitive functions, yet the intricacy of its underlying neural mechanisms remains incompletely understood. Intriguingly, attention has been drawn to the roles that the hippocampus and sharp-wave ripple complexes (SWRs) --- which are brief, synchronous neural events occurring in the hippocampus --- play in memory consolidation and retrieval. Nevertheless, the connection between these elements and WM tasks remains uncertain. Current research indicates that the patterns of multiunit activity in the hippocampus may function synergistically with SWRs, exhibiting distinctive dynamics during WM tasks. We analyzed an electroencephalogram dataset from the medial temporal lobe (MTL) in nine epilepsy patients who engaged in an eight-second Sternberg task. Low-dimensional neural representations, or 'trajectories,' within the MTL regions were extracted using Gaussian-process factor analysis while performing the WM task. The results demonstrate that the hippocampus displays significant variation in neural trajectories compared to the entorhinal cortex and the amygdala. Furthermore, the trajectory dissimilarity between the encoding and retrieval phases appears to be reliant on memory load. Notably, hippocampal trajectories fluctuate during the retrieval phase, suggesting task-dependent shifts between encoding and retrieval states, which occur during both baseline and SWR events. These fluctuations transition from encoding to retrieval states in synchrony with SWRs' occurrence. This supports the crucial role of the hippocampus in WM tasks and proposes a new hypothesis: the hippocampus transitions its functional state from encoding to retrieval during SWRs.
\end{abstract}