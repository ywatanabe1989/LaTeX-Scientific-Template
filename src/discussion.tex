\section{Discussion}
The focus of this investigation was to validate the hypothesis that distinct neuronal representations, or trajectories, are expressed in hippocampal neurons during low-dimensional space working memory (WM) tasks experienced by humans, particularly during sharp-wave ripple (SWR) periods. Initially, we projected multiunit spikes from medial temporal lobe regions during a Sternberg task onto three-dimensional spaces using Gaussian-process factor analysis (GPFA) (Figure~\ref{fig:01}D--E and Figure~\ref{fig:02}A) \cite{yu_gaussian-process_2009}. The trajectory distance among WM phases ($\mathrm{\lVert g_{F}g_{E} \rVert}$, $\mathrm{\lVert g_{F}g_{M} \rVert}$, $\mathrm{\lVert g_{F}g_{R} \rVert}$, $\mathrm{\lVert g_{E}g_{M} \rVert}$, $\mathrm{\lVert g_{E}g_{R} \rVert}$, and $\mathrm{\lVert g_{M}g_{R} \rVert}$) was found to be larger in the hippocampus as compared to the entorhinal cortex (EC) and amygdala (Figure~\ref{fig:02}E). This suggests increased neuronal activity in the hippocampus during a WM task. Additionally, the trajectory distance between the encoding and retrieval phases in the hippocampus ($\mathrm{\lVert g_{F}g_{E} \rVert}$) showed a positive correlation with memory load (Figure~\ref{fig:03}C--D). This implies that it reflects WM processing. The neural trajectory in the hippocampus exhibited a transient increase during SWRs (Figure~\ref{fig:05}). Ultimately, the hippocampal neural trajectory transitioned from encoding to retrieval states during SWR events (Figure~\ref{fig:07}). Collectively, these findings underscore the role that hippocampal neural activity plays in a WM task in humans \cite{naber_reciprocal_2001,van_strien_anatomy_2009,strange_functional_2014}.

It was observed that the distance of the neural trajectory among the four phases was further in the hippocampus compared to the EC and amygdala, even when adjusting for the distance from origin $O$ ($\mathrm{\lVert g_{F} \rVert}$, $\mathrm{\lVert g_{E} \rVert}$, $\mathrm{\lVert g_{M} \rVert}$, and $\mathrm{\lVert g_{R} \rVert}$) in those areas (Figure~\ref{fig:02}C--E). This observation aligns with preceding reports of hippocampal persistent firing in the maintenance phase \cite{boran_persistent_2019,kaminski_persistently_2017,kornblith_persistent_2017,faraut_dataset_2018}, reinforcing the hippocampus's role in the WM task. Notably, we noted that when applying GPFA to multiunit activity during a one-second resolution WM task, the neural trajectory in low dimensional space intimated a memory-load dependency between the encoding and retrieval phases, represented as $\mathrm{\lVert g_{E}g_{R} \rVert}$ (Figure~\ref{fig:03}). This strengthens the established correlation between the hippocampus and WM processing \cite{oso_boran_2020}.

Our analysis, restricted to supposed CA1 regions (Figure~\ref{fig:04}), is justified by several factors. This focused tactic is buttressed by consistent reports that SWRs are synchronously associated with spike bursts of interneurons and pyramidal neurons \cite{buzsaki_two-stage_1989,quyen_cell_2008,royer_control_2012,hajos_input-output_2013}, potentially encompassing a 50 $\mu$m radius about the recording site \cite{schomburg_spiking_2012}. Within this study, we noticed an increase in SWR occurrences during the retrieval phase at 0--400 ms (Figure~\ref{fig:04}D). This mirrors prior studies displaying increased SWR incidences before spontaneous verbal recall \cite{norman_hippocampal_2019, norman_hippocampal_2021}. This observation not only reinforces previous findings, but also extends to a triggered retrieval stage. Furthermore, the log-normal distributions of SWR duration and ripple band peak amplitude observed in this study (Figure~\ref{fig:04}C \& E) conform with the consensus in the field \cite{liu_consensus_2022}, suggesting our approach likely improved the precision of SWR detection by limiting recording sites to likely CA1 regions. It is important to note that the increase in trajectory distance from origin $O$ during SWR (Figure~\ref{fig:05}) might be slightly skewed due to the channel selection, though this probability doesn't significantly impact our primary findings.

Interestingly, trajectory directions in the retrieval phase transitioned between encoding and retrieval states during both baseline and SWR periods (Figure~\ref{fig:07}C \& D). Furthermore, these fluctuations shifted from encoding to retrieval states during SWR (Figure~\ref{fig:07}E \& F). This concurs with prior working theories that proposed the role of SWR in memory recall \cite{norman_hippocampal_2019, norman_hippocampal_2021}. Our results enhance this understanding by specifying that SWRs happen when the hippocampal representation transitions from encoding to retrieval states. Therefore, our results provide novel insights into hippocampal representations, i.e., (i) neural fluctuations between encoding and retrieval states during a WM task and (ii) SWR acts as a mechanism enabling the shift from encoding to retrieval states \cite{buzsaki_hippocampal_2015}.

Moreover, our study identifies WM-task-specific directions between encoding and retrieval SWRs (Figure~\ref{fig:07}E--F). Notably, encoding SWR and retrieval SWR pointed in opposite directions during the 'Mismatch OUT' task and not during the 'Match IN' task. These results might align with the memory engram theory \cite{liu_optogenetic_2012}. In the 'Match IN' task, subjects were shown a previously shown letter, whereas the 'Mismatch OUT' task involved the introduction of a new letter not displayed during the encoding phase. These results suggest a relationship between SWR and working cognitive processes in humans.

To summarize, our study has established that hippocampal activity oscillates between encoding and retrieval states during a WM task and transitions notably from encoding to retrieval during SWR periods.

\label{sec:discussion}