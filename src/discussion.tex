\section{Discussion}
This study proposed that within low-dimensional spaces during a working memory (WM) task in humans, hippocampal neurons shape unique trajectories, especially during sharp-wave ripple (SWR) periods. Initially, we projected the multiunit spikes from medial temporal lobe (MTL) regions onto three-dimensional spaces during a Sternberg task using Gaussian Process Factor Analysis (GPFA) (Figure~\ref{fig:01}D--E and Figure~\ref{fig:02}A). We noted that the trajectory distance across WM phases ($\mathrm{\lVert g_{F}g_{E} \rVert}$, $\mathrm{\lVert g_{F}g_{M} \rVert}$, $\mathrm{\lVert g_{F}g_{R} \rVert}$, $\mathrm{\lVert g_{E}g_{M} \rVert}$, $\mathrm{\lVert g_{E}g_{R} \rVert}$, and $\mathrm{\lVert g_{M}g_{R} \rVert}$) was significantly larger in the hippocampus than in the entorhinal cortex (EC) and amygdala (Figure~\ref{fig:02}E). This revelation indicated dynamic neural activity in the hippocampus during the WM task. Furthermore, a positive correlation between the trajectory distance from the encoding to the retrieval phase ($\mathrm{\lVert g_{F}g_{E} \rVert}$) and memory load was seen in the hippocampus (Figure~\ref{fig:03}C--D), which reflected WM processing. Transient increases were discovered in the hippocampal neural trajectory during SWRs (Figure~\ref{fig:05}). Lastly, the hippocampal neural trajectory switched between encoding and retrieval states, transitioning from encoding to retrieval during SWR events (Figure~\ref{fig:07}). These findings not only elucidate varying aspects of hippocampal neural activity during a WM task in humans but also offer new insights into how SWRs influence the switch in neural states.

The distance of the neural trajectory across the phases was discovered to be greater in the hippocampus compared to that in the EC and amygdala, even when considering the distance from $O$ in these regions (Figure~\ref{fig:02}C--E). This evidence backs the involvement of the hippocampus in the WM task, in agreement with prior reports of persistent firing within the hippocampus during the maintenance phase \cite{boran_persistent_2019} \cite{kaminski_persistently_2017} \cite{kornblith_persistent_2017} \cite{faraut_dataset_2018}. However, when GPFA was applied to multiunit activity at 1-second-level resolution during the WM task, the neural trajectory in low-dimensional space displayed a memory load dependence between the encoding and retrieval phases, expressed as $\mathrm{\lVert g_{E}g_{R} \rVert}$ (Figure~\ref{fig:03}). This result reinforces the association between the hippocampus and WM processing.

The analysis concentrated on putative CA1 regions (Figure~\ref{fig:04}), corroborated by multiple factors. This specific focus originates from established observations that SWRs synchronize with spike bursts of interneurons and pyramidal neurons \cite{buzsaki_two-stage_1989} \cite{quyen_cell_2008} \cite{royer_control_2012} \cite{hajos_input-output_2013}, potentially within a 50 $\mu$m radius of the recording site \cite{schomburg_spiking_2012}. We also noticed a heightened incidence of SWRs during the first 0--400 ms of the retrieval phase (Figure~\ref{fig:04}D), which is consistent with previous findings of increased SWR occurrence preceding spontaneous verbal recall \cite{norman_hippocampal_2019} \cite{norman_hippocampal_2021}, reinforcing our results under a triggered retrieval condition. The observed log-normal distributions of both SWR duration and ripple band peak amplitude in this research (Figure~\ref{fig:04}C \& E) conform to the consensus in this field \cite{liu_consensus_2022}. Consequently, limiting recording sites to putative CA1 regions likely improved the accuracy of SWR detection. However, the increase in trajectory distance from $O$ during SWRs (Figure~\ref{fig:05}) may have been biased towards higher values due to channel selection. Nevertheless, this potential bias does not significantly challenge our main findings.

Interestingly, trajectory directions oscillated between encoding and retrieval states during both baseline and SWR periods in the retrieval phase (Figure~\ref{fig:07}C \& D). Moreover, the balance of this oscillation switched from the encoding state to the retrieval state during SWR events (Figure~\ref{fig:07} E \& F). These results align with previous reports on the role of SWR in memory retrieval \cite{norman_hippocampal_2019} \cite{norman_hippocampal_2021}. Our findings shed a new light, suggesting that SWRs occur when the hippocampal representation transitions from encoding to retrieval states. Thus, these findings unmask two novel aspects of hippocampal representations: (i) neuronal oscillation between encoding and retrieval states during a WM task and (ii) SWR acting as a trigger for shifting neural states.

Additionally, our study unveiled WM-task type-specific differences between encoding- and retrieval-SWRs (Figure~\ref{fig:07}E--F). Importantly, opposing movements of encoding-SWR (eSWR) and retrieval-SWR (rSWR) were not demonstrated in the Match IN task but were clear in the Mismatch OUT task. These results can be explained by memory engram theory \cite{liu_optogenetic_2012}, which posits that the Match In task provided participants with previously presented letters, whereas the Mismatch OUT task introduced a new letter not present in the encoding phase. These interpretations underline the significant role of SWR in human cognitive processes.

In conclusion, this investigation demonstrated that hippocampal activity oscillates between encoding and retrieval states during a WM task and uniquely transitions from encoding to retrieval during SWR events. These findings offer substantial insight into the neural mechanisms and functionality of working memory in the hippocampus.
\label{sec:discussion}
