\section{Discussion}
This study aimed to validate the hypothesis that unique neuronal patterns, or trajectories, demonstrate significant activity in the hippocampus during low-dimensional space working memory (WM) tasks undertaken by humans, especially during sharp-wave ripple (SWR) periods. To begin with, multiunit spikes from the medial temporal lobe regions were projected onto three-dimensional spaces using Gaussian-process factor analysis (GPFA) during a Sternberg task (Figure~\ref{fig:01}D--E and Figure~\ref{fig:02}A) \cite{yu_gaussian-process_2009}. The trajectory distance among WM phases ($\mathrm{\lVert g_{F}g_{E} \rVert}$, $\mathrm{\lVert g_{F}g_{M} \rVert}$, $\mathrm{\lVert g_{F}g_{R} \rVert}$, $\mathrm{\lVert g_{E}g_{M} \rVert}$, $\mathrm{\lVert g_{E}g_{R} \rVert}$, and $\mathrm{\lVert g_{M}g_{R} \rVert}$) was observed to be greater in the hippocampus than in the entorhinal cortex (EC) and amygdala (Figure~\ref{fig:02}E), pointing to heightened neuronal activity within the hippocampus during a WM task. Trajectory distance between the encoding and retrieval phases in the hippocampus ($\mathrm{\lVert g_{F}g_{E} \rVert}$) showed a positive association with memory load (Figure~\ref{fig:03}C--D), indicating its role in WM processing. A transient increase was observed in the hippocampal neural trajectory during SWRs (Figure~\ref{fig:05}). Ultimately, the hippocampal neural trajectory transitioned from encoding to retrieval states during SWR events (Figure~\ref{fig:07}). The aforementioned findings highlight the critical role that hippocampal neural activity plays in human WM task completion \cite{naber_reciprocal_2001,van_strien_anatomy_2009,strange_functional_2014}.

We observed that the neural trajectory distance among the four phases was longer in the hippocampus than in the EC and amygdala, even when adjusting for the distance from origin $O$ ($\mathrm{\lVert g_{F} \rVert}$, $\mathrm{\lVert g_{E} \rVert}$, $\mathrm{\lVert g_{M} \rVert}$, and $\mathrm{\lVert g_{R} \rVert}$) in these regions (Figure~\ref{fig:02}C--E). These observations align with prior reports of enduring hippocampal activity in the maintenance phase \cite{boran_persistent_2019,kaminski_persistently_2017,kornblith_persistent_2017,faraut_dataset_2018}, reinforcing the hippocampus's involvement in the WM task. Notably, when we applied the GPFA to multiunit activity during a one-second resolution WM task, the neural trajectory in low-dimensional space demonstrated a memory-load dependency between the encoding and retrieval phases ($\mathrm{\lVert g_{E}g_{R} \rVert}$) (Figure~\ref{fig:03}). This enriches the prevailing understanding of the link between the hippocampus and WM processing \cite{oso_boran_2020}.

Our analysis, narrowed down to presumed CA1 regions (Figure~\ref{fig:04}), is supported by several factors. This focused approach aligns with regular reports demonstrating that SWRs synchronously associate with interneuronal and pyramidal neuronal spike bursts \cite{buzsaki_two-stage_1989,quyen_cell_2008,royer_control_2012,hajos_input-output_2013}, potentially encapsulating a 50 $\mu$m radius around the recording site \cite{schomburg_spiking_2012}. We noted an upswing in SWR occurrences during the retrieval phase at 0--400 ms (Figure~\ref{fig:04}D). This reinforces and extends comparable previous findings of increased SWR instances preceding spontaneous verbal recall to a driven retrieval stage \cite{norman_hippocampal_2019, norman_hippocampal_2021}. We also observed log-normal distributions of SWR duration and ripple band peak amplitude (Figure~\ref{fig:04}C \& E), which align with the current consensus in the field \cite{liu_consensus_2022}, suggesting that our method of restricting the recording sites to probable CA1 regions might have enhanced the precision of SWR detection. However, we must note that the increased trajectory distance from origin $O$ during SWR (Figure~\ref{fig:05}) might be slightly shifted due to the channel selection, though this likelihood does not significantly influence our key findings.

Intriguingly, trajectory directions throughout the retrieval phase alternated between encoding and retrieval states during both baseline and SWR periods (Figure~\ref{fig:07}C \& D). Moreover, these fluctuations transitioned from encoding to retrieval states during SWR (Figure~\ref{fig:07}E \& F). These findings are consistent with previous theories proposing the role of SWR in memory recall \cite{norman_hippocampal_2019, norman_hippocampal_2021}. Our results build on this understanding by specifying that SWRs occur when the hippocampal neural pattern transitions from encoding to retrieval states. Consequently, our findings provide novel insights into hippocampal representations, i.e., (i) the switching of neural patterns between encoding and retrieval states during a WM task, and (ii) SWR functioning as a mechanism facilitating the transition from encoding to retrieval states \cite{buzsaki_hippocampal_2015}.

Further, our study finds specific WM-task-related directions between encoding and retrieval SWRs (Figure~\ref{fig:07}E--F). Intriguingly, encoding SWR and retrieval SWR showed opposing directions during the 'Mismatch OUT' task, not observed during the 'Match IN' task. These findings may support the memory engram theory \cite{liu_optogenetic_2012}. In the 'Match IN' task, subjects were shown a previously displayed letter, while the 'Mismatch OUT' task introduced a new letter not shown during the encoding phase. These results suggest a connection between SWR and human cognitive processes during working memory tasks.

In conclusion, our study demonstrates that in a WM task, hippocampal activity oscillates between encoding and retrieval states and notably shifts from encoding to retrieval during SWR periods.

\label{sec:discussion}