\section{Discussion}
This study posits that during a working memory (WM) task in humans, hippocampal neurons form distinct trajectories in low-dimensional spaces, particularly during sharp-wave ripples (SWR) periods. Initially, multiunit spikes in the medial temporal lobe (MTL) regions were projected onto three-dimensional spaces during a Sternberg task, using Gaussian-process factor analysis (GPFA) (Figure~\ref{fig:01}D--E \& Figure~\ref{fig:02}A). The distances of the trajectories across WM phases ($\mathrm{\lVert g_{F}g_{E} \rVert}$, $\mathrm{\lVert g_{F}g_{M} \rVert}$, $\mathrm{\lVert g_{F}g_{R} \rVert}$, $\mathrm{\lVert g_{E}g_{M} \rVert}$, $\mathrm{\lVert g_{E}g_{R} \rVert}$, and $\mathrm{\lVert g_{M}g_{R} \rVert}$) were significantly larger in the hippocampus than in the entorhinal cortex (EC) and amygdala (Figure~\ref{fig:02}E), indicating dynamic neural activity in the hippocampus during the WM task. In addition, within the hippocampus, the distance of the trajectory between the encoding and retrieval phases ($\mathrm{\lVert g_{F}g_{E} \rVert}$) showed a positive correlation with memory load (Figure~\ref{fig:03}C--D), which reflects WM processing. The hippocampal neural trajectory briefly expanded during SWR events (Figure~\ref{fig:05}) and alternated between encoding and retrieval states, transitioning from the encoding to the retrieval state during SWR events (Figure~\ref{fig:07}). These findings provide insights into hippocampal neural activity during a WM task in humans and propose SWRs as crucial to the shift in hippocampal neural states.

The trajectory distance across phases was substantially longer in the hippocampus than in the EC and amygdala, even considering distances from $O$ in these regions (Figure~\ref{fig:02}C--E). This reinforces the role of the hippocampus in the WM task—consistent with earlier studies showing persistent hippocampal firing during the task's maintenance phase \cite{boran_persistent_2019} \cite{kaminski_persistently_2017} \cite{kornblith_persistent_2017} \cite{faraut_dataset_2018}. Applying GPFA to multiunit activity at one-second resolution during the WM task, this study found that the neural trajectory in low-dimensional space exhibited a memory-load dependency between the encoding and retrieval phases, represented as $\mathrm{\lVert g_{E}g_{R} \rVert}$ (Figure~\ref{fig:03}). These results support the hippocampus's association with WM processing.

Our analysis targeted putative CA1 regions (Figure~\ref{fig:04}), a decision supported by several factors. This specific focus stems from existing observations that SWRs synchronize with interneuron and pyramid neuron spike bursts \cite{buzsaki_two-stage_1989} \cite{quyen_cell_2008} \cite{royer_control_2012} \cite{hajos_input-output_2013}, potentially within a 50 $\mu$m radius of the recording site \cite{schomburg_spiking_2012}. Moreover, an elevated incidence of SWRs was identified during the first 0--400 ms of the retrieval phase (Figure~\ref{fig:04}D), aligning with previous reports of increased SWR occurrence before spontaneous verbal recall \cite{norman_hippocampal_2019} \cite{norman_hippocampal_2021}, which supports our results under a triggered retrieval condition. The observed log-normal distributions of both SWR duration and ripple band peak amplitude (Figure~\ref{fig:04}C \& E) agree with the current consensus in this scientific domain \cite{liu_consensus_2022}. Therefore, restricting recording sites to putative CA1 regions likely improved the precision, or true positive rate, of SWR detection. However, the trajectory distance increase from $O$ during SWRs (Figure~\ref{fig:05}) may be artificially inflated towards higher values due to channel selection. This potential bias does not significantly affect our main conclusions.

Interestingly, the trajectory directions oscillated between encoding and retrieval states during both baseline and SWR periods in a task-dependent manner during the retrieval phase (Figure~\ref{fig:07}C \& D). In addition, the balance of this fluctuation transitioned from the encoding to the retrieval state during SWR events (Figure~\ref{fig:07}E \& F). These results align with earlier studies on SWR's role in memory retrieval \cite{norman_hippocampal_2019} \cite{norman_hippocampal_2021}. Our findings suggest that: (i) neuronal oscillation between encoding and retrieval states occurs during a WM task, and (ii) SWR events indicate the transition from encoding to retrieval states during a WM task.

Furthermore, our study observed WM-task type-specific differences between encoding-SWRs (eSWR) and retrieval-SWRs (rSWR) (Figure~\ref{fig:07}E--F). Notably, opposing movements of eSWR and rSWR were not observed in the Match IN task but were apparent in the Mismatch OUT task. This observation could be attributed to memory engram theory \cite{liu_optogenetic_2012}. The Match IN task presented the participants with previously shown letters, while the Mismatch OUT task introduced a new letter that was not present in the encoding phase. This suggests the essential role of SWR in human cognitive processes.

In conclusion, this study shows that during a WM task in humans, hippocampal activity transitions between encoding and retrieval states, specifically shifting from encoding to retrieval during SWR events. These findings offer novel insights into the neural correlates and functionality of working memory within the hippocampus.
\label{sec:discussion}