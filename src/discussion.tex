
    \section{Discussion}
This study hypothesized that hippocampal neurons express distinct representations, or trajectories, in low-dimensional spaces during a working memory (WM) task in humans, specifically during sharp-wave ripple (SWR) periods. Initially, we projected the multiunit spikes from medial temporal lobe regions during a Sternberg task onto three-dimensional spaces using Gaussian-process factor analysis (GPFA) (Figure~\ref{fig:01}D--E and Figure~\ref{fig:02}A) \cite{yu_gaussian-process_2009}. The trajectory distance amongst WM phases ($\mathrm{\lVert g_{F}g_{E} \rVert}$, $\mathrm{\lVert g_{F}g_{M} \rVert}$, $\mathrm{\lVert g_{F}g_{R} \rVert}$, $\mathrm{\lVert g_{E}g_{M} \rVert}$, $\mathrm{\lVert g_{E}g_{R} \rVert}$, and $\mathrm{\lVert g_{M}g_{R} \rVert}$) was larger in the hippocampus than in the entorhinal cortex (EC) and amygdala (Figure~\ref{fig:02}E), suggesting increased neural activity in the hippocampus during the WM task. Moreover, the trajectory distance between the encoding and retrieval phases in the hippocampus ($\mathrm{\lVert g_{F}g_{E} \rVert}$) displayed a positive correlation with memory load (Figure~\ref{fig:03}C--D), indicating that it reflects WM processing. The neural trajectory in the hippocampus also showed a transient increase during SWRs (Figure~\ref{fig:05}). Finally, the hippocampal neural trajectory alternated between encoding and retrieval states, transitioning from encoding to retrieval during SWR events (Figure~\ref{fig:07}). Overall, these findings emphasize the role of hippocampal neural activity in a WM task in humans \cite{naber_reciprocal_2001,van_strien_anatomy_2009,strange_functional_2014}.

We found that the distance of the neural trajectory among the four phases was longer in the hippocampus compared to the EC and amygdala, even when adjusting for the distance from origin $O$ ($\mathrm{\lVert g_{F} \rVert}$, $\mathrm{\lVert g_{E} \rVert}$, $\mathrm{\lVert g_{M} \rVert}$, and $\mathrm{\lVert g_{R} \rVert}$) in those regions (Figure~\ref{fig:02}C--E). These findings align with existing reports of hippocampal persistent firing in the maintenance phase \cite{boran_persistent_2019} \cite{kaminski_persistently_2017} \cite{kornblith_persistent_2017} \cite{faraut_dataset_2018}, reinforcing the role of the hippocampus in the WM task. Notably, by applying GPFA to multiunit activity during a one-second resolution WM task, we observed that the neural trajectory in low dimensional space exhibits a memory-load dependency between the encoding and retrieval phases, represented as $\mathrm{\lVert g_{E}g_{R} \rVert}$ (Figure~\ref{fig:03}). These results reaffirm the association between the hippocampus and WM processing \cite{oso_boran_2020}.

The reliability of our analysis, confined to presumed CA1 regions (Figure~\ref{fig:04}), is supported by several factors. This focused approach is grounded in consistent reports that SWRs are synchronous with spike bursts of interneurons and pyramidal neurons \cite{buzsaki_two-stage_1989} \cite{quyen_cell_2008} \cite{royer_control_2012} \cite{hajos_input-output_2013}, potentially within a 50 $\mu$m radius of the recording site \cite{schomburg_spiking_2012}. In this study, we noted an increase in SWR occurrences at 0--400 ms of the retrieval phase (Figure~\ref{fig:04}D), paralleling previous studies demonstrating increased SWR occurrences before spontaneous verbal recall \cite{norman_hippocampal_2019} \cite{norman_hippocampal_2021}. This finding not only corroborates earlier observations, but also broadens them by extending to a triggered retrieval stage. Additionally, the log-normal distributions of SWR duration and ripple band peak amplitude observed in this study (Figure~\ref{fig:04}C & E) are consistent with the consensus in the field \cite{liu_consensus_2022}. Therefore, our approach of restricting recording sites to probable CA1 regions likely enhanced the accuracy of SWR detection. It is essential to mention that the increase in trajectory distance from origin $O$ during SWR (Figure~\ref{fig:05}) may be slightly skewed due to the channel selection; however, this does not severely impact our primary findings.

Interestingly, trajectory directions in the retrieval phase alternated between encoding and retrieval states in both baseline and SWR periods (Figure~\ref{fig:07}C \& D). Moreover, these fluctuations transitioned from encoding to retrieval states during SWR (Figure~\ref{fig:07}E \& F). These findings concur with previous working theories suggesting SWR’s role in memory recall \cite{norman_hippocampal_2019} \cite{norman_hippocampal_2021}. Our results add to this understanding, specifying that SWRs occur when the hippocampal representation transitions from encoding to retrieval states. Hence, our findings provide novel insights into hippocampal representations: (i) neural fluctuations between encoding and retrieval states during a WM task and (ii) SWR as a mechanism enabling the transition from encoding to retrieval states \cite{buzsaki_hippocampal_2015}.

Additionally, our study uncovers WM-task-specific directions between encoding and retrieval SWRs (Figure~\ref{fig:07}E--F). Notably, encoding SWR and retrieval SWR pointed in opposing directions not during a Match IN but during a Mismatch OUT task. These findings may align with the memory engram theory \cite{liu_optogenetic_2012}. Indeed, the Match In task presented subjects with a previously seen letter, while the Mismatch OUT task introduced a new letter not shown in the encoding phase. These results suggest that SWR relates to working cognitive processes in humans.

In conclusion, our study has demonstrated that hippocampal activity oscillates between encoding and retrieval states during a WM task and shifts significantly from encoding to retrieval during SWR periods.

\label{sec:discussion}