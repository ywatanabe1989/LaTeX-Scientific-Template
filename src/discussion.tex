\section{Discussion}
This study posited that hippocampal neurons demonstrate unique representations, or trajectories, in low-dimensional spaces during a working memory (WM) task in humans, specifically during sharp-wave ripple (SWR) periods. Initially, we projected the multiunit spikes from medial temporal lobe regions during a Sternberg task onto three-dimensional spaces using Gaussian-process factor analysis (GPFA) (Figure~\ref{fig:01}D--E and Figure~\ref{fig:02}A) \cite{yu_gaussian-process_2009}. The trajectory distance among WM phases ($\mathrm{\lVert g_{F}g_{E} \rVert}$, $\mathrm{\lVert g_{F}g_{M} \rVert}$, $\mathrm{\lVert g_{F}g_{R} \rVert}$, $\mathrm{\lVert g_{E}g_{M} \rVert}$, $\mathrm{\lVert g_{E}g_{R} \rVert}$, and $\mathrm{\lVert g_{M}g_{R} \rVert}$) was larger in the hippocampus than in the entorhinal cortex (EC) and amygdala (Figure~\ref{fig:02}E), indicating more dynamic neural activity in the hippocampus during the WM task. Moreover, the trajectory distance between the encoding and retrieval phases in the hippocampus ($\mathrm{\lVert g_{F}g_{E} \rVert}$) positively correlated with memory load (Figure~\ref{fig:03}C--D), thereby marking it as a reflection of WM processing. The neural trajectory in the hippocampus also exhibited a transient increase during SWRs (Figure~\ref{fig:05}). Lastly, the hippocampal neural trajectory fluctuated between encoding and retrieval states, transitioning from encoding to retrieval during SWR events (Figure~\ref{fig:07}). Overall, these findings highlight the role of hippocampal neural activity in a WM task in humans \cite{naber_reciprocal_2001,van_strien_anatomy_2009,strange_functional_2014}.

We discovered that the neural trajectory's distance among the four phases was longer in the hippocampus compared to the EC and amygdala, even when adjusting for the distance from origin $O$ ($\mathrm{\lVert g_{F} \rVert}$, $\mathrm{\lVert g_{E} \rVert}$, $\mathrm{\lVert g_{M} \rVert}$, and $\mathrm{\lVert g_{R} \rVert}$) in those regions (Figure~\ref{fig:02}C--E). These findings corroborate the role of the hippocampus in the WM task, which aligns with previous reports of hippocampal persistent firing in the maintenance phase \cite{boran_persistent_2019} \cite{kaminski_persistently_2017} \cite{kornblith_persistent_2017} \cite{faraut_dataset_2018}. However, by applying GPFA to multiunit activity during a one-second level resolution of WM task, we found that the low dimensional space neural trajectory exhibits a memory-load dependency between the encoding and retrieval phases, represented as $\mathrm{\lVert g_{E}g_{R} \rVert}$ (Figure~\ref{fig:03}). These results reaffirm the association between the hippocampus and WM processing \cite{oso_boran_2020}.

The reliability of our analysis, restricted to presumed CA1 regions (Figure~\ref{fig:04}), is supported by several contributing factors. This focused approach is grounded in consistent observations that SWRs are synchronous with spike bursts of interneurons and pyramidal neurons \cite{buzsaki_two-stage_1989} \cite{quyen_cell_2008} \cite{royer_control_2012} \cite{hajos_input-output_2013}, potentially within a 50 $\mu$m radius of the recording site \cite{schomburg_spiking_2012}. In this study, we observed an increase in SWR occurrences at 0--400 ms of the retrieval phase (Figure~\ref{fig:04}D), aligning with prior studies demonstrating increased SWR occurrences before spontaneous verbal recall \cite{norman_hippocampal_2019} \cite{norman_hippocampal_2021}. This finding not only corroborates but broadens the observation to a triggered retrieval condition. Additionally, the log-normal distributions of SWR duration and ripple band peak amplitude observed in this study (Figure~\ref{fig:04}C & E) correlate with the consensus in this field \cite{liu_consensus_2022}. Hence, our approach of confining recording sites to probable CA1 regions likely improved the precision of SWR detection. It should be noted that an increase in trajectory distance from origin $O$ during SWR (Figure~\ref{fig:05}) may have been skewed to a greater extent due to the channel selection; however, this does not dramatically impact our primary findings.

Interestingly, trajectory directions in the retrieval phase oscillated between encoding and retrieval states both in baseline and SWR periods (Figure~\ref{fig:07}C \& D). Furthermore, the balance of these fluctuations transitioned from encoding to retrieval states during SWR (Figure~\ref{fig:07}E \& F). These findings concur with previous studies suggesting SWR's role in memory recall \cite{norman_hippocampal_2019} \cite{norman_hippocampal_2021}. Our results add another layer of understanding, specifying that SWRs occur when hippocampal representation transitions from encoding to retrieval states. Therefore, our findings offer new insights into hippocampal representations: (i) neural fluctuations between encoding and retrieval states during a WM task and (ii) SWR as a mechanism facilitating the transition from encoding to retrieval states \cite{buzsaki_hippocampal_2015}.

Additionally, our research reveals WM-task-specific directions between encoding and retrieval SWRs (Figure~\ref{fig:07}E--F). Particularly, encoding SWR and retrieval SWR pointed in opposing directions not in a Match IN but in a Mismatch OUT task. These results might align with the memory engram theory \cite{liu_optogenetic_2012}. Indeed, the Match In task showed subjects a once-seen letter, while the Mismatch OUT task presented a novel letter not introduced in the encoding phase. These outcomes suggest that SWR relates to the working cognitive processes in humans.

In conclusion, our study has demonstrated that hippocampal activity fluctuates between encoding and retrieval states during a WM task and undergoes a significant shift from encoding to retrieval during SWR periods.

\label{sec:discussion}