%%%%%%%%%%%%%%%%%%%%%%%%%%%%%%%%%%%%%%%%%%%%%%%%%%%%%%%%%%%%%%%%%%%%%%%%%%%%%%%%
%% Methods
%%%%%%%%%%%%%%%%%%%%%%%%%%%%%%%%%%%%%%%%%%%%%%%%%%%%%%%%%%%%%%%%%%%%%%%%%%%%%%%%
\section{Methods}
\subsection{Dataset}
A publicly accessible dataset \cite{boran_dataset_2020} was employed, in which nine subjects performed a modified Sternberg task that consisted of fixation (1 s), encoding (2 s), maintenance (3 s), and retrieval (2 s) phases \cite{boran_dataset_2020}. During the encoding phase, participants were presented sets of either four, six, or eight alphabetical letters (set size). Subsequently, they were tasked with ascertaining whether a probe letter presented in the retrieval phase had been displayed (the correct choice for Match IN task) or not (the correct choice for Mismatch OUT task). iEEG signals were recorded at a sampling rate of 32 kHz, within the frequency range of 0.5--5,000 Hz, using depth electrodes implanted in the medial temporal regions. Specifically, electrodes were placed in the left and right hippocampal head (AHL and AHR), hippocampal body (PHL and PHR), entorhinal cortex (ECL and ECR), and amygdala (AL and AR) (Figure 1A and Table 1). Subsequently, iEEG signals were resampled at a rate of 2 kHz. Correlations were found among the experimental variables such as set size and correct rate (Figure S1). The timings of multiunit spikes were estimated by a spike sorting algorithm \cite{niediek_reliable_2016} by the Combinato package ((\url{https://github.com/jniediek/combinato})(Figure 1C).

\subsection{Calculation of neural trajectories using GPFA}
To calculate the neural trajectories (also referred to as factors; Figure 1D) in the hippocampus, EC, and amygdala (Figure 1D), GPFA \cite{yu_gaussian-process_2009}, was employed on the multiunit activity data for each session. GPFA was applied using the elephant package ((\url{https://elephant.readthedocs.io/en/latest/reference/gpfa.html}). The bin size was set as 50 ms, with no overlaps. Each factor was z-normalized across each session. From the trajectories, Euclidean distance from the origin ($O$ (0,0,0)) was calculated (Figure 1E).
\\
\indent
For every trajectory within a region (\textit{e.g.}, AHL), \textit{geometric medians} were determined by calculating the median coordinates of trajectory during the four phases: (\textit{i.e.}, $\mathrm{g_{F}}$ for fixation, $\mathrm{g_{E}}$ for encoding, $\mathrm{g_{M}}$ for maintenance, and $\mathrm{g_{R}}$ for retrieval phase) (Figure 1D). The optimal dimensionality for GPFA was determined as three via the elbow method utilizing the log-likelihood values using the threefold cross-validation approach (Figure 2B).

\subsection{Defining SWR candidates from hippocampal regions}
To identify potential SWR events within the hippocampus, we employed a detection method aligned with a consensus in this field \cite{liu_consensus_2022}. Specifically, local field potential (LFP) signals from a region of interest (ROI), such as AHL, were re-referenced by subtracting a control signal obtained by averaging signals from outside the ROI (\textit{e.g.}, AHR, PHL, PHR, ECL, ECR, AL, and AR) (see Figure 1A). The LFP signals were applied to a ripple-band filter (80--140 Hz) to isolate SWR candidates (SWR$^+$ candidates) (see Figure 1B). SWR detection was conducted using a published tool ((\url{https://github.com/Eden-Kramer-Lab/ripple_detection}) \cite{kay_hippocampal_2016}, with modifications such as an updated bandpass range of 80--140 Hz for humans from original 150--250 Hz range primarily for rodents.
\\
\indent
As control events for SWR$^+$ candidates, SWR$^-$ candidates were defined by shuffling the timestamps of SWR$^+$ candidates across all trials from all subjects. Finally, the SWR$^+$/SWR$^-$ candidates were visually inspected (see Figure 1).

\subsection{Defining SWRs from putative hippocampal CA1 regions}
SWRs were defined from SWR candidates as follows. First, putative CA1 regions were defined as follows. First, SWR$^+$/SWR$^-$ candidates in the hippocampus were embedded into a two-dimensional space based on their superimposed spike counts per unit using UMAP (uniform manifold approximation and projection) \cite{mcinnes_umap_2018} in a supervised fashion (Figure 4A). The silhouette score \cite{rousseeuw_silhouettes_1987}, a validation barometer for clustering, was calculated from clustered samples (Table 2). The hippocampal regions with silhouette scores greater than 0.6 on average across sessions $75^{th}$ percentile) (Figure 4B) were defined as putative CA1 regions, identifying five eletrode positions from five patients (Table 3).
\\
\indent
As a second step, SWR$^+$/SWR$^-$ candidates in putative CA1 regions were defined as SWR$^+$/SWR$^-$ (no longer candidates). The duration and ripple band peak amplitude of detected SWRs were calculated (Figure 4C \& E). SWR$^+$/SWR$^-$ were visually inspected as shown in Figure 1. The duration and ripple band peak amplitudes of detected SWRs were quantified (Figure 4C \& E). Each SWR was split into pre-SWR (= event at $-800$ to $-300$ ms from SWR center), mid-SWR (= event at $-250$ to $+250$ ms from SWR center), and post-SWR (= event at $+300$ to $+800$ ms from SWR center).

\subsection{Statistical evaluation}
The Brunner--Munzel test and the Kruskal-Wallis test were executed using the scipy package in Python \cite{virtanen_scipy_2020}. A correlational analysis was undertaken through the determination of the rank of the observed correlation coefficient in the associated set-size-shuffled surrogate, using a custom Python script. Additionally, the bootstrap test was carried out by utilizing an internally developed Python script.
\label{sec:methods}