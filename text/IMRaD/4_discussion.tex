%%%%%%%%%%%%%%%%%%%%%%%%%%%%%%%%%%%%%%%%%%%%%%%%%%%%%%%%%%%%%%%%%%%%%%%%%%%%%%%%
%% Discussion
%%%%%%%%%%%%%%%%%%%%%%%%%%%%%%%%%%%%%%%%%%%%%%%%%%%%%%%%%%%%%%%%%%%%%%%%%%%%%%%%
\section{Discussion}
This study hypothesized that hippocampal neurons exhibit distinct representations, or trajectories, in low-dimensional spaces during a WM task in humans, particularly during SWR periods. First, we projected the multiunit spikes in MTL regions during a Sternberg task onto three-dimensional spaces by GPFA (Figures 1D--E and Figure 2A). The distance of trajectory among WM phases ($\mathrm{\lVert g_{F}g_{E} \rVert}$, $\mathrm{\lVert g_{F}g_{M} \rVert}$, $\mathrm{\lVert g_{F}g_{R} \rVert}$, $\mathrm{\lVert g_{E}g_{M} \rVert}$, $\mathrm{\lVert g_{E}g_{R} \rVert}$, and $\mathrm{\lVert g_{M}g_{R} \rVert}$) was larger in the hippocampus than the EC and amygdala (Figure 2E), showing more dynamical neural activity in the hippocampus during the WM task. Additionally, the distance of trajectory between the encoding and retrieval phases in the hippocampus ($\mathrm{\lVert g_{F}g_{E} \rVert}$) was positively correlated with memory load (Figures 3C \& D), indicating it as a reflection of WM processing. Furthermore, the neural trajectory in the hippocampus showed transient increase during SWRs (Figure 5). Finally, the hippocampal neural trajectory fluctuated between encoding and retrieval states, with a shift from encoding to retrieval during SWR events (Figure 7). In sum, these results demonstrated the hippocampal neural behavior in a WM task in humans.
% For visualization purposes, peri-SWR trajectories were embedded into two-dimensional spaces, showing task-specifi
\\
\indent
First, we found that the distance of the neural trajectory among the four phases ($\mathrm{\lVert g_{F}g_{E} \rVert}$, $\mathrm{\lVert g_{F}g_{M} \rVert}$, $\mathrm{\lVert g_{F}g_{R} \rVert}$, $\mathrm{\lVert g_{E}g_{M} \rVert}$, $\mathrm{\lVert g_{E}g_{R} \rVert}$, and $\mathrm{\lVert g_{M}g_{R} \rVert}$) was longer in the hippocampus compared to the EC and amygdala, even considering the distance from $O$ ($\mathrm{\lVert g_{F} \rVert}$, $\mathrm{\lVert g_{E} \rVert}$, $\mathrm{\lVert g_{M} \rVert}$, and $\mathrm{\lVert g_{R} \rVert}$) in those regions (Figures 2C--E). These results indicate hippocampal participation in the WM task, which is partially supported by previous findings of hippocampal persistent firing in the maintenance phase \cite{boran_persistent_2019} \cite{kaminski_persistently_2017} \cite{kornblith_persistent_2017} \cite{faraut_dataset_2018}. However, by applying GPFA to multiunit activity during the 1-s level resolution of WM task, we revealed that the neural trajectory in low dimensional space displayed memory-load dependency between the encoding and retrieval phase, represented as $\mathrm{\lVert g_{E}g_{R} \rVert}$ (Figure 3). Overall, these results provide evidence that the hippocampus is linked to WM processing.
\\
\indent
The validity of our analysis of confining to putative CA1 regions (Figure 4) is supported by several factors. First, this targeted approach stems from well-established observations that SWRs are time-locked to synchronous spike bursts of interneurons and pyramidal neurons \cite{buzsaki_two-stage_1989} \cite{quyen_cell_2008} \cite{royer_control_2012} \cite{hajos_input-output_2013}, potentially around 50 $\mu$m radius of the recording site \cite{schomburg_spiking_2012}. Additionally, in the present study, we found the increase in SWRs' incidence at 0--400 ms of the retrieval phase (Figure 4D). This result is consistent with previous reports showing increased SWR occurrence before spontaneous verbal recall \cite{norman_hippocampal_2019} \cite{norman_hippocampal_2021}. Thus, our result is not only consistent but also extends the finding to a triggered retrieval condition. Moreover, the log-normal distributions of SWR duration and ripple band peak amplitude observed in this study (Figure 4C \& E) align with the consensus in this field \cite{liu_consensus_2022}. Therefore, our approach of limiting recording sites for putative CA1 regions would have contributed to precision of SWR detection. One limitation is that the increase in trajectory distance from $O$ during SWR (Figure 5) would have been biased to greater due to the channel selection; however, this is not critical for our major findings.
\\
\indent
Interestingly, the trajectory directions in the retrieval phase oscillated between the encoding and retrieval states both in baseline and SWR periods (Figures 7C \& D). In addition, the balance of such fluctuation was shifted from the encoding to retrieval state during SWR (Figures 7 E \& F). 
These results are again consistent with previous reports suggesting SWR's role in memory recall \cite{norman_hippocampal_2019} \cite{norman_hippocampal_2021}. Our result adds another layer of understanding, that is, SWR occurs when hippocampal representation proceeds "from encoding" to retrieval states. Therefore, our results provide new aspects of hippocampal representations: (i) neural fluctuations between encoding and retrieval states during a WM task and (ii) SWR as a switching representation from encoding to retrieval states.
\\
\indent
Moreover, our study reveals WM-task type specific directions between encoding- and retrieval-SWRs (Figure 7E--F). Specifically, eSWR and rSWR directed in the adverse direction not in Match IN but in Mismatch OUT task. These result might be explained by the memory engram theory \cite{liu_optogenetic_2012}. In fact, Match In task exposed subjects to once-seen letter, while Mismatch OUT task a novel letter which was not included in the encoding phase. These results suggest that SWR is related to working cognitive processess in humans.
\\
\indent
In conclusion, our study has demonstrated that hippocampal activity fluctuates between encoding and retrieval states during a WM task and exhibits a significant transition "from encoding" to retrieval during SWR periods.
\label{sec:discussion}
