%%%%%%%%%%%%%%%%%%%%%%%%%%%%%%%%%%%%%%%%%%%%%%%%%%%%%%%%%%%%%%%%%%%%%%%%%%%%%%%%
%% Introduction
%%%%%%%%%%%%%%%%%%%%%%%%%%%%%%%%%%%%%%%%%%%%%%%%%%%%%%%%%%%%%%%%%%%%%%%%%%%%%%%%
\section{Introduction}
Working memory (WM) is crucial in everyday life; however, its neural mechanism has yet to be fully elucidated. Specifically, the role of the hippocampus, an essential brain region guiding memory and spatial navigation, has been a topic of ongoing controversy \cite{scoville_loss_1957} \cite{squire_legacy_2009}  \cite{boran_persistent_2019} \cite{kaminski_persistently_2017} \cite{kornblith_persistent_2017} \cite{faraut_dataset_2018} \cite{borders_hippocampus_2022} \cite{li_functional_2023} \cite{dimakopoulos_information_2022}. Understanding the hippocampus' role in working memory is instrumental in deepening our knowledge of cognitive processes, ultimately aiding in developing cognitive training strategies and interventions.
\\
\indent
Among the hippocampal phenomena, a transient and synchronous oscillation called sharp-wave ripple (SWR) \cite{buzsaki_hippocampal_2015} is associated with various cognitive functions, including memory replay \cite{wilson_reactivation_1994} \cite{nadasdy_replay_1999} \cite{lee_memory_2002} \cite{diba_forward_2007} \cite{davidson_hippocampal_2009}, memory consolidation \cite{girardeau_selective_2009} \cite{ego-stengel_disruption_2010} \cite{fernandez-ruiz_long-duration_2019} \cite{kim_corticalhippocampal_2022}, memory recall \cite{wu_hippocampal_2017} \cite{norman_hippocampal_2019} \cite{norman_hippocampal_2021}, and neural plasticity \cite{behrens_induction_2005} \cite{norimoto_hippocampal_2018}. However, investigations into the effects of SWRs on working memory remain infrequent (\cite{jadhav_awake_2012} and limited to rodent models using navigation tasks, in which the precise timings of memory acquisition and recall are not separated.
\\
\indent
Hippocampal neurons may exhibit low-dimensional representations during WM tasks. For instance, the firing patterns of place cells \cite{okeefe_hippocampus_1971} \cite{okeefe_place_1976} \cite{ekstrom_cellular_2003} \cite{kjelstrup_finite_2008} \cite{harvey_intracellular_2009} in the hippocampus were found to be embedded within a dynamic, nonlinear 3D hyperbolic geometry while rodent navigating \cite{zhang_hippocampal_2022}. Furthermore, grid cells in the entorhinal cortex (EC) --- the primary gateway to the hippocampus \cite{naber_reciprocal_2001} \cite{van_strien_anatomy_2009} \cite{strange_functional_2014} --- exhibited toroidal topology during exploration \cite{gardner_toroidal_2022}. However, again, these experiments are limited to spatial navigation tasks in rodents so that the temporal resolution of WM tasks is constrained. Moreover, whther these findings are generalized to humans, especially other than navigation tasks, are not investigated yet.
\indent
\\
Given these backgrounds, in this study, we investigated the hypothesis that hippocampal neurons exhibit distinct representations in low-dimensional spaces as 'neural trajectory' during WM tasks, with a specific focus on SWR periods. To test this hypothesis, we utilized a dataset of patients performing an eight-second Sternbeug task with high temporal resolution (1 s for fixation, 2 s for encoding, 3 s for maintenance, and 2 s for retrieval) while their intrachranial electroencephalography signals (iEEG) in the medial temporal lobe (MTL) were recorded \cite{boran_dataset_2020}. To explore low-dimensional neural trajectories, we employed Gaussian-process factor analysis (GPFA) based on multiunit activities, a proven tool for the analysis of neural population dynamics \cite{yu_gaussian-process_2009}.
\label{sec:introduction}
